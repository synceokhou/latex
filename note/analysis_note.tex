%!TEX program = xelatex
\documentclass[UTF8]{ctexart}
\newcommand{\ud}{\,\mathrm{d}}
\usepackage{amsmath}
\usepackage{amssymb} % once we use this package, the \mathbb command is available
\DeclareMathOperator{\st}{s.t.}

\pagestyle{plain}
\begin{document}
\section{数列极限与函数极限}
\begin{itemize}
\item 在数学分析中的典型问题和方法P29,将$\sum \sin ka$分解的,我们可以在南开数分下P30见到
\begin{eqnarray*}
\left|\sum_{k=1}^n \sin kx \right| & = & \left|\frac{1}{\sin \frac{x}{2}}\sum_{k=1}^n \sin kx\sin \frac{x}{2} \right|\\
& = & \left|\frac{1}{\sin \frac{x}{2}}\sum_{k=1}^n \frac{1}{2}\left(\cos \left(k-\frac{1}{2}\right)x - \cos \left(k+\frac{1}{2}\right)x\right)\right|\\
& = & \left| \frac{\cos \frac{x}{2}-\cos(n+\frac{1}{2})x}{2\sin\frac{x}{2}}\right|\\
& \leq & \frac{1}{|\sin\frac{x}{2}|}
\end{eqnarray*}

\item 重要极限
\[
\lim_{n\to \infty} \left(1+\frac{1}{n} \right)^n = e
\]

\item 任一有界数列必有收敛子列

\item 常用的等价变换
\begin{eqnarray*}
x&\sim& \sin x \sim \tan x \sim \arctan x \sim \arcsin x\\
&\sim& \ln(1+x) \sim e^x-1\\
&\sim&\frac{a^x-1}{\ln a} \sim \frac{(1+x)^b-1}{b}
\end{eqnarray*}
还有$(1-\cos x)\sim \frac{1}{2}x^2$

\item $a>1,0<\alpha<1,k \in \mathbb{N},\st n\to \infty,\ln\ln n <\ln n < n^\alpha < n^k <a^n < n! < n^n $ 

\item Stirling公式
\[
n! = \sqrt{2\pi n} n^n e^{-n+\frac{\theta_n}{12n}},0\leq \theta_n \leq 1
\]

\item $\frac{\infty}{\infty}$型Stolz公式,设$\{x_n\}$严格递增,且$\lim_{n\to \infty}x_n=+\infty$,若
\[
\lim_{n\to \infty} \frac{y_n-y_{n-1}}{x_n-x_{n-1}} = a
\]
则
\[
\lim_{n\to \infty} \frac{y_n}{x_n} = a
\]

\item $\frac{0}{0}$型Stolz公式,设$n\to \infty ,y_n\to 0,\{x_n\}$严格单调下降趋向零,若
\[
\lim_{n\to \infty} \frac{y_n-y_{n-1}}{x_n-x_{n-1}} = a
\]
则
\[
\lim_{n\to \infty} \frac{y_n}{x_n} = a
\]

\item 变上限积分的求导
\begin{eqnarray*}
\phi(t) & = & \int_{\alpha (t)}^{\beta (t)} f(x,t) \ud x\\
\phi'(t) & = & \int_{\alpha (t)}^{\beta (t)} f_t'(x,t) \ud x + f(\beta (t),t)\beta' (t)- f(\alpha (t),t)\alpha' (t)
\end{eqnarray*}
\end{itemize}


\section{一元函数的连续性}
\begin{itemize}

\item 区间内,开闭都可以,函数单调,则每点处的单侧极限存在,这下看出来max和sup的区别了吧

\item 上确界定义,$\sup A$是$A$的上界,$\forall a\in A \Rightarrow a\leq \sup A$,另外再小一点就不是上界,$\forall \epsilon >0,\exists a_0\in A\Rightarrow a_0>\sup A-\epsilon$,max之类的只能在有限元素个数的集合使用

\item max、min和取绝对值等运算,保连续性,所以一些拼接函数可以用max、min表示,接着转换到绝对值表示

\item 遇上函数要证明无穷处的性质,可以通过变量代换把无穷区间映射为有限区间,例子可见数学分析中的典型问题和方法P152

\item 可去间断点:函数在该点左极限、右极限存在且相等,但不等于该点函数值或函数在该点无定义。

\item 跳跃间断点:函数在该点左极限、右极限存在,但不相等。

\item 无穷间断点:函数在该点可以无定义,且左极限、右极限至少有一个不存在,且函数在该点极限为$\infty$。

\item 振荡间断点:函数在该点可以无定义,当自变量趋于该点时,函数值在两个常数间变动无限多次。

\item 可去间断点和跳跃间断点称为第一类间断点。其它间断点称为第二类间断点。第一类间断点的左右极限都存在,而第二类间断点的左右极限至少有一个不存在。
\end{itemize}

\section{一元微分学}
\begin{itemize}
\item 可微性的证明就是极限
\[
f'(x_0) = \lim_{x\to x_0} \frac{f(x)-f(x_0)}{x-x_0} = \lim_{h\to 0} \frac{f(x_0+h)-f(x_0)}{h}
\]
存在性的证明

\item 要证明不可微性,可以证$f(x)$在$x_0$不连续,或者$f_+'(x_0)\neq f_-'(x_0)$,或$x$以不同的方式趋向于$x_0$时,上面的极限取不同的值

\item $(u\cdot v)^{(n)} = \sum_{k=0}^n \text{C}_n^k u^{(k)}v^{(n-k)}$

\item 数学分析中的典型问题和方法P216,零点存在性问题,借助介值性求解(连续函数有介值性、导函数也有介值性),或者借助Rolle定理求解

\item 式子中同时出现$f(x)$和$f'(x)$,就要想到要用$e^x$来构造$F(x)$,比如数学分析中的典型问题和方法P218

\item 例3.2.16和3.2.9有同样的转移变量方法,目的是为了求导之后能得到目标微分中值公式

\item 导数无第一类间断,其实很显然,若某点第一类间断,即左极限不等于右极限,此时极限不存在,即该点不可导

\item Darboux定理,若函数$f(x)$在区间$[a,b]$上处处可导,$f'(a)<f'(b)$,则$\forall c,f'(a)<c<f'(b),\exists \xi \in (a,b),\st f'(\xi)=c$

\item
\[
f(x) = \sum_{k=0}^n \frac{1}{k!}f^{(k)}(x_0)(x-x_0)^k + R_n(x)
\]
其中$R_n(x)$可取拉格朗日余项
\[
R_n(x) = \frac{1}{(n+1)!}f^{(n+1)}(\xi)(x-x_0)^{n+1}
\]
也可以取Peano余项$R_n(x)=o((x-x_0)^n)$

\item 例3.2.9和例3.3.1都有共同特点,要证某中值定理组合式,通过设最高导数次项为一值,再证存在最高导数等于该值

\item P267的习题3.3.5能不能用中值定理?二阶导存在,证明可求二阶导,中值定理的条件是什么?

\item P296的习题3.4.18是非常经典的分离系数法,取值可取极致,由于分离出来的函数单调,直接取边界

\item P300的例3.5.3,$f^{(k)}(x_0)=0(k=1,2,\cdots,n-1),f^{(n)}(x_0)\neq0$,此时直接泰勒展开,可知函数值由$f^{(n)}(x_0)$决定

\item P282凸性的定理3的推论5,若$f(x)$在区间$I$上为凸的,则$f$在任一内点$x\in I^o$上连续

\item P289的例3.4.9,若$f(x)$在区间$I$上为凸的,则$f$在$I$任一闭子区间上有界

\end{itemize}

\section{一元积分学}
\begin{itemize}
\item 在P350的例4.3.6和P360的例4.3.20都有在区间$[a,b]$中间做泰勒展开以后积分为零的现象

\item 一些特殊的可积条件,如P366的例子4.3.27,$f(x)$为$[0,+\infty)$上的凸函数,因此它在$(0,+\infty)$内连续,$f(x)$在$[0,x]$上有界,此时积分有意义

\item 一些特殊的可积条件,如P367的例子4.3.28,$g(x)$递增

\item P376的4.3.7做不出来,和4.3.21做法相同

\item Cauchy-Schwarz不等式的证法,即变成一个与$t$相关的二次函数再求判别式,可以逆着用即已知判别式反过来构造二次函数

\item Cauchy-Schwarz不等式的特点是把积分号内的平方拿到积分外
\[
\left(\int_a^b f(x)g(x)\ud x \right)^2 \leq \int_a^b f^2(x)\ud x \int_a^b g^2(x)\ud x
\]

\item 反常积分,如果函数非负,只需证其积分无论在哪个区间有界即可,即$F(x)$单增,有界必有极限

\item P436的例4.5.35中求极限用了一次洛必达之后,极限不存在,要用另外一种方式比如分部积分法求极限

\item P442的例4.5.41,积分中函数的两个位置相减可以做区间替换

\end{itemize}

\section{多元微分学}
\subsection{极限存在与连续等定理}
\begin{itemize}
\item 证明二元极限不存在,证径向路径的极限与幅角有关;证特殊路径极限不存在;证两个特殊路径极限不相等;在空心邻域里连续,但二累次极限存在不相等

\item 多元连续还是按定义证,就是两点距离小于$\epsilon$,然后证函数值小于$\delta$

\item 对微分方程做变量替换,要画出依赖关系,最难的是因变量和自变量都变,见P668

\item 可微性,公式见P671

\item 空间曲线$x=x(t),y=y(t),z=z(t)$的切向量是$(x'(t),y'(t),z'(t))$,曲面$F(x,y,z)=0$的法向量是$(F_x',F_y',F_z')$

\item 多元极值判定,Hessian矩阵正定取极小,负定取极大,不定无极值

\item 条件极值与Lagrange法,用Lagrange函数的二阶微分进行判断,大于0取极小

\item 隐函数存在P733
\[
y_x'=-\frac{F_x'(x,y)}{F_y'(x,y)}
\]
\end{itemize}

\subsection{题目}
\begin{itemize}
\item 在P683有一堆求多元函数连续可微的例题
\end{itemize}

\section{重积分}
\begin{itemize}
\item 对于积分$I=\iint_D f(x,y)\ud x\ud y$,作变换$x=x(u,v),y=y(u,v)$,找出变换后区域$D'=\{(u,v):a\leq u\leq b,\phi(u)\leq v\leq \psi(u)\}$
\[
I=\int_a^b\int_{\phi(u)}^{\psi(u)} f(x(u,v),y(u,v))|J|\ud v
\]
其中,雅各比还可以取逆
\begin{eqnarray*}
J&=&\frac{\partial(x,y)}{\partial(u,v)}\\
&=& \left(
\begin{array}{cc}
\frac{\partial x}{\partial u}&\frac{\partial x}{\partial v}\\
\frac{\partial y}{\partial u}&\frac{\partial y}{\partial v}\\
\end{array}
\right)\\
J^{-1}&=&\frac{\partial(u,v)}{\partial(x,y)}
\end{eqnarray*}

\item 极坐标系下二重积分
\begin{eqnarray*}
\iint_D f(X)\ud \sigma&=&\iint_D f(x,y)\ud x\ud y\\
&=&\iint_D f(r\cos \theta,r\sin \theta)r\ud r\ud \theta\\
&=&\int_\alpha^\beta \ud \theta \int_{r_1(\theta)}^{r_2(\theta)} f(r\cos \theta,r\sin \theta)r\ud r
\end{eqnarray*}

\item 三重积分两种方法,$V=\{(x,y,z)|(x,y)\in D,z_1(x,y)\leq z \leq z_2(x,y) \}$
\[
\iiint_v f(x,y,z)\ud V = \iint_D \ud x\ud y\int_{z_1(x,y)}^{z_2(x,y)} f(x,y,z)\ud z
\]
或者$V=\{(x,y,z)|a\leq z \leq b,(x,y)\in D_z \}$
\[
\iiint_v f(x,y,z)\ud V = \int_a^b \ud z \iint_{D_z}f(x,y,z)\ud x\ud y
\]

\item P882三重线性坐标系旋转变换,将$Oxy$旋转到平面$ax+by+cz=0$
\[
\zeta = \frac{ax+by+cz}{a^2+b^2+c^2}
\]
此时$x$轴与$y$轴被旋转到$\zeta=0$的平面内,把它们记为$\xi$轴与$\eta$轴,$|J|=1$

\item 球坐标变换,$x=r\sin\phi\cos\theta,y=r\sin\phi\sin\theta,z=r\cos\phi,J=r^2\sin\phi$

\item 广义球坐标变换,$x=ar\sin\phi\cos\theta,y=br\sin\phi\sin\theta,z=cr\cos\phi,J=abcr^2\sin\phi$

\end{itemize}


\section{曲线曲面积分}
\begin{itemize}

\item 一型曲线积分
\[
\int_L f(x,y)\ud s = \int_a^b f(x(t),y(t))\sqrt{x'^2(t)+y'^2(t)}\ud t
\]

\item P931解释了二型曲线积分的对称性,总的来说就是不单止考虑函数符号,还要考虑到积分方向投影到$\ud x$上的正负

\item P942二型变一型
\[
\int_L P\ud x+ Q\ud y+ R\ud z = \int_L (P\cos\alpha + Q\cos \beta + R\cos \gamma)\ud s
\]
接着用柯西不等式,后面直接变1

\item Green公式
\[
\int_{L^+} P\ud x+ Q\ud y = \iint_D (\frac{\partial Q}{\partial x}-\frac{\partial P}{\partial y})\ud x \ud y
\]
此外还有面积公式
\[
S = \iint_D\ud x\ud y = \int_{L^+}x\ud y = - \int_{L^+}y\ud x = \frac{1}{2} \int_{L^+}x\ud y-y\ud x
\]

\item 在P949、P969有一型变二型的$\ud s$变$\ud{x}$和$\ud{y}$的方法
\[
\int_L [P\cos(t,x) + Q\cos(t,y)] \ud s = \int_{L^+} P\ud x+ Q\ud y 
\]
其中$(t,x)$和$(t,y)$分别表示$x$轴正向,$y$轴正向与动点切线正向的夹角。另外也有用外法向量的
\[
\frac{\partial u}{\partial n} = \frac{\partial u}{\partial x}\cos(n,x)+\frac{\partial u}{\partial y}\cos(n,y)
\]
从而,但是观察与上面区别
\[
\frac{\partial u}{\partial n} \ud s = \frac{\partial u}{\partial x} \ud y - \frac{\partial u}{\partial y} \ud x
\]
\item P969非常清晰地解释了向量的运算
\begin{eqnarray*}
\mathbf{n_1} &=& (\cos(n,x),\cos(n,y))\\
\mathbf{l_1} &=& (\cos(l,x),\cos(l,y))\\
\end{eqnarray*}
其中$\mathbf{n_1}$表示$\mathbf{n}$的单位向量
\[
\cos(l,n) = \mathbf{n_1}\cdot \mathbf{l_1}
\]
接下来可以用上面的

\item 利用公式计算第一型曲面积分,首先选取平面,使便于求$S$的投影区域$\Delta$,$S:z=z(x,y),(x,y)\in \Delta$
\[
I = \iint_S f(x,y,z)\ud S = \iint_\Delta f(x,y,z(x,y))\sqrt{1+z_x'^2+z_y'^2}\ud x\ud y
\]

\item P981向坐标面投影不方便,可以做新的,如$S$为$x+y+z=t$被$x^2+y^2+z^2\leq 1$截取部分,作坐标轴旋转,令以下,并在$w=0$的平面上,任取二正交轴,使$x^2+y^2+z^2=u^2+v^2+w^2$
\[
w=\frac{x+y+z}{\sqrt{3}}
\]
接着直接变坐标轴,投影,计算,无需增加$J$

\item 用参数方程,P982,特别的$S$为球面$\ud S=\sqrt{EG-F^2}\ud \phi\ud \theta = R^2\sin\phi \ud \phi $

\item 两种曲面积分的关系,其中$\vec{n} = (\cos \alpha,\cos\beta,\cos\gamma)$是$\partial\Omega$的单位外法向量
\begin{eqnarray*}
\iint_{\partial\Omega^+}P\ud y\ud z + Q\ud z\ud x+ R\ud x\ud y &=& \iint_{\partial\Omega}(P\cos \alpha + Q\cos \beta+ R\cos \gamma)\ud S \\
&=& \iiint_{\Omega}\left(\frac{\partial P}{\partial x}+\frac{\partial Q}{\partial y}+\frac{\partial R}{\partial z}\right)\ud x\ud y\ud z
\end{eqnarray*}

\item Stock公式,其中$\vec{n} = (\cos \alpha,\cos\beta,\cos\gamma)$是$S$的单位法向量
\begin{eqnarray*}
\int_{\partial S}P\ud x+Q\ud y+R\ud z&=&\iint_S\left|
\begin{array}{ccc}
\ud y\ud z&\ud z\ud x&\ud x\ud y\\
\frac{\partial}{\partial x}&\frac{\partial}{\partial y}&\frac{\partial}{\partial z}\\
P&Q&R\\
\end{array}
\right| \\
&=& \iint_S\left|
\begin{array}{ccc}
\cos \alpha&\cos\beta&\cos\gamma\\
\frac{\partial}{\partial x}&\frac{\partial}{\partial y}&\frac{\partial}{\partial z}\\
P&Q&R\\
\end{array}
\right|
\end{eqnarray*}

\item P997径向与法向的夹角
\[
\cos(n,r) = \frac{\mathbf{r}}{r}\cdot \mathbf{n} = \frac{x}{r}\cos \alpha +\frac{y}{r}\cos\beta +\frac{z}{r}\cos \gamma
\]

\item P1007习题7.4.3有如何求某平面的单位法向量,如$F(x,y,x)=0$,则求$(F_x',F_y',F_z')$,接着单位化即可

\item P1013法向量和径向向量的夹角$\rho = r \cos\alpha = r\cdot\mathbf{r_1}\cdot\mathbf{n_1}$,其中$\mathbf{r_1}$是径向向量,$\mathbf{n_1}$是法向量

\item P1019有经典的多元积分中值得到某点为零然后缩小半径趋向某聚点来证零,即P1000的从积分性质导出微分性质

\end{itemize}

\section{广义积分}
\subsection{广义积分敛散性的判定}
\begin{itemize}
\item 分两种,一种是$x$跑无穷的,另一种是跑奇点的
\item 另外还分正的,与变号的,详见P413
\end{itemize}


\section{数项级数}
\subsection{正项级数收敛性判断}
\begin{itemize}
\item Cauchy准则
\item 正项级数判阶法,相对于$\frac{1}{n}$比较,可结合等价无穷小、洛必达、带Peano的Taylor
\item 正项级数比较法及极限形式
\[
\lim_{n\to\infty} \frac{a_n}{b_n} = r
\]
\item D'Alembert法,Cauchy根式法
\item 正项级数收敛的Cauchy积分判别法,若$f(x)>0$,在$[1,+\infty)$上递减,则$\sum_{n=1}^\infty f(n)$与广义积分$\int_1^{+\infty}f(x)\ud x$同时敛散
\item 部分和$\sum_{k=1}^n a_k$有界
\end{itemize}

\subsection{变号级数收敛性判断}
\begin{itemize}
\item 对$\sum |a_n|$用D'Alembert法,Cauchy根式法,若收敛则绝对收敛,若$\sum |a_n|$发散,则$a_n \nrightarrow 0$从而$\sum a_n$发散
\item Leibniz,$a_n\geq 0$,单调下降趋向零,则$\sum (-1)^{n-1}a_n$收敛
\item Abel,$\sum a_n$收敛,$\{b_n\}$单调有界
\item Dirichlet,$\sum a_n$有界,$\{b_n\}$单调趋向零
\item Cauchy准则
\item 条件收敛是$\sum a_n$收敛,$\sum|a_n|$发散
\item Abel变换,$\sum_{k=1}^m a_kb_k =a_mB_m + \sum_{k=1}^{m-1}(a_k-a_{k-1})B_k$
\end{itemize}

\subsection{题目}
\begin{itemize}
\item P452关于$\cos$求和的变形
\begin{eqnarray*}
\sum_{k=1}^n \cos kx &=& \frac{1}{2\sin\frac{x}{2}} \sum_{k=1}^n2\sin\frac{x}{2}\cos kx\\
&=& \frac{1}{2\sin\frac{x}{2}} \sum_{k=1}^n \left( \sin(k+\frac{1}{2})x-\sin(k-\frac{1}{2})x\right)\\
&=& \frac{1}{2\sin\frac{x}{2}} (\sin \frac{2n+1}{2}x-\sin\frac{x}{2})
\end{eqnarray*}

\item P468的例5.1.35体会和Abel变换的区别,这是中途截断的Abel变换

\item P487的5.1.7综合了柯西证发散和有级数,涉及不等式放缩

\item P488的5.1.19,单项转求和,再用Stolz

\end{itemize}

\section{函数项级数}
\subsection{一致收敛的判断}
\begin{itemize}
\item $\epsilon-N$,求$S(x)=\sum_{n=1}^\infty u(x)$,找与$x$无关的$N$,使得$n>N$有$|S(x)-S_n(x)|<\epsilon$,最重要是把$x$的$\epsilon-N$移到$n$上来

\item 放大$|S(x)-S_n(x)|<\alpha_n$,且$\alpha_n\to 0$,技巧有通过已知不等式、求极值(确界法?)、用已知的余项估计
\item 确界法$\lim_{n\to \infty}\sup_{x\in I}|S(x)-S_n(x)|=0$
\item 用Cauchy准则,优点是不需$S(x)$,重要的是将片段变形,利用已知的收敛
\item M判别法,求$u(x)$在区间的最大值,利用已知不等式,Taylor、微分中值变形
\item Abel,$\sum_{n=1}^\infty a_n(x)$在$I$上一致收敛,$\{b_n(x)\}$一致有界,对固定$x$关于$n$单调
\item Dirichlet,$\sum_{n=1}^\infty a_n(x)$在$I$上一致有界,$\{b_n(x)\}$一致收敛于零,对固定$x$关于$n$单调
\end{itemize}

\subsection{一致收敛的性质}
\begin{itemize}
\item 逐项取极限即极限号(有限点)与求和号可交换
\item 和函数的连续性,$S(x)$在$I$上连续,在$x_0$处连续,在$(a,b)$内闭一致收敛则在$(a,b)$内连续
\item 可微与逐项求导要验证三个条件,$\sum u_n(x)$在$I$上收敛(或者至少有一个收敛点),$u_n(x)$在$I$上有连续导数,$u_n'(x)$在$I$上一致收敛(或在$I$上内闭一致收敛)
\item 逐项积分即积分与求和可交换,$\sum u_n(x)$在$I$上一致收敛,且每个都可积,则$S(x)$也可积
\item 若不满足以上条件,又要证明积分与求和可交换,则要验证
\[
\lim_{n\to\infty}\int_a^b \sum_{k=n+1}^\infty u_k(x)\ud x=0
\]
\end{itemize}
\subsection{题目}
\begin{itemize}
\item P492的例5.2.1很经典

\item P542的5.2.3用到了之前的分$\epsilon$的积分方法

\item P505有Abel变换,P488的5.1.19,P470的例5.1.40

\item P499有Leibniz有,$|S(x)-S_{n-1}(x)|=|r_n(x)| \leq a_{n}(x)$只要在对$a_{n}(x)$取最大

\item P532的例5.2.48非常清晰地阐述了求导的过程
\end{itemize}

\section{幂级数}
\subsection{幂级数的收敛半径和收敛范围}
\begin{itemize}
\item
\[
R=\frac{1}{\lim_{n \to \infty}\sqrt[n]{|a_n|}}
\]
分母为0时,$R=+\infty$,分母为正无穷时,$R=0$
\item
\[
R=\lim_{n\to \infty}\frac{|a_n|}{|a_{n+1}|}
\]
求了收敛区间,要验证端点的敛散性
\item 利用收敛半径求极限,其实就是上面第一个式子反过来用

\item 初等函数展开为幂级数,可用已知的展开,用逐项积分或微分,计算指定点各阶导数,然后Taylor,用级数的运算

\item 上面的逐项积分或微分只能在区间内进行,但是要验证在端点的情况,见P559

\item Taylor的方法,先求各阶导,再求半径,在半径内把$\sim$变成$=$即判断解是否满足微分方程(用上解的唯一性),最后验证端点收敛,见P561的例5.3.15

\item 求和问题,用逐项求导或积分,构造方程式然后解,用Abel第二定理计算

\item 根据Abel第二定理,求$\sum_{n=1}^\infty a_n$,先求$\sum_{n=1}^\infty a_nx^n$在$(-1,1)$内的$S(x)$然后$x\to 1_-$,即$\sum_{n=1}^\infty a_n=\lim_{x\to 1_-}S(x)$,而$S(x)$用上法求出

\item 将被积函数展开为幂级数,然后逐项积分,其中有可能用到函数项级数积分

\end{itemize}

\subsection{题目}
\begin{itemize}
\item P556的例5.3.10有$(1-x)\sum_{n=1}^\infty(a_1+a_2+\cdots+a_n)x^n=\sum_{n=1}^\infty a_nx^n$,同理P554的$(1-x)\sum_{n=0}^\infty S_nx^n=\sum_{n=0}^\infty a_nx^n$,而且乘上$(1-x)$都使得半径小了

\item P576的两道题目,在幂级数的基础上交换求和与积分、极限,典型题目
\end{itemize}


\section{幂级数展开}
\begin{itemize}
\item 在$\mathbf{R}$上
\[
\sin x =\sum_{n=0}^\infty (-1)^n \frac{x^{2n+1}}{(2n+1)!}
\]

\item 在$\mathbf{R}$上
\[
\cos x =\sum_{n=0}^\infty (-1)^n \frac{x^{2n}}{(2n)!}
\]

\item 在$\mathbf{R}$上
\[
e^x=\sum_{n=0}^\infty \frac{x^n}{n!}
\]

\item 在$(-1,1]$上
\[
\ln(1+x)=\sum_{n=1}^\infty (-1)^{n-1}\frac{x^n}{n}
\]

\item 在$(-1,1)$上
\[
(1+x)^\alpha = 1+ \sum_{n=1}^\infty \frac{\alpha(\alpha-1)\cdots(\alpha-n+1)}{n!}x^n
\]
\end{itemize}

\section{泰勒展开}
\begin{eqnarray*}
\sin x &=&\sum_{k=0}^n (-1)^k \frac{x^{2k+1}}{(2k+1)!}+(-1)^{n+1} \frac{\cos\theta x}{(2n+3)!} x^{2n+3}\\
&=& x-\frac{x^3}{3!}+\frac{x^5}{5!}+o(x^5)\\
\cos x &=&\sum_{k=0}^n (-1)^k \frac{x^{2k}}{(2k)!} + (-1)^{n+1} \frac{\cos\theta x}{(2n+2)!}x^{2n+2}\\
&=& 1-\frac{x^2}{2!}+\frac{x^4}{4!}+o(x^4)\\
e^x&=&\sum_{k=0}^n \frac{x^k}{k!}+\frac{e^{\theta x}}{(n+1)!}x^{n+1}\\
&=& 1+\frac{x}{1!}+\frac{x^2}{2!}+\frac{x^3}{3!}+o(x^3)\\
\ln(1+x)&=&\sum_{k=1}^n (-1)^{k-1}\frac{x^k}{k}+(-1)^n\frac{x^{n+1}}{(n+1)(1+\theta x)^{n+1}}\\
&=& x-\frac{x^2}{2}+\frac{x^3}{3}+o(x^3)\\
(1+x)^\alpha &=& 1+ \sum_{k=1}^n \frac{\alpha(\alpha-1)\cdots(\alpha-k+1)}{k!}x^k+\frac{\alpha(\alpha-1)\cdots(\alpha-n)(1+\theta x)^{\alpha-n-1}}{(n+1)!}x^{n+1}\\
&=& 1+\frac{a}{1!}x+\frac{a(a-1)}{2!}x^2+\frac{a(a-1)(a-2)}{3!}x^3+o(x^3)\\
\frac{1}{1-x} &=& 1+x+x^2+x^3+o(x^3)\\
\arctan x &=& x-\frac{x^3}{3}+\frac{x^5}{5}-\cdots+(-1)^n\frac{x^{2n+1}}{2n+1}+o(x^{2n+1})\\
\arcsin x &=& x+\frac{1}{3}\cdot\frac{1}{2!!}x^3+\frac{1}{5}\cdot\frac{3!!}{4!!}x^5+\cdot+\frac{1}{2n+1}\cdot\frac{(2n-1)!!}{(2n)!!}x^{2n+1}+o(x^{2n+1})
\end{eqnarray*}

\section{傅立叶级数}
\begin{itemize}
\item $f(x)$在$(-\infty,+\infty)$上展开成一致收敛的三角级数
\[
	S(x) = \frac{a_0}{2}+\sum_{n=1}^\infty(a_n \cos nx+ b_n \sin nx)
\]
在$[- \pi,\pi]$上逐项积分
\begin{eqnarray*}
a_n &=& \frac{1}{\pi} \int_{-\pi}^{\pi} f(x)\cos nx\ud x, n=0,1,2,\cdots\\
b_n &=& \frac{1}{\pi} \int_{-\pi}^{\pi} f(x)\sin nx\ud x, n=1,2,\cdots
\end{eqnarray*}

\item $f(x)$的傅立叶级数$S(x)$在点$x_0$收敛到
\[
\frac{f(x_0+0)+f(x_0-0)}{2}
\]
\end{itemize}


\section{积分常用}
\begin{eqnarray*}
\int \frac{1}{x}\ud x &=& \ln|x|+C \\
\int a^x \ud x&=&\frac{a^x}{\ln a}+C \\
\int \sec^2 x\ud x &=& \tan x+C \\
\int \csc^2 x\ud x &=& -\cot x+C \\
\int \frac{\ud x}{\sqrt{1-x^2}} &=&\arcsin x+C = -\arccos x+C \\
\int \frac{\ud x}{1+x^2} &=& \arctan x + C \\
\int \frac{\ud x}{a^2+x^2}&=&\frac{1}{a} \arctan\frac{x}{a} +C \\
\int \frac{\ud x}{a^2-x^2}&=&\frac{1}{2a} \ln\left|\frac{a+x}{a-x}\right| +C \\
\int \frac{\ud x}{\sqrt{a^2-x^2}}&=&\arcsin\frac{x}{a} +C \\
\int \frac{\ud x}{\sqrt{x^2\pm a^2}}&=&\ln|x+\sqrt{x^2\pm a^2}| +C \\
\int \sqrt{a^2-x^2}\ud x&=&\frac{x}{2}\sqrt{a^2-x^2}+\frac{a^2}{2}\arcsin\frac{x}{a} +C \\
\int \sqrt{x^2\pm a^2}\ud x&=&\frac{x}{2}\sqrt{x^2\pm a^2}\pm \frac{a^2}{2}\ln|x+\sqrt{x^2\pm a^2}| +C \\
\int \tan x\ud x&=&-\ln|\cos x|+C\\
\int \cot x\ud x&=&\ln|\sin x|+C\\
\int \sec x\ud x&=&\ln|\sec x+\tan x|+C\\
\int \csc x\ud x&=&-\ln|\csc x-\cot x|+C\\
\end{eqnarray*}

\section{微分常用}
\begin{eqnarray*}
(a^x)'&=&a^x\ln a\\
(\log_a |x|)' &=& \frac{1}{x\ln a}\\
(\ln |x|)' &=& \frac{1}{x}\\
(\tan x)' &=& \sec^2 x\\
(\cot x)' &=& -\csc^2 x\\
(\sec x)' &=& \tan x\sec x\\
(\csc x)' &=& \cot x\csc x\\
(\arcsin x)' &=& \frac{1}{\sqrt{1-x^2}}\\
(\arccos x)' &=& -\frac{1}{\sqrt{1-x^2}}\\
(\arctan x)' &=& \frac{1}{1+x^2}\\
\end{eqnarray*}


\section{和差化积积化和差}
\begin{eqnarray*}
\sin a \cos b&=&\frac{[\sin(a+b)+\sin(a-b)]}{2} \\
\cos a \sin b&=&\frac{[\sin(a+b)-\sin(a-b)]}{2} \\
\cos a \cos b&=&\frac{[\cos(a+b)+\cos(a-b)]}{2} \\
\sin a \sin b&=&-\frac{[\cos(a+b)-\cos(a-b)]}{2} \\
\sin x+\sin y&=&2\sin(\frac{a+b}{2})\cos(\frac{a-b}{2})\\
\sin x-\sin y&=&2\cos(\frac{a+b}{2})\sin(\frac{a-b}{2})\\
\cos x+\cos y&=&2\cos(\frac{a+b}{2})\cos(\frac{a-b}{2})\\
\cos x-\cos y&=&-2\sin(\frac{a+b}{2})\sin(\frac{a-b}{2})\\
\end{eqnarray*}

\section{复数与$e^x,\sin,\cos$}
\begin{eqnarray*}
\binom{k+1}{i}&=&\binom{k}{i-1}+\binom{k}{i}\\
e^{i\theta} &=& \cos\theta +i\sin\theta\\
\sin \theta &=& \frac{e^{i\theta}-e^{-i\theta}}{2i}\\
\cos \theta &=& \frac{e^{i\theta}+e^{-i\theta}}{2}\\
\sin 3x &=& 3\sin x-4\sin^3 x\\
\cos 3x &=& 4\cos^3 x-3\cos x\\
\arctan\left(\frac{A+B}{1-AB}\right) &=& \arctan A+\arctan B
\end{eqnarray*}
\section{空间向量与解析几何}
\begin{eqnarray*}
|\mathbf{a}\times \mathbf{a}|&=&|\mathbf{a}|\cdot|\mathbf{a}|\cdot \sin \theta\\
\text{rot}\mathbf{A} & = & \nabla \times \mathbf{A}\\
\text{div}\mathbf{A} & = & \nabla \cdot \mathbf{A}
\end{eqnarray*}






















\end{document}