%!TEX program = xelatex
\documentclass[UTF8]{ctexart}
\pagestyle{plain}
\newcommand{\ud}{\,\mathrm{d}}
\usepackage{amsmath}
\usepackage{amssymb} % once we use this package, the \mathbb command is available
\DeclareMathOperator{\Ima}{Im}
\DeclareMathOperator{\tr}{tr}
\DeclareMathOperator{\diag}{diag}
\DeclareMathOperator{\rank}{rank}
\DeclareMathOperator{\res}{res}

\pagestyle{plain}
\begin{document}

\section{多项式}
\begin{itemize}
\item 若多项式$f=gh$,其中$f,g,h$都不为常数,则$f$可约,否则不可约

\item $p(x)$是不可约多项式$\Leftrightarrow \forall f(x)$,有$p(x)|f(x)$或$(p(x),f(x))=1$

\item 解题方法P23的16,取模后可以乘式子

\item 解题方法P25的20没看懂可以参考高等代数下P47,还可以看23题,这个deg运算要加强

\item 解题方法很少,Bezout等式,带余除法,重因式即$(f,f'),不可约$

\item 解题方法P29,还有用拉格朗日插值公式的

\item 解题方法P33,根与系数的关系,高阶版

\item 解题方法P60的17,其实是左右移项,两边都配出$x-1$

\item $f(x)=a_nx^n+a_{n-1}x^{n-1}+\cdots+a_0$是整系数多项式,而$\frac{r}{s}$是一个有理根,其中$r,s$互素,必有$s|a_n,r|a_0$,如果$f(x)$的首项系数$a_n=1$,那么有理根都是整根,而且是$a_0$的因子。比如$f(x)=x^3-5x+1$,若在有理域可约,至少有有一个一次因子,也即有一个有理根,但是$f(x)$的有理根只可能是$\pm 1$,直接算出,不是根
\end{itemize}

\section{行列式}
\begin{itemize}
\item Binet-Cauchy,设$A$与$B$分别为$n\times s$与$s\times n$矩阵,则当$n<s$
\[
\sum_{1\leq k_1<k_2<\cdots<k_n\leq s}\det A\left(
\begin{array}{cccc}
1 & 2 & \cdots & n\\
k_1 & k_2 & \cdots & k_n
\end{array}
\right)\cdot \det B\left(
\begin{array}{cccc}
k_1 & k_2 & \cdots & k_n\\
1 & 2 & \cdots & n
\end{array}
\right)
\]
当$n>s$时为0

\item 解题方法P66,降阶法,加边法,分项拆开找递推公式,$\det(AB)=\det(A)\cdot\det(B)$,$\det(I_n\pm AB)=\det(I_m\pm BA)$

\item 解题方法P73的(3)第二种方法用$\det(I_n\pm AB)=\det(I_m\pm BA)$,观察力

\item 解题方法P79的26,P84的(3),P87的40和41加边,加边的特点是发现共同行列部分

\item 解题方法P93的2,用$\det(AB)=\det(A)\cdot\det(B)$,二项式展开之后有深刻研究

\end{itemize}

\section{线性方程组}
\subsection{方阵的简单性质}
\begin{itemize}
\item 如果存在方阵$B$,使得$AB=BA=I$,则方阵$A$可逆
\item $(A^T)^{-1}=(A^{-1})^T,(AB)^{-1}=B^{-1}A^{-1}$
\item 齐次与非齐次方程组有解的条件
\item 向量组线性无关的证明方法
\end{itemize}
\subsection{题目}
\begin{itemize}
\item P122的32、33、34都要注意,非常普通的题目,计算准确
\item P131的1和P116的15很像
\end{itemize}

\section{矩阵的运算}
\subsection{一些有用的定律和性质}
\begin{itemize}
\item 初等方阵,左乘是行变化,右乘是列变换
\[
P = \left(
\begin{array}{ccccc}
0 & 0 & 0 & 0 & 0 \\
0 & 0 & 0 & a_{ij} & 0 \\
0 & 0 & 0 & 0 & 0 \\
0 & 0 & 0 & 0 & 0 \\
0 & 0 & 0 & 0 & 0 \\
\end{array}
\right)
\]
上面的意思是第$i$行,第$j$列为1,那么$PA$相当于,把$A$的$j$行搬到了$i$行去;$AP$相当于把$A$的$i$列搬到了$j$列去
\item $C$列满秩$\Leftrightarrow Cx=0$只有零解 $\Leftrightarrow X^TC=0$,其中$X$为列满秩;同理,$R$行满秩$\Leftrightarrow Rx=b$总有解 $\Leftrightarrow RY=I$

\item $\phi_A$为单射$\Leftrightarrow A$为列独立阵;$\phi_A$为满射$\Leftrightarrow A$为行独立阵;$\phi_A$为双射$\Leftrightarrow A$为可逆阵。于是在同一空间的线性变换有,单射$\Leftrightarrow$满射$\Leftrightarrow$双射

\item 若$M=I_n-\alpha\beta^T$,$\alpha,\beta$为列向量,则
\[
M^{-1} = (I_n-\alpha\beta^T)^{-1} = I_n +\frac{\alpha\beta^T}{1-\beta^T\alpha}
\]

\item $B$可逆,若$M=B-\alpha\beta^T$可逆
\[
M^{-1} = (B-\alpha\beta^T)^{-1} = \left(I_n +\frac{B^{-1}\alpha\beta^T}{1-\beta^TB^{-1}\alpha}\right)B^{-1}
\]
\item 矩阵的秩的公式P144,在P158,P173有证
\item 广义逆的公式,满足$AXA=A$的矩阵$X$称为$A$的广义逆,记为$A^-$,若$A_{m\times n}$则$A_{n\times m}^-$;若左,则右
\[
A=P\left(
\begin{array}{cc}
I_r & 0\\
0 & 0\\
\end{array}
\right)Q,\qquad A^- = Q^{-1}\left(
\begin{array}{cc}
I_r & Y_2\\
Y_3 & Y_4\\
\end{array}
\right)
\]
\end{itemize}
\subsection{题目}
\begin{itemize}
\item P152的17,解决求逆求伴随求转置可换序
\item P155的29,用了Cauchy不等式
\item P156的31,秩为$r$的矩阵可以表示为$r$个秩为$1$的矩阵的和
\item P175的8,之前都有很多类似的可以找一下总结规律
\item P177的9,高代原题
\end{itemize}


\section{线性空间}
\subsection{定义及定理}
\begin{itemize}
\item 线性空间的定义,对加法满足4条,对乘法满足4条

\item $W$是$V$的子空间$\Leftrightarrow W$对加法和数乘封闭

\item 线性函数满足$\phi(\alpha+\beta)=\phi(\alpha)+\phi(\beta),\phi(\lambda\alpha)=\lambda\phi(\alpha)$

\item $(\eta_1,\cdots,\eta_n)=(\epsilon_1,\cdots,\epsilon_n)A$,则$A$是$\epsilon_1,\cdots,\epsilon_n$到$\eta_1,\cdots,\eta_n$的过渡矩阵,设$\alpha\in V$,在$\epsilon_1,\cdots,\epsilon_n$下的坐标为$x=(x_1,\cdots,x_n)^T$,在$\eta_1,\cdots,\eta_n$下的坐标为$x=(y_1,\cdots,y_n)^T$,则有如下的坐标变换公式$y=A^{-1}x$

\item 求两个子空间$W_1$与$W_2$的和与交,设$W_1=(\alpha_1,\alpha_2,\cdots,\alpha_s)\in V_n(F),W_2=(\beta_1,\cdots,\beta_t)\in V_n(F)$,则$W_1+W_2=(\alpha_1,\alpha_2,\cdots,\alpha_s,\beta_1,\cdots,\beta_t)$,求极大线性无关组,即初等行变化,然后$r$阶子式所在的列向量为基

\item $\alpha\in W_1\cap W_2$,有$\alpha\in W_1$且$\alpha\in W_2$,可求解$\lambda_1\alpha_1+\cdots+\lambda_{s_1}\alpha_{s_1}=\mu_1\beta_1+\cdots+\mu_{t_1}\beta_{t_1}$,解方程,得到的基础解系为组合系数的${\gamma_i}$,就是$W_1\cap W_2$的基
\end{itemize}
\subsection{题目}
\begin{itemize}
\item P199的27求坐标,最基本的题目
\item P201的29,求和与交的基
\item P212的5,是某年原题
\end{itemize}

\section{线性变换}
\subsection{定义及定理}
\begin{itemize}
\item 设$\phi:V_1\to V_2$,则$\dim V_1=\dim(\ker\phi)+\dim(\Im\phi)$
\item $\mathcal{A}(\alpha_1,\cdots,\alpha_2)=(\beta_1,\cdots,\beta_n)A$
\item 若$\beta=\mathcal{A}\alpha$,则$y=Ax$,其中$x,y$为$\alpha,\beta$的坐标
\item $V$中的$\mathcal{A}$在$\epsilon_1,\cdots,\epsilon_n$和$\eta_1,\cdots,\eta_n$的矩阵分别为$A$和$B$,从$\epsilon_1,\cdots,\epsilon_n$到$\eta_1,\cdots,\eta_n$的过渡矩阵是$X$,则$B=X^{-1}AX$
\item $f(\lambda)=\det(\lambda I-A)=\lambda^n-a_1\lambda^{n-1}+a_2\lambda^{n-2}+\cdots+(-1)^n a_n$是特征多项式则,$a_i$是$A$的$i$阶主子式之和,若有$n$个根$\lambda_1,\cdots,\lambda_n$则$a_1=\lambda_1+\cdots+\lambda_n,a_n=\lambda_1\lambda_2\cdots\lambda_n$
\end{itemize}
\subsection{题目}
\begin{itemize}
\item P220的1都很基础的计算

\item P225的7,$\mathcal{B}^n=0,\mathcal{B}^{n-1}\neq 0$,所以$\exists \beta\neq 0$,使得$\mathcal{B}^n\beta=0,\mathcal{B}^{n-1}\beta\neq 0$,于是$\beta,\mathcal{B}\beta,\cdots,\mathcal{B}^{n-1}\beta$线性无关

\item 从P258的1可以看到,到底什么时候用坐标什么时候用向量,首先看用的基是什么,比如用的基是$\epsilon_1,\epsilon_2,\epsilon_3$,那显然$\alpha,\beta,\gamma$和$\alpha_1,\beta_1,\gamma_1$就表示为坐标,用$Y=AX$,若用的基是$\alpha,\beta,\gamma$,我们就要找出$\alpha_1,\beta_1,\gamma_1$在$\alpha,\beta,\gamma$的坐标,也就是$(\alpha_1,\beta_1,\gamma_1)=(\alpha,\beta,\gamma)A$,而正好$A$是$\mathcal{A}$的矩阵

\item P259的2,已知$\tr(A^k)=0$,证$A^n=0$,很简单,$k$从1到$n$遍历,可以解出特征值全部为0,然后可以用特征多项式,其相似于一个对角线为0的上三角
\end{itemize}


\section{方阵相似标准型}
\subsection{定理及定义}
\begin{itemize}
\item 求$A$的Jordan标准型和可逆矩阵P,分四步,其中$J_{n_i}(\lambda_i)$中Jordan块的个数$t_i$
\[
t_i=\dim V_{\lambda_i} = n - \rank(A-\lambda_iI)
\]
记$d_k$为$k$阶Jordan块的个数
\[
d_k = [\rank(A-\lambda_iI)^{k-1}-\rank(A-\lambda_iI)^{k}] - [\rank(A-\lambda_iI)^{k}-\rank(A-\lambda_iI)^{k+1}]
\]

\item 请注意P263的特征多项式和最小多项式,以及最小多项式得到的不变子空间,见高等代数下P370的13
\end{itemize}
\subsection{题目}
\begin{itemize}
\item P284的18教你证对角线上两元素相等
\item P285的20证二,证法奇特
\item P302的46有循环向量的例子
\item P308有教怎么求不变因子
\item P315有教怎么求J和P
\item P335的74是考试原题?
\item P336的75解矩阵函数
\end{itemize}

\section{二次型}
\subsection{定义与定理}
\begin{itemize}
\item P352有正定的等价条件
\end{itemize}

\subsection{题目}
\begin{itemize}
\item P366、P371、P378有化二次型为标准型的
\item P371的18用了一个取特殊向量的方法
\item P378的28有用到之前的对角优
\item P387的40,与柯西不等式类似可以用二次判别式
\item P389的44再次用二次判别式
\end{itemize}

\section{欧几里得空间}
\subsection{定义与定理}
\begin{itemize}
\item $V$是$\mathcal{R}$上的线性空间,若$g(\alpha,\beta)=\langle\alpha,\beta\rangle$,有$\langle\alpha,\alpha\rangle\geq 0$,可交换,线性加,线性乘,则$g$为内积且$V$为欧几里得空间

\item 正交方阵定义,$A$满足$A^TA=I$,则$A$为正交方阵

\item P414有把实二次型化为标准型的方法
\end{itemize}

\subsection{题目}
\begin{itemize}
\item P416的施密特正交化
\item P424开始,求正交阵使得对角化
\item P438的16,在没有工具的时候怎么证对角矩阵
\item P441打洞的精髓都在注里
\item P446的28结合坐标变换
\item P450正交矩阵要是有特征值1,则$|I-A|=|AA^T-A|=|A||A^T-I|=|A||A-I|=|A-I|$
\item P483求两个矩阵公共特征向量
\end{itemize}






































\end{document}