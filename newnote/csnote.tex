%!TEX program = xelatex
\documentclass[UTF8]{ctexart}

\pagestyle{plain}
\begin{document}


\section{线性结构}
\begin{itemize}
	\item 队列是一种特殊的线性表,特殊之处在于它只允许在表的前端(front)进行删除操作,而在表的后端(rear)进行插入操作
\end{itemize}

\section{字符串}
\begin{itemize}
	\item KMP及其变种
\end{itemize}

\section{树}
\begin{itemize}
	\item B-树,多路搜索树,根结点的儿子数为[2, M];除根结点以外的非叶子结点的儿子数为[M/2, M];所有叶子结点位于同一层;还有儿子与父母大小关系
	\item B+树,是B-树变体,叶子结点存所有数据,父母节点存儿子节点最小的元素
\end{itemize}

\section{图}
\begin{itemize}
	\item AOE,先算事件发生最早,然后事件发生最迟,然后活动发生最早,活动发生最迟,最后活动差额为0的
	\item 连通分量与极大连通子图
\end{itemize}


\section{查找}
\begin{itemize}
	\item 各种查找算法的查找长度,以二叉树为例,成功的数节点,失败的数边
	\item 平衡二叉树的递推公式,$N_0 = 0$,$N_1 = 1$,$N_2= 2$,$N_h = 1+N_{h-1}+N_{h-2}$
\end{itemize}



\section{排序}
\begin{itemize}
	\item 各种内部排序的比较P311
	\item 堆排序,建堆的时候用shiftDown而不是插入的shiftup
	\item 二路归并,比如是10个,第一轮2、2、2、2、2,第二轮4、4、2
\end{itemize}

\section{计算机概述}
\begin{itemize}
	\item CPI定义:执行一条指令花费时钟周期数
	\item MIPS:每秒执行多少百万条指令
\end{itemize}

\section{机器数}
\begin{itemize}
	\item 单精度32位:1位符号位S,8位移码E,范围是1到255(原码+127),23位隐含1的原码M,即$(-1)^S\times1.M\times2^{E-127}$,
	\item 运算的对阶,小对大
	\item 只有上溢出中断,没有下
	\item 大端、小端(地位低地址,放在内存的前面)
	\item 奇偶校验法,奇校验整个校验码中1的个数为奇数
	\item 补码运算对PSW的影响第四套43
	\item 当用移码表示阶码,能用全零表示机器零的阶码
	\item 最简单的浮点数舍入处理是截尾法
	\item 浮点数规格化,右规:尾数右移一位,阶码+1,左规:尾数左移一位,阶码-1
\end{itemize}

\section{指令}
\begin{itemize}
	\item 各种寻址方式的比较P153
	\item 与基址变址寻址方式相类似,相对寻址以程序计数器PC的当前值(R15中的值)为基地址,指令中的地址标号作为偏移量,将两者相加后得到操作数的有效地址
	\item 微程序的指令流程
	\item 水平型编码控制的微指令格式P207
	\item 流水线的优化
	\item 流水线的时空图
	\item 流水线的吞吐率TP=任务数N/任务周期数T
	\item 流水线的效率=任务所占面积/总面积
	\item 指令字长一般是存储字长的整数倍
	\item RISC指令集


\end{itemize}

\section{CPU}
\begin{itemize}
	\item PSW不属于控制器
\end{itemize}

\section{存储器}
\begin{itemize}
	\item SRAM与DRAM的定义与比较P92,前者比较高级
	\item 读写周期

	\item ROM分两次传送地址,所以地址线是RAM的一半

	\item cache分为直接映射(模个数)、x-路组相联映射(模个数除以x)
	\item TLB有效存储时间
	\item 交叉存储器,高位多体交叉在单个存储器中连续存放,不能保证程序局部性,低位多体交叉在多个存储器中交叉存放
	\item cache主存系统的访问效率,cache命中率$H = \frac{N_c}{N_c+N_m}$,主存慢于cache倍数$r=\frac{T_m}{T_c}$,访问效率$e=\frac{1}{H+r(1-H)}=\frac{T_c(N_c+N_m)}{T_cN_c+T_m(N_m)}$

\end{itemize}

\section{总线}
\begin{itemize}
	\item 总线仲裁三种比较P244
	\item PCI总线的特点
\end{itemize}

\section{中断}
\begin{itemize}
	\item 进中断的流程P271
	\item 中断向量是中断服务程序的入口地址,中断向量地址是“中断服务程序的入口地址”的地址
	\item 一个高级中断屏蔽了低级中断,则要返回到主程序才能相应低级中断
	\item 中断程序有可能是用户程序,但是中断处理程序一定是OS程序
	\item 缺页中断与一般中断的区别,缺页中断在指令执行期间产生并处理,一般中断要指令结束处理,一条指令可能产生多次缺页中断
	\item 缺页,访问内存页表+中断处理+访问内存页表+访问内存
\end{itemize}

\section{IO}
\begin{itemize}
	\item DMA,在内存和IO设备之间直接进行数据交换,不需要CPU的干预。当需要IO数据传输时,CPU将DMA初始化,之后DMA接管总线的使用权,将所需要的数据全部读入内存后,IO设备的控制器才会发出中断
	\item 通道有独立的处理器,有自身IO指令,没有自己内存,与主存共享空间,通道指令放在内存;能执行有限通道指令的IO控制器,代替CPU管理控制外设。通道有自己的指令系统,是一个协处理器,一般用在大型计算机系统中(不是大型机)。通道实质是一台能够执行有限的输入输出指令,并能被多台外设共享的小型DMA专用处理机
	\item 通道管理涉及的数据结构
\end{itemize}

\section{CPU调度}
\begin{itemize}
	\item 各种调度算法的比较
	\item 高响应比优先,响应比=(等待时间+要求服务时间)/要求服务时间
	\item CPU利用率
	\item 系统吞吐率表示单位时间内CPU完成的作业数量
	\item 周转时间=作业完成时间-作业提交时间
	\item 平均周转时间=(作业1周转时间+作业2周转时间+作业n周转时间)/n
	\item 带权周转时间=作业周转时间/作业实际运行时间
	\item 平均带权周转时间=(作业1带权周转时间+作业2带权周转时间+作业n带权周转时间)/n
	\item 等待时间指进程处于等处理机状态时间之和
	\item 响应时间指从用户提交请求到系统首次产生响应所用时间
	\item FCFS有利于长作业,不利于短作业,CPU繁忙为长作业,IO繁忙为短作业
\end{itemize}

\section{内存管理}
\begin{itemize}
	\item 静态重定位是装配程序完成的,动态重定位是运行时计算的,页式是动态重定位
	\item 多级页表算法,出到页表项之后可以直接指数除
\end{itemize}

\section{磁盘管理}
\begin{itemize}
	\item 最短寻道优先,每次都找最近的
	\item 读写磁盘的时间
\end{itemize}


\section{同步互斥}
\begin{itemize}
	\item 信号量
	\item PV操作请背题目,管程
	\item 两个通信模块之间同步通信、异步通信全锁半锁无锁
	\item 互斥信号量初始值为1,同步为自定
\end{itemize}

\section{死锁}
\begin{itemize}
	\item 死锁充分必要条件
	\item 银行家算法
\end{itemize}

\section{单位的区别}
\begin{itemize}
	\item 硬盘传送速率中的K是按1000来算的,不是1024
\end{itemize}


\section{网络的基本参数}
\begin{itemize}
	\item 总时延=发送时延+传播时延
	\item 信道利用率=发送时间/(发送时间+2X传播时间)
	\item 网络接口层的SAP是MAC地址,网络层的SAP是IP地址,TCP、UDP是端口号
\end{itemize}

\section{以太网}
\begin{itemize}
	\item 介质访问冲突,二进制指数退避算法
	\item CSMA三种:1-坚持,空闲则发,不空闲坚持监听;p-坚持,空闲则p概率发,1-p概率延迟再监听;非坚持,空闲则发,不空闲延迟再监听
	\item CSMA/CD协议,最大帧碎片长度必须小于最小帧长度,任何接收到帧长小于最小帧长度的就丢弃
\end{itemize}

\section{IP}
\begin{itemize}
	\item 记得IP数据报因太大分片,每个片去掉头20,合在一起要加回20
	\item 广播域,IP层,路由器分割
	\item 冲突域,MAC层,争用信道则有冲突,路由器、交换机分割
	\item 路由表的格式
	\item 直接广播地址要知道网络号+主机号全1,本地广播地址32bit都是1
	\item ARP响应是单播
	\item TCP/IP报文头的字段的单位,IP首部长度32位为单位,总长度字节为单位,片移位8字节为单位,TCP序号字节单位
\end{itemize}

\section{TCP}
\begin{itemize}
	\item 握手与挥手

	\item 拥堵窗口算法:三个重复报文就是避免拥堵,减到一半,超时就是慢开始,减到0
\end{itemize}

\end{document}