%!TEX program = xelatex
\documentclass[UTF8]{ctexart}
\newcommand{\ud}{\,\mathrm{d}}
\usepackage{amsmath}
\usepackage{amssymb} % once we use this package, the \mathbb command is available
\DeclareMathOperator{\st}{s.t.}

\pagestyle{plain}
\begin{document}
\section{数列极限与函数极限}
\begin{itemize}

\item 常用的等价变换
\begin{eqnarray*}
x&\sim& \sin x \sim \tan x \sim \arctan x \sim \arcsin x\\
&\sim& \ln(1+x) \sim e^x-1\\
&\sim&\frac{a^x-1}{\ln a} \sim \frac{(1+x)^b-1}{b}
\end{eqnarray*}
还有$(1-\cos x)\sim \frac{1}{2}x^2$

\item $a>1,0<\alpha<1,k \in \mathbb{N},\st n\to \infty,\ln\ln n <\ln n < n^\alpha < n^k <a^n < n! < n^n $

\item $\lim_{n \to \infty}\sqrt[n]{a_1^n+a_2^n+\cdots+a_m^n} = \max_{1\leq i \leq m}\{a_i\}$

\item 变上限积分的求导
\begin{eqnarray*}
\phi(t) & = & \int_{\alpha (t)}^{\beta (t)} f(x,t) \ud x\\
\phi'(t) & = & \int_{\alpha (t)}^{\beta (t)} f_t'(x,t) \ud x + f(\beta (t),t)\beta' (t)- f(\alpha (t),t)\alpha' (t)
\end{eqnarray*}
\end{itemize}


\section{一元函数的连续性}
\begin{itemize}


\item max、min和取绝对值等运算,保连续性,所以一些拼接函数可以用max、min表示,接着转换到绝对值表示

\item 可去间断点:极限$\lim_{x\rightarrow x_0} f(x)$存在,但是不等于$f(x_0)$,或者$f(x)$在点$x_0$没有定义

\item 第一类间断点:$\lim_{x\rightarrow x_0^-} f(x)$与$\lim_{x\rightarrow x_0^+} f(x)$都存在但不相等

\item 第二类间断点:$f(x)$在点$x_0$的左、右极限至少有一个不存在


\end{itemize}

\section{一元微分学}
\begin{itemize}
\item 可微性的证明就是极限
\[
f'(x_0) = \lim_{x\to x_0} \frac{f(x)-f(x_0)}{x-x_0} = \lim_{h\to 0} \frac{f(x_0+h)-f(x_0)}{h}
\]
存在性的证明

\item 要证明不可微性,可以证$f(x)$在$x_0$不连续,或者$f_+'(x_0)\neq f_-'(x_0)$,或$x$以不同的方式趋向于$x_0$时,上面的极限取不同的值

\item $(u\cdot v)^{(n)} = \sum_{k=0}^n \text{C}_n^k u^{(k)}v^{(n-k)}$

\item 数学分析中的典型问题和方法P216,零点存在性问题,借助介值性求解(连续函数有介值性、导函数也有介值性),或者借助Rolle定理求解

\item 式子中同时出现$f(x)$和$f'(x)$,就要想到要用$e^x$来构造$F(x)$,比如数学分析中的典型问题和方法P218

\item 导数无第一类间断,其实很显然,若某点第一类间断,即左极限不等于右极限,此时极限不存在,即该点不可导

\item Darboux定理,若函数$f(x)$在区间$[a,b]$上处处可导,$f'(a)<f'(b)$,则$\forall c,f'(a)<c<f'(b),\exists \xi \in (a,b),\st f'(\xi)=c$

\item
\[
f(x) = \sum_{k=0}^n \frac{1}{k!}f^{(k)}(x_0)(x-x_0)^k + R_n(x)
\]
其中$R_n(x)$可取拉格朗日余项
\[
R_n(x) = \frac{1}{(n+1)!}f^{(n+1)}(\xi)(x-x_0)^{n+1}
\]
也可以取Peano余项$R_n(x)=o((x-x_0)^n)$

\item P300的例3.5.3,$f^{(k)}(x_0)=0(k=1,2,\cdots,n-1),f^{(n)}(x_0)\neq0$,此时直接泰勒展开,可知函数值由$f^{(n)}(x_0)$决定

\item 渐近线,$a=\lim_{x\to +\infty}\frac{f(x)}{x}$,$b=\lim_{x\to +\infty}[f(x)-ax]$

\item 曲率$K$与曲率半径$\rho$
\[
	K = \frac{|f''(t)|}{[1+(f'(t))^2]^{\frac{3}{2}}} \qquad\rho = \frac{1}{K}
\]


\end{itemize}

\section{一元积分学}
\begin{itemize}
\item 在P350的例4.3.6和P360的例4.3.20都有在区间$[a,b]$中间做泰勒展开以后积分为零的现象


\item Cauchy-Schwarz不等式的证法,即变成一个与$t$相关的二次函数再求判别式,可以逆着用即已知判别式反过来构造二次函数

\item Cauchy-Schwarz不等式的特点是把积分号内的平方拿到积分外
\[
\left(\int_a^b f(x)g(x)\ud x \right)^2 \leq \int_a^b f^2(x)\ud x \int_a^b g^2(x)\ud x
\]

\item P442的例4.5.41,积分中函数的两个位置相减可以做区间替换

\end{itemize}

\section{多元微分学}
\subsection{极限存在与连续等定理}
\begin{itemize}
\item 证明二元极限不存在,证径向路径的极限与幅角有关;证特殊路径极限不存在;证两个特殊路径极限不相等;在空心邻域里连续,但二累次极限存在不相等

\item 多元连续还是按定义证,就是两点距离小于$\epsilon$,然后证函数值小于$\delta$

\item 可微,若存在$A$和$B$,使得
\[
\lim_{\rho\to 0}\frac{f(x_0+\Delta x,y_0+\Delta y)-f(x_0,y_0) - A\Delta x - B\Delta y}{\rho} =0
\]
其中$\rho=\sqrt{(\Delta x)^2+(\Delta y)^2}$

\item 偏导数连续$\rightarrow$可微$\rightarrow$偏导数存在,可微$\rightarrow$连续,任何一个箭头都不可逆

\item 空间曲线$x=x(t),y=y(t),z=z(t)$的切向量是$(x'(t),y'(t),z'(t))$,曲面$F(x,y,z)=0$的法向量是$(F_x',F_y',F_z')$

\item 多元极值判定,Hessian矩阵正定取极小,负定取极大,不定无极值

\item 条件极值与Lagrange法,用Lagrange函数的二阶微分进行判断,大于0取极小

\item 隐函数存在的条件,$F(x_0,y_0)=0$,$F(x,y)$和$F_y'(x,y)$在$(x_0,y_0)$的一个邻域内连续,$F_y'(x_0,y_0)\neq 0$
\[
y_x'=-\frac{F_x'(x,y)}{F_y'(x,y)}
\]


\item 方向导数
\[
\frac{\partial f}{\partial \vec{l}}(X_0) = \lim_{t\to 0^+} \frac{f(X_0 +t\vec{l}) -f(X_0)}{t}
\]


\item 若$f(X)$在$X_0$可微,则对于任意单位向量$\vec{l}$的方向导数存在
\[
\frac{\partial f}{\partial \vec{l}}(X_0) = \langle\nabla f(X_0),\vec{l} \rangle \leq |\nabla f(X_0)|
\]
其中梯度为$\nabla f(x)$,当$\vec{l}$方向与$\nabla f(x)$方向相同,得最大方向导数$|\nabla f(x)|$


\item 对任意非零向量
\[
\frac{\partial f}{\partial \vec{l}}(X_0) = \frac{\vec{l}}{|\vec{l}|} \cdot \nabla f(X_0)
\]

\item 梯度
\[
\text{grad}f(X_0) = \nabla f(X_0) = \left(\frac{\partial f}{\partial x_1}(X_0),\cdots,\frac{\partial f}{\partial x_n}(X_0) \right)
\]


\end{itemize}


\section{重积分}
\begin{itemize}
\item 对于积分$I=\iint_D f(x,y)\ud x\ud y$,作变换$x=x(u,v),y=y(u,v)$,找出变换后区域$D'=\{(u,v):a\leq u\leq b,\phi(u)\leq v\leq \psi(u)\}$
\[
I=\int_a^b\ud u\int_{\phi(u)}^{\psi(u)} f(x(u,v),y(u,v))|J|\ud v
\]
其中,雅各比还可以取逆
\begin{eqnarray*}
J&=&\frac{\partial(x,y)}{\partial(u,v)}\\
&=& \left(
\begin{array}{cc}
\frac{\partial x}{\partial u}&\frac{\partial x}{\partial v}\\
\frac{\partial y}{\partial u}&\frac{\partial y}{\partial v}\\
\end{array}
\right)\\
J^{-1}&=&\frac{\partial(u,v)}{\partial(x,y)}
\end{eqnarray*}

\item 极坐标系下二重积分
\begin{eqnarray*}
\iint_D f(X)\ud \sigma&=&\iint_D f(x,y)\ud x\ud y\\
&=&\iint_D f(r\cos \theta,r\sin \theta)r\ud r\ud \theta\\
&=&\int_\alpha^\beta \ud \theta \int_{r_1(\theta)}^{r_2(\theta)} f(r\cos \theta,r\sin \theta)r\ud r
\end{eqnarray*}

\item 三重积分两种方法,$V=\{(x,y,z)|(x,y)\in D,z_1(x,y)\leq z \leq z_2(x,y) \}$
\[
\iiint_v f(x,y,z)\ud V = \iint_D \ud x\ud y\int_{z_1(x,y)}^{z_2(x,y)} f(x,y,z)\ud z
\]
或者$V=\{(x,y,z)|a\leq z \leq b,(x,y)\in D_z \}$
\[
\iiint_v f(x,y,z)\ud V = \int_a^b \ud z \iint_{D_z}f(x,y,z)\ud x\ud y
\]

\item P882三重线性坐标系旋转变换,将$Oxy$旋转到平面$ax+by+cz=0$
\[
\zeta = \frac{ax+by+cz}{a^2+b^2+c^2}
\]
此时$x$轴与$y$轴被旋转到$\zeta=0$的平面内,把它们记为$\xi$轴与$\eta$轴,$|J|=1$

\item 球坐标变换,$x=r\sin\phi\cos\theta,y=r\sin\phi\sin\theta,z=r\cos\phi,J=r^2\sin\phi$

\item 广义球坐标变换,$x=ar\sin\phi\cos\theta,y=br\sin\phi\sin\theta,z=cr\cos\phi,J=abcr^2\sin\phi$

\item 重心
\[
\overline{x} = \frac{\iiint_{\Omega} x \ud x}{\iiint_{\Omega} \ud x}
\]

\end{itemize}


\section{曲线曲面积分}
\begin{itemize}

\item 一型曲线积分
\[
\int_L f(x,y)\ud s = \int_a^b f(x(t),y(t))\sqrt{x'^2(t)+y'^2(t)}\ud t
\]

\item P931解释了二型曲线积分的对称性,总的来说就是不单止考虑函数符号,还要考虑到积分方向投影到$\ud x$上的正负

\item P942二型变一型
\[
\int_L P\ud x+ Q\ud y+ R\ud z = \int_L (P\cos\alpha + Q\cos \beta + R\cos \gamma)\ud s
\]
接着用柯西不等式,后面直接变1

\item Green公式
\[
\int_{L^+} P\ud x+ Q\ud y = \iint_D (\frac{\partial Q}{\partial x}-\frac{\partial P}{\partial y})\ud x \ud y
\]
此外还有面积公式
\[
S = \iint_D\ud x\ud y = \int_{L^+}x\ud y = - \int_{L^+}y\ud x = \frac{1}{2} \int_{L^+}x\ud y-y\ud x
\]
积分与路径无关
\[
\frac{\partial Q}{\partial x}-\frac{\partial P}{\partial y} = 0
\]

\item 在P949、P969有一型变二型的$\ud s$变$\ud{x}$和$\ud{y}$的方法
\[
\int_L [P\cos(t,x) + Q\cos(t,y)] \ud s = \int_{L^+} P\ud x+ Q\ud y 
\]
其中$(t,x)$和$(t,y)$分别表示$x$轴正向,$y$轴正向与动点切线正向的夹角。另外也有用外法向量的
\[
\frac{\partial u}{\partial n} = \frac{\partial u}{\partial x}\cos(n,x)+\frac{\partial u}{\partial y}\cos(n,y)
\]
从而,但是观察与上面区别
\[
\frac{\partial u}{\partial n} \ud s = \frac{\partial u}{\partial x} \ud y - \frac{\partial u}{\partial y} \ud x
\]
\item P969非常清晰地解释了向量的运算
\begin{eqnarray*}
\mathbf{n_1} &=& (\cos(n,x),\cos(n,y))\\
\mathbf{l_1} &=& (\cos(l,x),\cos(l,y))\\
\end{eqnarray*}
其中$\mathbf{n_1}$表示$\mathbf{n}$的单位向量
\[
\cos(l,n) = \mathbf{n_1}\cdot \mathbf{l_1}
\]
接下来可以用上面的

\item 利用公式计算第一型曲面积分,首先选取平面,使便于求$S$的投影区域$\Delta$,$S:z=z(x,y),(x,y)\in \Delta$
\[
I = \iint_S f(x,y,z)\ud S = \iint_\Delta f(x,y,z(x,y))\sqrt{1+z_x'^2+z_y'^2}\ud x\ud y
\]

\item P981向坐标面投影不方便,可以做新的,如$S$为$x+y+z=t$被$x^2+y^2+z^2\leq 1$截取部分,作坐标轴旋转,令以下,并在$w=0$的平面上,任取二正交轴,使$x^2+y^2+z^2=u^2+v^2+w^2$
\[
w=\frac{x+y+z}{\sqrt{3}}
\]
接着直接变坐标轴,投影,计算,无需增加$J$

\item 用参数方程,P982,特别的$S$为球面$\ud S=\sqrt{EG-F^2}\ud \phi\ud \theta = R^2\sin\phi \ud \phi $

\item 两种曲面积分的关系,其中$\vec{n} = (\cos \alpha,\cos\beta,\cos\gamma)$是$\partial\Omega$的单位外法向量
\begin{eqnarray*}
\iint_{\partial\Omega^+}P\ud y\ud z + Q\ud z\ud x+ R\ud x\ud y &=& \iint_{\partial\Omega}(P\cos \alpha + Q\cos \beta+ R\cos \gamma)\ud S \\
&=& \iiint_{\Omega}\left(\frac{\partial P}{\partial x}+\frac{\partial Q}{\partial y}+\frac{\partial R}{\partial z}\right)\ud x\ud y\ud z
\end{eqnarray*}

\item Stock公式,其中$\vec{n} = (\cos \alpha,\cos\beta,\cos\gamma)$是$S$的单位法向量
\begin{eqnarray*}
\int_{\partial S}P\ud x+Q\ud y+R\ud z&=&\iint_S\left|
\begin{array}{ccc}
\ud y\ud z&\ud z\ud x&\ud x\ud y\\
\frac{\partial}{\partial x}&\frac{\partial}{\partial y}&\frac{\partial}{\partial z}\\
P&Q&R\\
\end{array}
\right| \\
&=& \iint_S\left|
\begin{array}{ccc}
\cos \alpha&\cos\beta&\cos\gamma\\
\frac{\partial}{\partial x}&\frac{\partial}{\partial y}&\frac{\partial}{\partial z}\\
P&Q&R\\
\end{array}
\right| \ud S
\end{eqnarray*}

\item P997径向与法向的夹角
\[
\cos(n,r) = \frac{\mathbf{r}}{r}\cdot \mathbf{n} = \frac{x}{r}\cos \alpha +\frac{y}{r}\cos\beta +\frac{z}{r}\cos \gamma
\]

\item P1007习题7.4.3有如何求某平面的单位法向量,如$F(x,y,x)=0$,则求$(F_x',F_y',F_z')$,接着单位化即可

\item P1013法向量和径向向量的夹角$\rho = r \cos\alpha = r\cdot\mathbf{r_1}\cdot\mathbf{n_1}$,其中$\mathbf{r_1}$是径向向量,$\mathbf{n_1}$是法向量
\end{itemize}



\section{数项级数}
\subsection{正项级数收敛性判断}
\begin{itemize}
\item Cauchy准则
\item 正项级数判阶法,相对于$\frac{1}{n}$比较,可结合等价无穷小、洛必达、带Peano的Taylor
\item 正项级数比较法及极限形式
\[
\lim_{n\to\infty} \frac{a_n}{b_n} = r
\]
\item D'Alembert法,Cauchy根式法
\item 正项级数收敛的Cauchy积分判别法,若$f(x)>0$,在$[1,+\infty)$上递减,则$\sum_{n=1}^\infty f(n)$与广义积分$\int_1^{+\infty}f(x)\ud x$同时敛散
\item 部分和$\sum_{k=1}^n a_k$有界
\end{itemize}

\subsection{变号级数收敛性判断}
\begin{itemize}
\item 对$\sum |a_n|$用D'Alembert法,Cauchy根式法,若收敛则绝对收敛,若$\sum |a_n|$发散,则$a_n \nrightarrow 0$从而$\sum a_n$发散
\item Leibniz,$a_n\geq 0$,单调下降趋向零,则$\sum (-1)^{n-1}a_n$收敛
\item Abel,$\sum a_n$收敛,$\{b_n\}$单调有界
\item Dirichlet,$\sum a_n$有界,$\{b_n\}$单调趋向零
\item Cauchy准则
\item 条件收敛是$\sum a_n$收敛,$\sum|a_n|$发散
\item Abel变换,$\sum_{k=1}^m a_kb_k =a_mB_m + \sum_{k=1}^{m-1}(a_k-a_{k-1})B_k$
\end{itemize}

\subsection{题目}
\begin{itemize}
\item P452关于$\cos$求和的变形
\begin{eqnarray*}
\sum_{k=1}^n \cos kx &=& \frac{1}{2\sin\frac{x}{2}} \sum_{k=1}^n2\sin\frac{x}{2}\cos kx\\
&=& \frac{1}{2\sin\frac{x}{2}} \sum_{k=1}^n \left( \sin(k+\frac{1}{2})x-\sin(k-\frac{1}{2})x\right)\\
&=& \frac{1}{2\sin\frac{x}{2}} (\sin \frac{2n+1}{2}x-\sin\frac{x}{2})
\end{eqnarray*}

\end{itemize}


\section{幂级数}
\subsection{幂级数的收敛半径和收敛范围}
\begin{itemize}
\item
\[
R=\frac{1}{\lim_{n \to \infty}\sqrt[n]{|a_n|}}
\]
分母为0时,$R=+\infty$,分母为正无穷时,$R=0$
\item
\[
R=\lim_{n\to \infty}\frac{|a_n|}{|a_{n+1}|}
\]
求了收敛区间,要验证端点的敛散性
\item 利用收敛半径求极限,其实就是上面第一个式子反过来用

\item 初等函数展开为幂级数,可用已知的展开,用逐项积分或微分,计算指定点各阶导数,然后Taylor,用级数的运算

\item 上面的逐项积分或微分只能在区间内进行,但是要验证在端点的情况,见P559

\item Taylor的方法,先求各阶导,再求半径,在半径内把$\sim$变成$=$即判断解是否满足微分方程(用上解的唯一性),最后验证端点收敛,见P561的例5.3.15

\item 求和问题,用逐项求导或积分,构造方程式然后解,用Abel第二定理计算

\item 根据Abel第二定理,求$\sum_{n=1}^\infty a_n$,先求$\sum_{n=1}^\infty a_nx^n$在$(-1,1)$内的$S(x)$然后$x\to 1_-$,即$\sum_{n=1}^\infty a_n=\lim_{x\to 1_-}S(x)$,而$S(x)$用上法求出

\item 将被积函数展开为幂级数,然后逐项积分,其中有可能用到函数项级数积分

\end{itemize}

\subsection{题目}
\begin{itemize}
\item P556的例5.3.10有$(1-x)\sum_{n=1}^\infty(a_1+a_2+\cdots+a_n)x^n=\sum_{n=1}^\infty a_nx^n$,同理P554的$(1-x)\sum_{n=0}^\infty S_nx^n=\sum_{n=0}^\infty a_nx^n$,而且乘上$(1-x)$都使得半径小了

\end{itemize}


\section{幂级数展开}
\begin{itemize}
\item 在$\mathbf{R}$上
\[
\sin x =\sum_{n=0}^\infty (-1)^n \frac{x^{2n+1}}{(2n+1)!}
\]

\item 在$\mathbf{R}$上
\[
\cos x =\sum_{n=0}^\infty (-1)^n \frac{x^{2n}}{(2n)!}
\]

\item 在$\mathbf{R}$上
\[
e^x=\sum_{n=0}^\infty \frac{x^n}{n!}
\]

\item 在$(-1,1]$上
\[
\ln(1+x)=\sum_{n=1}^\infty (-1)^{n-1}\frac{x^n}{n}
\]

\item 在$(-1,1)$上
\[
(1+x)^\alpha = 1+ \sum_{n=1}^\infty \frac{\alpha(\alpha-1)\cdots(\alpha-n+1)}{n!}x^n
\]
\end{itemize}

\section{泰勒展开}
\begin{eqnarray*}
\sin x &=&\sum_{k=0}^n (-1)^k \frac{x^{2k+1}}{(2k+1)!}+(-1)^{n+1} \frac{\cos\theta x}{(2n+3)!} x^{2n+3}\\
&=& x-\frac{x^3}{3!}+\frac{x^5}{5!}+o(x^5)\\
\cos x &=&\sum_{k=0}^n (-1)^k \frac{x^{2k}}{(2k)!} + (-1)^{n+1} \frac{\cos\theta x}{(2n+2)!}x^{2n+2}\\
&=& 1-\frac{x^2}{2!}+\frac{x^4}{4!}+o(x^4)\\
e^x&=&\sum_{k=0}^n \frac{x^k}{k!}+\frac{e^{\theta x}}{(n+1)!}x^{n+1}\\
&=& 1+\frac{x}{1!}+\frac{x^2}{2!}+\frac{x^3}{3!}+o(x^3)\\
\ln(1+x)&=&\sum_{k=1}^n (-1)^{k-1}\frac{x^k}{k}+(-1)^n\frac{x^{n+1}}{(n+1)(1+\theta x)^{n+1}}\\
&=& x-\frac{x^2}{2}+\frac{x^3}{3}+o(x^3)\\
(1+x)^\alpha &=& 1+ \sum_{k=1}^n \frac{\alpha(\alpha-1)\cdots(\alpha-k+1)}{k!}x^k+\frac{\alpha(\alpha-1)\cdots(\alpha-n)(1+\theta x)^{\alpha-n-1}}{(n+1)!}x^{n+1}\\
&=& 1+\frac{a}{1!}x+\frac{a(a-1)}{2!}x^2+\frac{a(a-1)(a-2)}{3!}x^3+o(x^3)\\
\frac{1}{1-x} &=& 1+x+x^2+x^3+o(x^3)\\
\arctan x &=& x-\frac{x^3}{3}+\frac{x^5}{5}-\cdots+(-1)^n\frac{x^{2n+1}}{2n+1}+o(x^{2n+1})\\
\arcsin x &=& x+\frac{1}{3}\cdot\frac{1}{2!!}x^3+\frac{1}{5}\cdot\frac{3!!}{4!!}x^5+\cdot+\frac{1}{2n+1}\cdot\frac{(2n-1)!!}{(2n)!!}x^{2n+1}+o(x^{2n+1})
\end{eqnarray*}

\section{傅立叶级数}
\begin{itemize}
\item 周期为$2\pi$的$f(x)$在$(-\infty,+\infty)$上展开成一致收敛的三角级数
\[
	S(x) = \frac{a_0}{2}+\sum_{n=1}^\infty(a_n \cos nx+ b_n \sin nx)
\]
在$[- \pi,\pi]$上逐项积分
\begin{eqnarray*}
a_n &=& \frac{1}{\pi} \int_{-\pi}^{\pi} f(x)\cos nx\ud x, n=0,1,2,\cdots\\
b_n &=& \frac{1}{\pi} \int_{-\pi}^{\pi} f(x)\sin nx\ud x, n=1,2,\cdots
\end{eqnarray*}

\item 周期为$T$的$f(x)$在$(-\infty,+\infty)$上展开成一致收敛的三角级数
\[
	S(x) = \frac{a_0}{2}+\sum_{n=1}^\infty(a_n \cos \frac{2n\pi}{T}x+ b_n \sin \frac{2n\pi}{T}x)
\]
在$[- \frac{T}{2},\frac{T}{2}]$上逐项积分
\begin{eqnarray*}
a_n &=& \frac{2}{T} \int_{-\frac{T}{2}}^{\frac{T}{2}} f(x)\cos \frac{2n\pi}{T}x\ud x, n=0,1,2,\cdots\\
b_n &=& \frac{2}{T} \int_{-\frac{T}{2}}^{\frac{T}{2}} f(x)\sin \frac{2n\pi}{T}x\ud x, n=1,2,\cdots
\end{eqnarray*}

\item $f(x)$的傅立叶级数$S(x)$在点$x_0$收敛到
\[
\frac{f(x_0+0)+f(x_0-0)}{2}
\]
\end{itemize}


\section{积分常用}
\begin{eqnarray*}
\int \frac{1}{x}\ud x &=& \ln|x|+C \\
\int a^x \ud x&=&\frac{a^x}{\ln a}+C \\
\int \sec^2 x\ud x &=& \tan x+C \\
\int \csc^2 x\ud x &=& -\cot x+C \\
\int \frac{\ud x}{\sqrt{1-x^2}} &=&\arcsin x+C = -\arccos x+C \\
\int \frac{\ud x}{1+x^2} &=& \arctan x + C \\
\int \frac{\ud x}{a^2+x^2}&=&\frac{1}{a} \arctan\frac{x}{a} +C \\
\int \frac{\ud x}{a^2-x^2}&=&\frac{1}{2a} \ln\left|\frac{a+x}{a-x}\right| +C \\
\int \frac{\ud x}{\sqrt{a^2-x^2}}&=&\arcsin\frac{x}{a} +C \\
\int \frac{\ud x}{\sqrt{x^2\pm a^2}}&=&\ln|x+\sqrt{x^2\pm a^2}| +C \\
\int \sqrt{a^2-x^2}\ud x&=&\frac{x}{2}\sqrt{a^2-x^2}+\frac{a^2}{2}\arcsin\frac{x}{a} +C \\
\int \sqrt{x^2\pm a^2}\ud x&=&\frac{x}{2}\sqrt{x^2\pm a^2}\pm \frac{a^2}{2}\ln|x+\sqrt{x^2\pm a^2}| +C \\
\int \tan x\ud x&=&-\ln|\cos x|+C\\
\int \cot x\ud x&=&\ln|\sin x|+C\\
\int \sec x\ud x&=&\ln|\sec x+\tan x|+C\\
\int \csc x\ud x&=&-\ln|\csc x-\cot x|+C\\
\int_0^{\frac{\pi}{2}} \sin^n x \ud x = \int_0^{\frac{\pi}{2}} \cos^n x \ud x & = & 
\begin{cases}
\frac{n-1}{n} \cdot \frac{n-3}{n-2} \cdots \frac{1}{2} \cdot \frac{\pi}{2} \qquad n\text{为正偶数} \\
\frac{n-1}{n} \cdot \frac{n-3}{n-2} \cdots \frac{2}{3} \cdot 1\qquad n\text{为大于1的奇数}
\end{cases}
\end{eqnarray*}

\section{微分常用}
\begin{eqnarray*}
(a^x)'&=&a^x\ln a\\
(\log_a |x|)' &=& \frac{1}{x\ln a}\\
(\ln |x|)' &=& \frac{1}{x}\\
(\tan x)' &=& \sec^2 x\\
(\cot x)' &=& -\csc^2 x\\
(\sec x)' &=& \tan x\sec x\\
(\csc x)' &=& \cot x\csc x\\
(\arcsin x)' &=& \frac{1}{\sqrt{1-x^2}}\\
(\arccos x)' &=& -\frac{1}{\sqrt{1-x^2}}\\
(\arctan x)' &=& \frac{1}{1+x^2}\\
\end{eqnarray*}

\section{简单二元函数的全微分}
\begin{eqnarray*}
y\ud x + x\ud y&=& \ud(xy)\\
\frac{y\ud x - x\ud y}{y^2} &=& \ud(\frac{x}{y})\\
\frac{-y\ud x + x\ud y}{x^2} &=& \ud(\frac{y}{x})\\
\frac{y\ud x - x\ud y}{xy} &=& \ud(\ln|\frac{x}{y}|)\\
\frac{y\ud x - x\ud y}{x^2+y^2} &=& \ud(\arctan\frac{x}{y})\\
\frac{y\ud x - x\ud y}{x^2-y^2} &=& \frac{1}{2}\ud(\ln|\frac{x-y}{x+y}|)\\
\end{eqnarray*}



\section{和差化积积化和差}
\begin{eqnarray*}
\sin a \cos b&=&\frac{[\sin(a+b)+\sin(a-b)]}{2} \\
\cos a \sin b&=&\frac{[\sin(a+b)-\sin(a-b)]}{2} \\
\cos a \cos b&=&\frac{[\cos(a+b)+\cos(a-b)]}{2} \\
\sin a \sin b&=&-\frac{[\cos(a+b)-\cos(a-b)]}{2} \\
\sin x+\sin y&=&2\sin(\frac{a+b}{2})\cos(\frac{a-b}{2})\\
\sin x-\sin y&=&2\cos(\frac{a+b}{2})\sin(\frac{a-b}{2})\\
\cos x+\cos y&=&2\cos(\frac{a+b}{2})\cos(\frac{a-b}{2})\\
\cos x-\cos y&=&-2\sin(\frac{a+b}{2})\sin(\frac{a-b}{2})\\
\end{eqnarray*}

\section{复数与$e^x,\sin,\cos$}
\begin{eqnarray*}
\binom{k+1}{i}&=&\binom{k}{i-1}+\binom{k}{i}\\
e^{i\theta} &=& \cos\theta +i\sin\theta\\
\sin \theta &=& \frac{e^{i\theta}-e^{-i\theta}}{2i}\\
\cos \theta &=& \frac{e^{i\theta}+e^{-i\theta}}{2}\\
\sin 3x &=& 3\sin x-4\sin^3 x\\
\cos 3x &=& 4\cos^3 x-3\cos x\\
\arctan\left(\frac{A+B}{1-AB}\right) &=& \arctan A+\arctan B\\
\sec^2x-1 &=& \tan^2x\\
\csc^2x-1 &=& \cot^2x\\
\sin(\frac{\pi}{2}+x) &=& \cos x\\
\sin(\frac{\pi}{2}-x) &=& \cos x\\
\cos(\frac{\pi}{2}+x) &=& -\sin x\\
\cos(\frac{\pi}{2}-x) &=& \sin x\\
\tan(\frac{\pi}{2}+x) &=& -\cot x\\
\tan(\frac{\pi}{2}-x) &=& \cot x\\
\cot(\frac{\pi}{2}+x) &=& -\tan x\\
\cot(\frac{\pi}{2}-x) &=& \tan x\\
\int_{-\infty}^{+\infty}\frac{1}{\sqrt{2\pi}}e^{-\frac{x^2}{2}} &=& 1\\
I_n&=&\int \tan^n x\ud{x}\\
&=& \int[\tan^{n-2}x (1+\tan^2x)]\ud{x}-\int\tan^{n-2}x \ud{x}\\
&=& \frac{\tan^{n-1}x}{n-1}-I_{n-2}
\end{eqnarray*}


\section{空间向量与解析几何}
\begin{eqnarray*}
\mathbf{a} \cdot \mathbf{b} &=& |\mathbf{a}| |\mathbf{b}| \cos \theta \\
S_{triangle} &=& \frac{1}{2}|\mathbf{a} \times \mathbf{b}| \\
|\mathbf{a}\times \mathbf{a}|&=&|\mathbf{a}|\cdot|\mathbf{a}|\cdot \sin \theta\\
\text{rot}\mathbf{A} & = & \nabla \times \mathbf{A}\\
\text{div}\mathbf{A} & = & \nabla \cdot \mathbf{A}
\end{eqnarray*}


\section{可分离变量微分方程}
\begin{itemize}

	\item 一阶微分方程
	\[
		\frac{\ud y}{\ud x} = P(x)y + Q(x)
	\]
	通解为
	\[
		y=e^{\int P(x)\ud x}(\int Q(x)e^{-\int P(x)\ud x}\ud x+c)
	\]

	\item 非隐函数情况
	\[
		\frac{\ud y}{\ud x} = g(\frac{y}{x})
	\]
	作变换
	\[
		u = \frac{y}{x}
	\]
	

	\item 隐函数情况一
	\[
		y = f(x,\frac{\ud y}{\ud x}) \qquad x = f(y,\frac{\ud y}{\ud x})
	\]
	作变换
	\[
		p = \frac{\ud y}{\ud x}
	\]

	\item 隐函数情况二
	\[
		F(x,y') = 0 \qquad F(y,y') = 0
	\]
	作变换,引入参数$t$
	\[
		p = y' \rightarrow \ud y = p \ud x
	\]

\end{itemize}

\section{齐次与非齐次微分方程}
\begin{itemize}

	\item $e^{Kt} = e^{(\alpha+i\beta)t}=e^{\alpha t}(cos\beta t + i\sin \beta t)$,即若出现$\sin\beta t$、$\cos \beta t$的解,则特征方程有$\pm i\beta$的根

	\item 若齐次微分方程的特征方程有$k_1$重根$\lambda_1$,则$e^{\lambda_1t},te^{\lambda_1t},t^2e^{\lambda_1t},\cdots,t^{k_1-1}e^{\lambda_1t}$为其$k_1$个解

	\item 设$f(x)=(b_0t^m+b_1t^{m-1}+\cdots+b_{m-1}t+b_m)e^{\lambda t}$,非齐次微分方程有特解$x=t^k(B_0t^m+B_1t^{m-1}+\cdots+B_{m-1}t+B_m)e^{\lambda t}$,其中$k$为特征方程的根$\lambda$的重数
\end{itemize}

\section{线性方程组}
\begin{itemize}

	\item 把$n$元线性方程组的系数矩阵$A$的第$j$列换成常数向量,得到矩阵记做$B_j$,当$|A|\neq 0$,方程组有唯一解
	\[
		\left(\frac{|B_1|}{|A|},\frac{|B_2|}{|A|},\cdots,\frac{|B_n|}{|A|} \right)
	\]

\end{itemize}


\section{矩阵运算}
\begin{itemize}
\item 范德蒙行列式
\[
\left(
\begin{array}{cccc}
1 & 1 & \cdots & 1 \\
a_{1} & a_{2} & \cdots & a_{n} \\
\vdots & \vdots & \ddots & \vdots \\
a_{1}^{n-1} & a_{2}^{n-1} & \cdots & a_{n}^{n-1} \\
\end{array}
\right) = \prod_{1 \leq j < i \leq n} (a_i - a_j)
\]
三阶的例子
\[
\left(
\begin{array}{cccc}
1 & 1 & 1 \\
a_{1} & a_{2} & a_{3} \\
a_{1}^2 & a_{2}^2 & a_{3}^2 \\
\end{array}
\right) = (a_3 - a_2)(a_3 - a_1)(a_2 - a_1)
\]
\end{itemize}

\section{矩阵运算}

\begin{itemize}
\item $A$与$B$为$m\times n$矩阵
\[
\lambda^n |\lambda E_{m\times m} - AB^T| = \lambda^m|\lambda E_{n\times n} - B^{T}A|
\]

\item 如果$m\times n$矩阵$A$的秩为$r$,那么它的任何$s$行组成的子矩阵$A_1$的秩大于或等于$r+s-m$,证明见P115

\item $r((A,B)) \leq r(A)+r(B)$
\item 见P117
\[
rank\left(
\begin{array}{cc}
A & C\\
0 & B\\
\end{array}
\right) \geq rank(A) + rank(B)
\]

\item $r((A,B))\geq \max\{r(A),r(B)\}$
\item $r(AB)\leq \min\{r(A),r(B)\}$
\item $r(A+B)\leq r(A)+r(B)$

\item 基本矩阵,左乘是行变化,右乘是列变换
\[
E_{ij} = \left(
\begin{array}{ccccc}
0 & 0 & 0 & 0 & 0 \\
0 & 0 & 0 & a_{ij} & 0 \\
0 & 0 & 0 & 0 & 0 \\
0 & 0 & 0 & 0 & 0 \\
0 & 0 & 0 & 0 & 0 \\
\end{array}
\right)
\]
上面的意思是第$i$行,第$j$列为1,那么$PA$相当于,把$A$的$j$行移到了$i$行去;$AP$相当于把$A$的$i$列移到了$j$列去

\item 初等矩阵,由单位矩阵经过一次初等行(列)变换得到
\item $P(j,i(k))$左乘$A$,相当于把$A$的第$i$行的$k$倍加到第$j$行上,其余行不变

\item $P(j,i(k))$右乘$A$,相当于把$A$的第$j$列的$k$倍加到第$i$列上,其余列不变
\item $P(i,j)$左(右)乘$A$,相当于把$A$的第$i$行(列)与第$j$行(列)互换,其余行(列)不变
\item $P(i(c))$左(右)乘$A$,相当于用$c$乘$A$的第$i$行(列),其余行(列)不变
\item $P(j,i(-k))P(j,i(k)) = I$
\item $P(i,j)P(i,j) = I$
\item $P(i(\frac{1}{c}))P(i(c)) = I$


\item $(AB)^* = B^* A^*$,$(AB)^{-1}=B^{-1}A^{-1}$
\item $(A^T)^{-1}=(A^{-1})^T$,$(A^*)^{-1} = (A^{-1})^*$

\item 伴随矩阵,其中$A_{ij}$为$a_{ij}$的代数余子式
\[
\left(
\begin{array}{cccc}
a_{11} & a_{12} & \cdots & a_{1n} \\
a_{21} & a_{22} & \cdots & a_{2n} \\
\vdots & \vdots & \ddots & \vdots \\
a_{n1} & a_{n2} & \cdots & a_{nn} \\
\end{array}
\right) \left(
\begin{array}{cccc}
A_{11} & A_{21} & \cdots & A_{n1} \\
A_{12} & A_{22} & \cdots & A_{n2} \\
\vdots & \vdots & \ddots & \vdots \\
A_{1n} & A_{2n} & \cdots & A_{nn} \\
\end{array}
\right) = \left(
\begin{array}{cccc}
|A| & 0 & \cdots & 0 \\
0 & |A| & \cdots & 0 \\
\vdots & \vdots & \ddots & \vdots \\
0 & 0 & \cdots & |A| \\
\end{array}
\right)
\]
即$AA^{*}=|A|I$

\end{itemize}




\section{特征值与对角化}
\begin{itemize}

	\item 设$A$、$B$是可逆矩阵,$P^{-1}AP = B $


	\item 区分若当儿型矩阵与可对角化矩阵
	\item 施密特正交化,$\alpha_1,\alpha_2,\dots,\alpha_n$是一个线性无关的向量组
	\begin{eqnarray*}
	\beta_1 &=& \alpha_1\\
	\beta_2 &=& \alpha_2 - \frac{(\alpha_2,\beta_1)}{(\beta_1,\beta_1)}\beta_1\\
	\beta_s &=& \alpha_s - \sum_{j=1}^{s_1}\frac{(\alpha_s,\beta_j)}{(\beta_j,\beta_j)}\beta_1
	\end{eqnarray*}
	然后单位化
	\[
	\eta_i = \frac{1}{|\beta_i|} \beta_i
	\]
\end{itemize}

\section{线性空间}
\begin{itemize}

\item $(\eta_1,\cdots,\eta_n)=(\epsilon_1,\cdots,\epsilon_n)A$,则$A$是$\epsilon_1,\cdots,\epsilon_n$到$\eta_1,\cdots,\eta_n$的过渡矩阵,设$\alpha\in V$,在$\epsilon_1,\cdots,\epsilon_n$下的坐标为$x=(x_1,\cdots,x_n)^T$,在$\eta_1,\cdots,\eta_n$下的坐标为$x=(y_1,\cdots,y_n)^T$,则有如下的坐标变换公式$y=A^{-1}x$
\item $V$中的$\mathcal{A}$在$\epsilon_1,\cdots,\epsilon_n$和$\eta_1,\cdots,\eta_n$的矩阵分别为$A$和$B$,从$\epsilon_1,\cdots,\epsilon_n$到$\eta_1,\cdots,\eta_n$的过渡矩阵是$X$,则$B=X^{-1}AX$
\end{itemize}


\section{随机变量的数字特征}
\begin{itemize}
	\item $E(X+Y) = E(X) + E(Y)$
	\item 若$X$、$Y$是相互独立的随机变量,$E(XY) = E(X)E(Y)$
	\item 均值$D(X) = E\{ [X- E(x)]^2 \}$,标准差$\sigma(X) = \sqrt{D(X)}$
	\item $D(X) = E(X^2)- [E(x)]^2 $
	\item $D(CX) = C^2D(X)$,且$D(C+X) = D(X)$
	\item $D(X+Y) = D(X) + D(Y) +2Cov(X,Y)$
	\item $Cov(X,Y) = E\{[X-E(X)][Y-E(Y)]\} = E(XY) - E(X)E(Y)$
	\item $\rho_{XY} = \frac{Cov(X,Y)}{\sqrt{D(X)}\sqrt{D(Y)}}$
	\item $Cov(aX,bY) = abCov(X,Y)$
	\item $Cov(X_1+X_2,Y) = Cov(X_1,Y)+Cov(X_2,Y)$

\end{itemize}




\section{事件集合}
\begin{itemize}
	\item $AB=A\cap B$,交集缩写

	\item $A+B=(A\cup B)+(A\cap B)$

	\item $A-B = \{x|x\in A$且$x \notin B\}$

	\item 无论$A$、$B$事件是否独立,都有$AB \subset (A\cup B)$,故$P(AB)\leq P(A\cup B)$

	\item 德摩根律$\overline{A\cup B}=\overline{A} \cap \overline{B}$,$\overline{A\cap B}=\overline{A} \cup \overline{B}$


\end{itemize}

\section{正态分布}
\begin{itemize}

	\item 若$X\sim N(\mu,\sigma^2)$,则
	\[
		Z = \frac{X-\mu}{\sigma} \sim N(0,1)
	\]
	即,$X$的分布
	\[
		F(x) = \Psi(\frac{x-\mu}{\sigma})
	\]
	$X$的密度
	\[
		f(x) = \frac{1}{\sigma} \psi(\frac{x-\mu}{\sigma})
	\]
	$X$的期望
	\[
		E(X) = E(\mu+\sigma Z) = E(\mu) + E(\sigma Z) = \mu
	\]

	\item 若$X_i\sim N(\mu_i,\sigma_i^2)$,且它们相互独立,则$Z=C_1X_1+C_2X_2+\dots + C_nX_n$仍服从正态分布
	\[
		Z \sim N(C_1\mu_1 +C_2\mu_2+ \dots +C_n\mu_n,C_1^2\sigma^2_1 +C_2^2\sigma^2_2+ \dots +C_n^2\sigma^2_n)
	\]


	\item 伯努利不等式,设随机变量$X$的$E(X) = \mu $,$D(X) = \sigma^2$,则
	\[
		P\{ |X-\mu| \geq \epsilon \} \leq \frac{\sigma^2}{\epsilon^2}
	\]

\end{itemize}


\section{二维随机变量分布}
\begin{itemize}

	\item $P\{x_1 < X \leq x_2, y_1< Y \leq y_2\}=F(x_2,y_2)-F(x_2,y_1)+F(x_1,y_1)-F(x_1,y_2)$

	\item 边缘分布函数$F_X(x)=P\{X\leq x\}=P\{X\leq x,Y<\infty\}=F(x,\infty)$
	\[
		F_X(x)= \int_{-\infty}^x\left[\int_{-\infty}^{\infty}f(x,y)\ud{y}\right]\ud{x}
	\]
	\item 离散随机变量条件分布律
	\[
		P\{X = x_i| Y=y_j\}=\frac{P\{X = x_i, Y=y_j\}}{P\{Y=y_j\}} = \frac{p_{ij}}{p_{\cdot j}}
	\]

	\item 条件概率密度
	\[
		f_{X|Y}(x|y) = \frac{f(x,y)}{f_Y(y)}
	\]

	\item $Z=X+Y$的分布,$f_{X+Y}(z) = \int_{-\infty}^{\infty}f(x,z-x)\ud{x}=\int_{-\infty}^{\infty}f_X(x)f_Y(z-x)\ud{x}$
\end{itemize}



\section{大数定律和中心极限定理}
\begin{itemize}


\item 辛勤大数定理,$X_i$是独立同分布的随机变量序列,且$E(X_i)=\mu<\infty$,对于任意的$\epsilon>0$,有
\[
	\lim_{n\to \infty}P\left(\left| \frac{1}{n}\sum_{i=1}^nX_i-\frac{1}{n}\sum_{i=1}^nE(X_i)\right|<\epsilon \right) = 1
\]

\item 中心极限定理,$X_i$是独立同分布的随机变量序列,且$E(X_i)=\mu<\infty$,$D(X_i)=\sigma^2$,则随机变量之和标准化变量的分布函数
\[
	\lim_{n\to \infty} P\left\{\frac{\sum_{k=1}^n X_k-n\mu}{\sqrt{n}\sigma}\leq x \right\} = \int_{-\infty}^x \frac{1}{\sqrt{2\pi}}e^{-\frac{t^2}{2}}\ud{t} = \Phi(x)
\]

\end{itemize}


\section{样本与统计分布}
\begin{itemize}
	\item 设$X_i$是来自总体$N(0,1)$的样本,则称统计量
	\[
		\chi^2 = X_1^2 + X_2^2 + \dots + X_n^2
	\]
	服从自由度为$n$的$\chi^2$分布,记为$\chi^2 \sim \chi^2(n)$
	\item 设$\chi_1^2 \sim \chi^2(n_1)$,$\chi_2^2 \sim \chi^2(n_2)$,并且$\chi_1^2$,$\chi_2^2$相互独立,有
	\[
		\chi_1^2 + \chi_2^2 \sim \chi^2(n_1 + n_2)
	\]
	\item $\chi^2$分布的上$\alpha$分位点在右边的尾部,单侧
	\[
		P\{\chi^2 > \chi_\alpha^2(n)\} = \int_{\chi_\alpha^2(n)}^\infty f(y) \ud y = \alpha
	\]

	\item 设$X \sim N(0,1)$,$Y \sim \chi^2(n)$,并且$X$,$Y$相互独立,则称随机变量
	\[
		t = \frac{X}{\sqrt{Y/n}}
	\]
	服从自由度为$n$的$t$分布,记为$t\sim t(n)$

	\item $t$分布的分位点,可以单边,可以双边,下面为上$\alpha$分位点的公式
	\[
		P\{t > t_\alpha (n)\} = \int_{t_\alpha (n)}^\infty h(t)\ud t = \alpha
	\]

	\item $ t_{1-\alpha} (n) = -t_\alpha (n)$


	\item 设$U \sim \chi^2(n_1)$,$V \sim \chi^2(n_2)$,并且$U$,$V$相互独立,则称随机变量
	\[
		F = \frac{U/n_1}{V/n_2}
	\]
	服从自由度为$(n_1,n_2)$的$F$分布,记为$F\sim F(n_1,n_2)$

	\item 若$F\sim F(n_1,n_2)$,则
	\[
		\frac{1}{F} \sim F(n_2,n_1)
	\]
	\item $F$分布的分位点,单侧,称上$\alpha$分位点
	\[
		P\{F> F_\alpha(n_1,n_2)\} = \int_{F_\alpha(n_1,n_2)} ^ \infty \psi(y) \ud y = \alpha
	\]

	\item $F$分布的上$\alpha$分位点性质
	\[
		F_{1-\alpha}(n_1,n_2) = \frac{1}{F_\alpha(n_2,n_1)}
	\]
\end{itemize}

\section{正态总体的样本}
\begin{itemize}
	\item 设总体$X$(不管服从什么分布,只要均值和方差存在)的均值为$\mu$,方差为$\sigma^2$,$X_i$是来自总体$X$的一个样本,$\overline{X}$,$S^2$分布是样本均值和样本方差,则有
	\[
		E(\overline{X}) = \mu,\qquad D(\overline{X}) = \sigma^2 /n
	\]
	而
	\begin{eqnarray*}
		E(S^2) &=& E[\frac{1}{n-1}(\sum_{i=1}^n X_i^2-n\overline{X}^2)]\\
		       &=& \frac{1}{n-1}[\sum_{i=1}^n E(X_i^2) - nE(\overline{X}^2)]\\
		       &=& \frac{1}{n-1}[\sum_{i=1}^n (\sigma^2 + \mu^2) - n(\sigma^2/n + \mu^2)]\\
		       &=& \sigma^2
	\end{eqnarray*}

	\item $X_i$是来自正态总体$N(\mu,\sigma^2)$的一个样本,$\overline{X}$,$S^2$分布是样本均值和样本方差,有
	\[
		\overline{X} \sim N(\mu,\sigma^2/n)
	\]
	和
	\[
		\frac{(n-1)S^2}{\sigma^2} \sim \chi^2(n-1)
	\]
	且$\overline{X}$和$S^2$相互独立


	\item $X_i$是来自正态总体$N(\mu,\sigma^2)$的一个样本,$\overline{X}$,$S^2$分布是样本均值和样本方差,有
	\[
		\frac{\overline{X}-\mu}{S/\sqrt{n}} \sim t(n-1)
	\]

	\item $X_{n_1}$是来自正态总体$N(\mu_1,\sigma_1^2)$的一个样本,$Y_{n_2}$是来自正态总体$N(\mu_2,\sigma_2^2)$的一个样本,并且这两个样本相互独立,$\overline{X}$,$S_1^2$是样本均值和样本方差,$\overline{Y}$,$S_2^2$是样本均值和样本方差,有
	\[
		\frac{S_1^2/S_2^2}{\sigma_1^2/\sigma_2^2} \sim F(n_1-1,n_2-1)
	\]
	当$\sigma_1^2 = \sigma_2^2 = \sigma^2$时
	\[
		\frac{(\overline{X} - \overline{Y})-(\mu_1 - \mu_2)}{S_w\sqrt{\frac{1}{n_1}+\frac{1}{n_2}}}\sim t(n_1+n_2 -2)
	\]
	其中
	\[
		S_w^2 = \frac{(n_1-1)S_1^2+(n_2-1)S_2^2}{n_1+n_2 -2}
	\]
\end{itemize}


\section{参数估计}
\begin{itemize}

	\item 点估计要求无偏性$E(\hat{\theta})=\theta$,有效性$D(\hat{\theta_1})<D(\hat{\theta_2})$,相合性$\lim_{n\to \infty}P\{ |\hat{\theta}-\theta | < \epsilon\} = 1$

	\item 区间估计,上$\alpha$分位点,区间是开的,如果是双边,记得是$\alpha/2$
\end{itemize}



\section{假设检验}
\begin{itemize}

	\item 在显著性水平$\alpha$下,检验假设$H_0:\mu = \mu_0,H_1:\mu \neq \mu_0$,双边假设

	\item 检验统计量如
	\[
		Z=\frac{\overline{X}-\mu_0}{\sigma/\sqrt{n}}
	\]
	\item 检验统计量$Z$的观察值的绝对值$|z|\geq k$,则称$\overline{x}$与$\mu_0$的差异是显著的,拒绝$H_0$

	\item 确定了显著性水平$\alpha$,以上门槛值$k$可被确定

	\item 弃真错误,第一类错误,即$H_0$实际上为真,但是在样本上拒绝$H_0$

	\item 取伪错误,第二类错误,即$H_0$实际上不真,但是在样本上接受$H_0$
\end{itemize}
















\end{document}