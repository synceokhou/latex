\section{2007数分}
\begin{enumerate}
\item 求$\lim_{x \to +\infty}x[e-(1+\frac{1}{x})^x]$
\begin{description}
\item[解] 用$y=\frac{1}{x}$代入,具体见吉米多维奇数学分析习题集题解2P127[1359]
\end{description}

\item 设$f(x)=\arctan x -\frac{\sqrt{3}}{4}\ln x$,求
\begin{description}
\item[(1)] $f'(x)$
\item[(2)] $f(x)$在区间$[0.01,100]$上的最小值
\item[解]
\begin{eqnarray*}
f'(x) & = & \frac{1}{1+x^2}-\frac{\sqrt{2}}{4x}\\
& = & - \frac{\sqrt{3}(x-\frac{\sqrt{3}}{3})(x-\sqrt{3})}{4x(1+x^2)}
\end{eqnarray*}
然后画图,知道$f(\frac{\sqrt{3}}{3})$是极小值点,和$f(100)$比最小值
\end{description}

\item 求
\[
\lim_{n \to \infty}\frac{1}{n}(1+e^{\sqrt{\frac{1}{n}}}+e^{\sqrt{\frac{2}{n}}}+\cdots +e^{\sqrt{\frac{n-1}{n}}} +e)
\]
\begin{description}
\item[解]
\begin{eqnarray*}
& = & \lim_{n \to \infty}\frac{1}{n}(e^{\sqrt{\frac{1}{n}}}+e^{\sqrt{\frac{2}{n}}}+\cdots +e^{\sqrt{\frac{n-1}{n}}} +e)\\
& = & \int_0^1 e^{\sqrt{x}}\ud x \\
& = & 2 \int_0^1 te^t \ud t \\
& = & 2
\end{eqnarray*}
\end{description}


\item 设$0<x<1$,证明$(1+x)\ln^2(1+x)<x^2$
\begin{description}
\item[证] 由Cauchy中值定理
\[
\frac{\ln(1+x)}{\frac{x}{(1+x)^\frac{1}{2}}}=\frac{\frac{1}{1+\xi}}{\frac{2+\xi}{2(1+\xi)^{\frac{3}{2}}}} = \frac{(1+\xi)^\frac{1}{2}}{1+\frac{\xi}{2}} = \left( \frac{1+\xi}{1+\xi+\frac{\xi^2}{4}}\right)^{\frac{1}{2}} < 1
\]
\end{description}

\item 求
\[
\int_0^{\frac{\pi}{2}} \frac{\cos^2x}{\sin x+\cos x}\ud x
\]
\begin{description}
\item[解] 首先想到代换$t=\frac{\pi}{2}-x$
\begin{eqnarray*}
& = & \frac{1}{2}\int_0^{\frac{\pi}{2}} \frac{1}{\sqrt{2}\cos(x-\frac{\pi}{4})}\ud x\\
& = & \frac{\sqrt{2}}{2} \int_0^{\frac{\pi}{4}} \frac{\sec x(\sec x+\tan x)}{\sec x+\tan x} \ud x \\
& = & \frac{\sqrt{2}}{2} \ln(\sqrt{2}+1)
\end{eqnarray*}
\end{description}

\item 求星形线$x^{\frac{2}{3}}+y^{\frac{2}{3}}=a^{\frac{2}{3}}(a>0)$的切线与两条坐标轴围成的三角面积的最大值
\begin{description}
\item[解] 在$(x_0,y_0)$处切线方程$y-y_0=-(\frac{y_0}{x_0})^\frac{1}{3}(x-x_0)$
\begin{eqnarray*}
S & = & \frac{1}{2} x_0^{\frac{1}{3}}|y_0^\frac{2}{3}+x_0^\frac{2}{3}| y_0^{\frac{1}{3}}|y_0^\frac{2}{3}+x_0^\frac{2}{3}| \\
& = & \frac{1}{2} a^{\frac{4}{3}}x_0^{\frac{1}{3}}y_0^{\frac{1}{3}}\\
& \leq & \frac{1}{2} a^{\frac{4}{3}}\frac{x_0^{\frac{1}{3}}+y_0^{\frac{1}{3}}}{2}
\end{eqnarray*}
\end{description}

\item 设地球的半径为$R$,距离地球中心$r(r\geq R)$处空气密度为
\[
\rho(r)=\rho_0e^{k(1-\frac{r}{R})},(\rho,k\in \mathbf{R}^+)
\]
求地球上空气总质量
\begin{description}
\item[解]
\begin{eqnarray*}
M & = & \int_0^\infty \rho_0 e^{k(1-\frac{r}{R})}4\pi r^2 \ud r \\
& = & 4\pi R^3\rho_0\frac{k^2+2k+2}{k^3}
\end{eqnarray*}
\end{description}

\item 设函数$f(x)$在$G=(-\infty,+\infty)$内二阶可导,记$M_k=\sup_{x\in G}|f^{(k)}(x)|(k=0,1,2)$如果$M_0,M_2$均有限,且$M_2>0$,证明$M_1\leq \sqrt{2M_0M_2}$
\begin{description}
\item[证]
\begin{eqnarray*}
f(x+h) = f(x)+f'(x)h+\frac{f''(\xi)}{2}h^2 \\
f(x-h) = f(x)-f'(x)h+\frac{f''(\eta)}{2}h^2 \\
f(x+h) - f(x-h) = 2f'(x)h+\frac{h^2}{2}[f''(\xi)-f''(\eta)] \\
|f'(x)| = \left|\frac{f(x+h)-f(x-h)}{2h}-\frac{h}{4}[f''(\xi)-f''(\eta)]\right| \leq \frac{M_0}{h}+\frac{M_2}{2}h
\end{eqnarray*}
\end{description}

\item 设$\{\psi_k\}$与$\{\delta_k\}$为两个无穷非负实数列,满足$\psi_{k+1}\leq(1+\delta_k)\psi_k+\delta_k(k\geq 1)$,并且$\sum_{k=1}^{\infty}\delta_k\leq+\infty$,证明
\begin{description}
\item[(1)] 数列$\{\prod_{j=1}^k(1+\delta_j)\}$收敛
\item[(2)] 数列$\{\psi_k\}$有界
\item[(3)] 数列$\{\psi_k\}$收敛
\item[证(1)] $\sum_{k=1}^{\infty}\delta_k $收敛$\Rightarrow \sum_{k=1}^{\infty}\ln(1+\delta_k)$收敛(比值判别法)$\Rightarrow \left\{ \ln \left[\prod_{j=1}^k(1+\delta_j)\right]\right\}$收敛(部分和)$\Rightarrow \left\{\prod_{j=1}^k(1+\delta_j)\right\}$收敛($e^x$连续)
\item[证(2)] $1\leq \psi_{k+1}+1\leq (1+\delta_k)(1+\psi_k+1)$
\item[证(3)] 由$\{\psi_k\}$有界$M$
\[
\psi_{k+1}-\psi_k\leq \delta_k(\psi_k+1)\leq M \delta_k
\]
及$\sum_{k=1}^\infty \delta_k$收敛知$\forall \epsilon >0,\exists K>0,\st$
\[
\psi_{k+l}-\psi_{k}=\sum_{j=k}^{k+l-1}(\psi_{j+1}-\psi_j)\leq M\sum_{j=k}^{k+l-1}\delta_j<\frac{\epsilon}{2}(k\geq K)
\]
由$\{\psi_k\}$有界及Weierstrass聚点定理,$\exists \{k_j\},\psi,\st \psi_{k_j}\rightarrow\psi$,而
\[
\exists J>K,\st j\geq J \Rightarrow|\psi_{k_j}-\psi|<\frac{\epsilon}{2}
\]
综上,$\forall \epsilon>0,\exists K_J, \st k\geq K_J,\exists j(k)\geq J,\st k_{j(k)}\leq k<k_{j(k)+1}$,而
\begin{eqnarray*}
\psi_k &= &\psi_k - \psi_{k_{j(k)}}+\psi_{k_{j(k)}}\\
&\leq& \frac{\epsilon}{2}+(\psi+\frac{\epsilon}{2})(k_{j(k)}\geq j(k)\geq J\geq K)\\
&=&\psi+\epsilon
\end{eqnarray*}
又
\begin{eqnarray*}
\psi_k &= &\psi_{k_{j(k)+1}}-(\psi_{k_{j(k)+1}}-\psi_k )\\
&\geq& (\psi-\frac{\epsilon}{2})-\frac{\epsilon}{2} (j(k)+1> J,k\geq k_{j(k)} \geq K)\\
&=&\psi-\epsilon
\end{eqnarray*}
\end{description}

\item 设$f(x)$是是实轴上的可微函数,$\forall x,y \in \mathbf{R}$满足
\[
f(x+y)=\frac{f(x)+f(y)}{1+f(x)f(y)}
\]
并且$f'(0)=1$,求$f(x)$的表达式
\begin{description}
\item[证] 令$x=0$
\[
f(y)=\frac{f(0)+f(y)}{1+f(0)f(y)}
\]
$f(0)[f^2(y)-1]=0(\forall y \in \mathbf{R})$,因$f'(0)=1$,而$f(0)=0$。再对原式两边对$y$求导
\[
f'(x+y)=\frac{f'(y)[1+f(x)f(y)]-[f(x)+f(y)]f(x)f'(y)}{[1+f(x)f(y)]^2}
\]
令$y=0$,有
\begin{eqnarray*}
f'(x)=1-f^2(x)\\
\left[\frac{1}{f(x)-1} -\frac{1}{f(x)+1} \right]\ud f(x)=-2\ud x\\
\ln|\frac{f(x)-1}{f(x)+1}|=-2x+C_1\\
\frac{f(x)-1}{f(x)+1} = Ce^{-2x}(C=e^{C_1})
\end{eqnarray*}
因$f(0)=0$,而$C=-1$,于是
\[
\frac{f(x)-1}{f(x)+1} = -e^{-2x} \Rightarrow f(x)=\frac{1-e^{-2x}}{1+e^{-2x}}=\tanh x
\]
\end{description}

















\end{enumerate}