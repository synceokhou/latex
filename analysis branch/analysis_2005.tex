\section{2005数分}
\begin{enumerate}
\item 计算
\[
\lim_{x \to 0} \frac{\sqrt[5]{1+3x^4}-\sqrt{1-2x}}{\sqrt[3]{1+x}-\sqrt{1+x}}
\]
\begin{description}
\item[解] 用泰勒展开的皮亚诺展开项
\[
\lim_{x\to 0}\frac{[1+o(x)]-[1+\frac{1}{2}(-2x)+o(x)]}{[1+\frac{1}{3}x+o(x)]-[1+\frac{1}{2}x+o(x)]}=-6
\]
\end{description}

\item 设$a,b > 0,a\neq b$,证明
\[
\frac{2}{a+b} < \frac{\ln a -\ln b}{a-b} < \frac{1}{\sqrt{ab}}
\]
\begin{description}
\item[证] 由泰勒展开
\begin{eqnarray*}
\ln b &=& \ln \frac{a+b}{2}+\frac{2}{a+b}\frac{b-a}{2}-\frac{1}{2\xi^2}\left(\frac{b-a}{2}\right)^2\left(\frac{a+b}{2}<\xi<b\right)\\
\ln a &=& \ln \frac{a+b}{2}+\frac{2}{a+b}\frac{a-b}{2}-\frac{1}{2\eta^2}\left(\frac{a-b}{2}\right)^2\left(a<\eta<\frac{a+b}{2}\right)
\end{eqnarray*}
于是
\begin{eqnarray*}
\ln b -\ln a&=&2\frac{b-a}{a+b}-\frac{(b-a)^2}{8}\left[\frac{1}{\xi^2}-\frac{1}{\eta^2}\right]\\
&>&2\frac{b-a}{a+b}
\end{eqnarray*}
证右边
\begin{eqnarray*}
\ln b -\ln a &=& \ln\frac{b}{a}\\
&=& 2\ln\left(z=\sqrt{\frac{b}{a}}\right)\\
&=& \frac{2}{z}\cdot z\ln z\\
&=& \frac{2}{z}\int_1^z\left(1+\ln t\right)\ud t\\
&<& \frac{2}{z}\int_1^zt\ud t\\
&=& \frac{1}{z}(z^2-1)\\
&=& \frac{b-a}{\sqrt{ab}}
\end{eqnarray*}
\end{description}

\item 求
\[
\lim_{n \to \infty}\left(\frac{1}{n} + \frac{1}{n+1}+\cdots +\frac{1}{2n}\right)
\]
\begin{description}
\item[证]
\begin{eqnarray*}
\lim_{n \to \infty}\left( \frac{1}{n+1}+\cdots +\frac{1}{2n}\right) &=&\lim_{n \to \infty}\frac{1}{n}\sum_{i=1}^n\frac{1}{\frac{i}{n}}\\
&=& \int_0^1 \frac{1}{1+x}\ud x\\
&=& \ln 2
\end{eqnarray*}
\end{description}

\item 判断$\sum_{n=1}^{\infty} (-1)^n \frac{\ln n}{\sqrt{n}}$的收敛性
\begin{description}
\item[证] 记$f(x)=\frac{\ln x}{x}$,则由
\[
f'(x)=\frac{1-\ln x}{x^2}<0(x>e)
\]
知$\{\frac{\ln n}{\sqrt{n}}\}_{n=3}^\infty$递减,由莱布尼兹判别法知原级数收敛
又
\[
\left|(-1)^n \frac{\ln n}{\sqrt{n}}\right|\geq\frac{1}{\sqrt{n}} (n\geq 3)
\]
知原级数条件收敛
\end{description}

\item 设$f(x,y)$在点$(0,0)$的某个邻域中连续,$F(x)=\iint_{x^2+y^2\leq t^2}f(x,y)\ud x \ud y$,求$\lim_{t\to 0^+}\frac{F'(t)}{t}$
\begin{description}
\item[解]
\begin{eqnarray*}
F(t)&=&\int_0^t\int_0^{2\pi}f(r\cos\theta,r\sin\theta)r\ud \theta\ud r\\
F'(t)&=&\int_0^{2\pi}f(t\cos\theta,t\sin\theta)t\ud \theta\\
\frac{F'(t)}{t}&=& \lim_{t\to 0+} \int_0^{2\pi}f(t\cos\theta,t\sin\theta)\ud \theta = 2\pi f(0,0)
\end{eqnarray*}
\end{description}

\item 求$x^2+y^2+z^2=a^2$包含在柱面$\frac{x^2}{a^2}+\frac{y^2}{b^2}=1(b\leq a)$内的那部分面积
\begin{description}
\item[解] 背曲面积分公式
\begin{eqnarray*}
S&=& 2\iint_{\frac{x^2}{a^2}+\frac{y^2}{b^2}\leq 1}\frac{a}{\sqrt{a^2-x^2-y^2}}\ud x\ud y\\
&=&8a \int_0^a\int_0^{b\sqrt{1-\frac{x^2}{a^2}}}\frac{a}{\sqrt{a^2-x^2-y^2}}\ud y\ud x
\end{eqnarray*}
\end{description}


\item 设$f(x,y)=\psi(|xy|)$,其中$\psi(0)=0$,且$\psi(u)$在$u=0$的某个邻域中满足$|\psi(u)|\leq |u|^\alpha (\alpha > \frac{1}{2})$。证明$f(x,y)$在$(0,0)$处可微,但$g(x,y)=\sqrt{|xy|}$在$(0,0)$处不可微
\begin{description}
\item[证]
\[
\frac{|f(x,y)-f(0,0)-0\cdot x-0\cdot y|}{\sqrt{x^2+y^2}}\leq \frac{|xy|^\alpha}{|xy|^\frac{1}{2}}=|xy|^{\alpha-\frac{1}{2}}\to 0(x^2+y^2\to 0)
\]
反证法
\[
\lim_{x^2+y^2\to 0}\left|\frac{g(x,y)-g(0,0)-g_x(0,0)x-g_y(0,0)y}{\sqrt{x^2+y^2}}\right|=\lim_{x^2+y^2\to 0}\frac{\sqrt{|xy|}}{\sqrt{x^2+y^2}}
\]
存在,矛盾
\end{description}

\item 设$\psi(x)$在$[0,\infty)$上有连续导函数,并且$\psi(0)=1$。令
\[
f(r)=\iiint_{x^2+y^2+z^2\leq r^2} \psi(x^2+y^2+z^2)\ud x \ud y \ud z (r\geq 0)
\]
证明$f(r)$在$r=0$处三次可微,并求$f_+'''(0)$
\begin{description}
\item[证] 变量替换,注意$x^2+y^2+z^2$使得有对称性,代换$\phi$的$\sin\phi$要拆成两个对称区间,乘二
\[
f(r)=4\pi \int_0^r s^2\psi(s^2)\ud s
\]
由$\psi(x)$在$[0,\infty)$上有连续导函数
\begin{eqnarray*}
f'(r)=4\pi r^2\psi(r^2) (r\in[0,\infty))\\
f''(r)=4\pi[2r\psi(r^2)+2r^3\psi'(r^2)](r\in[0,\infty))\\
f_+'''(0) = \lim_{r\to0+}\frac{f''(r)-f''(0)}{r} = 8\pi\psi(0)=8\pi
\end{eqnarray*}
\end{description}

\item 设$f(x)$在有限区间$[a,b]$上可微,且满足$f_+'(a)f_-'(b)< 0$,则存在$c\in (a,b)$使得$f'(c)=0$
\begin{description}
\item[证] 这是达布定理
\end{description}


\item 设
\[
e^{e^x} = \sum_{n=0}^{\infty} a_n x^n
\]
求$a_0,a_1,a_2,a_3$并证明
\[
a_n\geq e(\gamma\ln n)^{-n} (n\geq 2)
\]
其中$\gamma$是某个大于$e$的常数
\begin{description}
\item[证] 由
\begin{eqnarray*}
\left(e^{e^x}\right)'= e^{e^x}e^x\\
\left(e^{e^x}\right)''= e^{e^x}e^2x+e^{e^x}e^x\\
\left(e^{e^x}\right)'''= e^{e^x}e^3x+3e^{e^x}e^2x+e^{e^x}e^x\\
\end{eqnarray*}
再乘上泰勒展开的系数,$a_0=a_1=a_2=e,a_3=\frac{5e}{6}$
\end{description}
































\end{enumerate}