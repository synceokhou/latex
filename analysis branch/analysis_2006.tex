\section{2006数分}
\begin{enumerate}

\item 求$a,b$使下列函数在$x=0$处可导
\[
y=\begin{cases}
ax+b & x\geq 0;\\
x^2+1 & x<0
\end{cases}
\]
\begin{description}
\item[解] 在$x=0$可导,所以连续,由$f_+(0)=f_-(0)$得$b=-1$,又有$f'_+(0)=f'_-(0)$得$a=0$
\end{description}

\item 已知$a_n>0,\sum_{n=1}^{\infty}\frac{1}{a_n}$发散,求证$\sum_{n=1}^{\infty}\frac{1}{a_n+1}$也发散
\begin{description}
\item[证] 都大于零,比较判别法
\[
\lim_{n\to \infty} \frac{a_n}{a_n+1}=1
\]
同收敛同发散?
\end{description}

\item 设$m,n\geq 0$为整数,求积分$\int_0^1 x^m(1-x)^n \ud x$的值
\begin{description}
\item[解] 原式记为$I(m,n)$
\begin{eqnarray*}
I(m,n) &= & (1-x)^n\frac{x^{m+1}}{m+1}|_0^1-\int_0^1n(1-x)^{n-1}(-1)\frac{x^{m+1}}{m+1}\ud x\\
&= & \frac{n}{m+1}I(m+1,n-1)\\
&=&\frac{n}{m+1}\frac{n-1}{m+2}\cdots\frac{1}{m+n}I(m+n,0)
\end{eqnarray*}
\end{description}

\item 设$a > 0$,$f(x)$是定义在$[-a,a]$上的连续的偶函数,证
\[
\int_{-a}^a\frac{f(x)}{1+e^x}\ud x =\int_0^a f(x)\ud x
\]
\begin{description}
\item[证] 由
\[
\frac{f(-x)}{1+e^{-x}} = \frac{f(x)}{1+e^{-x}}
\]
可得以下
\begin{eqnarray*}
\int_{-a}^a\frac{f(x)}{1+e^x}\ud x &=& \int_{0}^a\frac{f(x)}{1+e^x}\ud x +\int_{-a}^0\frac{f(x)}{1+e^x}\ud x \\
& = & \int_{0}^a\frac{f(x)}{1+e^x}\ud x + \int_{a}^0\frac{f(x)}{1+e^{-x}}\ud -x \\
& = & \int_{0}^a\frac{f(x)}{1+e^x}\ud x + \int_{0}^a\frac{f(x)}{1+e^{-x}}\ud x \\
& = & \int_{0}^af(x)[\frac{1}{1+e^x}+\frac{1}{1+e^{-x}}]\ud x
\end{eqnarray*}
\end{description}

\item 设函数$f(x)$在含有$[a,b]$的某个开区间内二次可导且$f'(a)=f'(b)=0$,证$\exists \xi \in (a,b)$使得
\[
|f''(\xi)|\geq \frac{4}{(b-a)^2}|f(b)-f(a)|
\]
\begin{description}
\item[证] 由
\begin{eqnarray*}
f\left(\frac{a+b}{2}\right)& = &f(a) + f'(a)\frac{b-a}{2}+\frac{f''(\xi_1)}{2}\left(\frac{b-a}{2}\right)^2\\
f\left(\frac{a+b}{2}\right)& = &f(b) + f'(b)\frac{a-b}{2}+\frac{f''(\xi_2)}{2}\left(\frac{a-b}{2}\right)^2
\end{eqnarray*}
知
\begin{eqnarray*}
|f(b)-f(a)|&=&\frac{(b-a)^2}{8}|f''(\xi_1)-f''(\xi_2)|\\
&\leq&\frac{(b-a)^2}{4}\max\{|f''(\xi_1)|,|f''(\xi_2)|\}
\end{eqnarray*}
\end{description}


\item 设实值函数$f(x)$及其一阶导数在区间$[a,b]$上均连续,而且$f(a)=0$,则
\begin{eqnarray*}
\max_{x\in [a,b]}|f(x)| & \leq & \sqrt{b-a}(\int_a^b|f'(t)|^2 \ud t)^\frac{1}{2}\\
\int_a^b f^2(x)\ud x & \leq & \frac{1}{2}(b-a)^2\int_a^b|f'(t)|^2\ud x
\end{eqnarray*}
\begin{description}
\item[证] 首先$[\int_a^bhg\ud x]^2\leq [\int_a^bh^2\ud x][\int_a^bg^2\ud x]$这个东西要证,然后观察到$|f(x)|=|\int_a^xf'(t)\ud t|$,于是令$h=1,g=f'(t)$
\begin{eqnarray*}
\left|\int_a^xf'(t)\ud t\right| &\leq &\sqrt{x-a}\left(\int_a^x|f'(t)|^2\ud t\right)^\frac{1}{2}\\
&\leq &\sqrt{b-a}\left(\int_a^b|f'(t)|^2\ud t\right)^\frac{1}{2}
\end{eqnarray*}
由上式,代入$\int_a^b f^2(x)\ud x$里面
\begin{eqnarray*}
\int_a^b f^2(x)\ud x &\leq & \int_a^b \left[(x-a)\int_a^x|f'(t)|^2\ud t \right]\ud x\\
&\leq & \int_a^b (x-a)\ud x \cdot \int_a^b|f'(t)|^2\ud x\\
\end{eqnarray*}
\end{description}


\item 设$n$是平面区域$D$的正向边界线$C$的外法向,则
\[
\oint_C \frac{\partial u}{\partial n}\ud s=\iint_D\left(\frac{\partial^2 u}{\partial x^2}+\frac{\partial^2 u}{\partial y^2} \right)\ud x \ud y
\]
\begin{description}
\item[证]
\begin{eqnarray*}
\oint_C \frac{\partial u}{\partial n}\ud s & = & \oint_C [\frac{\partial u}{\partial x}\cos(n,x)+ \frac{\partial u}{\partial y}\cos(n,y)]\ud s\\
& = & \oint_C [\frac{\partial u}{\partial x}\cos(t,y)- \frac{\partial u}{\partial y}\cos(t,x)]\ud s\\
& = & \oint_C \frac{\partial u}{\partial x}\ud y - \frac{\partial u}{\partial y}\ud x\\
\end{eqnarray*}
\end{description}

\item 设$\Gamma : \frac{x^2}{a^2}+\frac{y^2}{b^2}=1$的周长和所围成的面积分别是$L$和$S$,还令
\[
J = \oint_{\Gamma}(b^2x^2+2xy+a^2y^2)\ud s
\]
证$J=\frac{S^2L}{\pi^2}$
\begin{description}
\item[证] 注意$xy$在$\Gamma$上的第一型曲线积分正好消掉了为0,然后用椭圆柱代换
\begin{eqnarray*}
J &=& \oint_{\Gamma}(b^2x^2+a^2y^2)\ud s\\
&=&a^2b^2L\\
&=&\frac{(\pi ab)^2L}{\pi^2}\\
& =& \frac{S^2L}{\pi^2}
\end{eqnarray*}
\end{description}

\item 计算
\[
\int_0^1 \frac{\ud x}{1+x^3}
\]
的值,并证明
\[
\int_0^1 \frac{\ud x}{1+x^3} = \sum_{n=0}^\infty \frac{(-1)^{n-1}}{3n-2}
\]
\begin{description}
\item[证]
\begin{eqnarray*}
\int_0^1 \frac{\ud x}{1+x^3} &= & \int_0^1 \frac{\ud x}{(1+x)(1-x+x^2)}\\
&= & \frac{1}{3}\int_0^1 \left(\frac{1}{1+x} +\frac{-x+2}{1-x+x^2}\right)\ud x\\
&=& \int_0^1\sum_{n=0}^\infty(-1)^nx^{3n}\ud x\\
&=&\sum_{n=0}^\infty \frac{(-1)^n}{3n+1}\\
&=&\sum_{n=1}^\infty \frac{(-1)^{n-1}}{3n-2}
\end{eqnarray*}
\end{description}

\item 求$x=a\cos^3 t,y=a\sin^3 t,a>0$绕直线$y=x$旋转所围成的曲面的表面积
\begin{description}
\item[解] 硬解,画图观察,切成对称两个面
\begin{eqnarray*}
S &= & 2 \int_{\frac{\pi}{4}}^{\frac{3\pi}{4}} 2\pi \frac{|a\cos^3t-a\sin^3t|}{\sqrt{2}}\sqrt{x'^2(t)+y'^2(t)}\ud t\\
&=&6\sqrt{2}\pi a^2\left[\int_{\frac{\pi}{4}}^{\frac{3\pi}{4}}|\cos^3t-\sin^3t||\sin t\cos t| \right]\ud t
\end{eqnarray*}
\end{description}


















\end{enumerate}