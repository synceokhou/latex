\section{2000数分}
\begin{enumerate}

\item 定义函数
\[
f(x,y)=\begin{cases}
\frac{x^3}{x^2+y^2} & x^2+y^2>0 \\
0 & x=y=0
\end{cases}
\]
证明函数$f(x,y)$在$(0,0)$处连续但是不可微

\item 设$f_n(x)=x^n\ln x,n\in \mathbf{N}^+$
\begin{description}
\item[(1)] 证明
\[
\frac{f_n^{(n)}(x)}{n!} = \frac{f_{n-1}^{(n-1)}(x)}{(n-1)!}+\frac{1}{n},n=1,2,\cdots
\]
\item[(2)] 计算极限
\[
\lim_{n \to \infty} \frac{f_n^{(n)}(\frac{1}{n})}{n!}
\]
\end{description}

\item 在$\mathbf{R}^3$中,由$y=1,y=-x,x=0,z=0,z=-x$围成的闭区域记为$D$,计算$I=\iiint_D e^{x+y+z}\ud x\ud y\ud z$

\item 定义向量场
\[
\overrightarrow{F}(x,y)=\left(\frac{xe^{\sqrt{x^2+y^2}}}{\sqrt{x^2+y^2}},\frac{ye^{\sqrt{x^2+y^2}}}{\sqrt{x^2+y^2}} \right),x^2+y^2>0
\]
证明$\overrightarrow{F}(x,y)$是有势场,并求出$\overrightarrow{F}(x,y)$的一个势函数

\item 设
\[
f(x) = \sum_{n=0}^{\infty}\frac{1}{x+2^n},x\in[0,+\infty)
\]
证
\begin{description}
\item[(1)] $f(x)$在$[0,+\infty)$上连续
\item[(2)] $\lim_{x \to \infty}f(x)=0$
\item[(3)] $\forall x\in (0,+\infty)$有$0<f(x)-\frac{\ln(1+x)}{x\ln 2}<\frac{1}{1+x}$
\item[证(1)] 见南开大学数分下P135,$f(x)$在$[0,+\infty)$一致收敛,每一项又连续
\item[证(2)] 既然$\lim_{x \to \infty}$并且$\sum_{n=0}^{\infty}$,有很多个无穷,我们想方法把$n>N$之后的$\lim_{x \to \infty}$放缩放掉就好了
\item[证(3)] 整个时候要观察$\frac{\ln(1+x)}{x\ln 2}$和$\frac{1}{x+2^n}$的关系
\[
\int_0^\infty\frac{1}{x+2^y}\ud y = \frac{\ln(1+x)}{x\ln 2}
\]
于是,可以令
\[
I_n(x)=\int_n^{n+1} \frac{1}{x+2^y}\ud y,\frac{1}{x+2^{n+1}}<I_n(x)<\frac{1}{x+2^{n}}
\]
这个两边的不等式拆开,然后分别求和
\[
f(x)<\frac{1}{1+x}+\sum_{n=0}^\infty I_n(x)
\]
请考虑为何上式比下式多$\frac{1}{1+x}$
\[
f(x)>\sum_{n=0}^\infty I_n(x)
\]
\end{description}

\item 设$f(x)$在$[-a,a]$上有连续的导数,证
\[
\lim_{\lambda \to \infty}\int_{-a}^a \frac{1-\cos\lambda x}{x}f(x)\ud x=\int_0^a\frac{f(x)-f(-x)}{x}\ud x
\]
\begin{description}
\item[证] 首先可以拆开然后对称性变量代换
\[
=\int_0^a\frac{f(x)-f(-x)}{x}\ud x-\int_0^a\frac{f(x)-f(-x)}{x}\cos\lambda x\ud x
\]
后者$\frac{f(x)-f(-x)}{x}$是连续函数,由黎曼引理
\[
\lim_{\lambda\to\infty}2\int_0^a\frac{f(x)-f(-x)}{2x}\cos\lambda x\ud x=0
\]
\end{description}
















 
\end{enumerate}