\section{2006数分}
\begin{enumerate}

\item 设$f(x)$在$(a,+\infty)$内可导,且$\lim_{x \to a+}f(x)=\lim_{x \to +\infty}f(x)=\beta$,其中$\beta$为有限数或者$+\infty$或者$-\infty$。证明$\exists \xi \in (a,+\infty)$,使得$f'(\xi)=0$
\begin{description}
\item[证] 情况一:$\beta$为有限数,先补$f(a)$,$f(x)$与直线$y=A+\epsilon$或$y=A-\epsilon$至少有两个交点,然后经典罗尔定理;情况二:$\beta$为无穷,不妨设$\beta=+\infty$,令$F(x)=e^{-f(x)}$,则$F(x)$在$(a,+\infty)$内连续,且$\lim_{x \to a+}F(x)=\lim_{x \to +\infty}F(x)=0,F(x)>0(a<x<+\infty)$,且在$[a,+\infty)$上连续,并在$(a,+\infty)$上取得最大值。设最大值点为$b\in (a,+\infty)$,则有$e^{-f(b)}=F(b)\geq F(x)= e^{-f(x)}(a<x<+\infty)$,即$\forall x \in (a,+\infty),f(b)\leq f(x)$,故$f(b)$为$f(x)$在$(a,+\infty)$内的最小值,由费马定理知$f'(b)=0$(情况二证不对,要和一那样画线)
\end{description}

\item 计算积分
\begin{description}
\item[(1)]
\[
\int_0^1 \frac{\ln(1+x)}{1+x^2}\ud x
\]
\item[(2)]
\[
\int_{-\infty}^{+\infty}e^{-\frac{x^2}{2}} \ud x
\]
\item[(3)]
\[
\int_{-\infty}^{+\infty}\int_{-\infty}^{+\infty}e^{-[x^2+(y-x)^2+y^2]}\ud x\ud y
\]
\item[解(2)] 平方之后变二重积分再换元再开方
\item[解(3)] 拆括号,分两个,前一个用二,后一个直角坐标45度代换再用二
\end{description}

\item 设函数$f(x)$在$[0,1]$上二阶可导,且$f(0)=f(1)=0,\min_{0\leq x\leq 1}f(x)=-1$。证明$\exists \xi \in (0,1)$,使得$f''(\xi)\geq 8$
\begin{description}
\item[证] 两层拉格朗日
\end{description}

\item 证明
\begin{description}
\item[(1)] 任一实数列必为两个单调递增数列之差
\item[(2)] 设$\lim_{n \to \infty}a_n =a$,求
\[
\lim_{n \to \infty} \frac{\sum_{i=1}^nia_i}{n^2}
\]
\item[证(1)] 构造性证明。设$\{a_n\}$为任意数列,$\{b_n\}$为一递增数列,令$\{c_n\}$满足$c_n=a_n+b_n$,只要$c_n-c_{n-1}=(b_n-b_{n-1})+(a_n-a_{n-1})>0$即可,如
\[
b_n=
\begin{cases}
(n-1)+b_1+\sum_{i=1}^n|a_i-a_{i-1}| & n>1\\
b_1& n=1
\end{cases}
\]
和
\[
c_n=
\begin{cases}
(n-1)+b_1+\sum_{i=1}^n(|a_i-a_{i-1}|+a_i-a_{i-1}) & n>1\\
b_1+a_1 & n=1
\end{cases}
\]
\item[证(2)] 这题我回想到3个技巧,第一个是用$\alpha_n = a_n - a$,接下去用定义证,可以取另一条线,第二个是,记$S_n=\sum_{k=1}^na_k$,则$\sum_{k=1}^na_k=\sum_{k=1}^n[(k+(n-k))a_k-(n-k)a_k]=nS_n-\sum_{k=1}^{n-1}S_k$,第三个是,到这一步可以用stolz公式
\end{description}

\item 设$f(x)$在$[a,b]$上连续,且$\forall x \in [a,b],\exists y \in [a,b],\exists z \in (0,\frac{1}{2})$,使得$|f(y)|\leq z|f(x)|$。证$f(x)$在$[a,b]$上至少有一个零点
\begin{description}
\item[证] 反证,不妨设$f(x)>0$,且有最小值,在$x_0$处取到,又blabla,还有更小,矛盾
\end{description}

\item 求幂级数$\sum_{n=0}^{\infty}(3n+5)x^n$的收敛域,并求其和函数
\begin{description}
\item[解] 开根号或者比值,然后判断边界,最后拆开,积分,可见南开大学数分下P157
\end{description}

\item 计算二重积分$\iint_D\sqrt{|y-x^2|}\ud x\ud y$,其中$D$为区域$|x|\leq 1,0\leq y \leq 2$
\begin{description}
\item[解] 分块,只用算一边
\end{description}

\item 设$f(x),x\in \mathbf{R}^n$存在二阶连续偏导数,$\nabla f(x)$表示$f(x)$的剃度,$\nabla^2 f(x)=(h_{ij})_{n\times n}$表示$f(x)$的Hesse矩阵,其中
\[
h_{ij}=\frac{\partial^2 f}{\partial x_i \partial x_j}(i,j=1,2,\cdots,n)
\]
\begin{description}
\item[(1)] 设$a\in \mathbf{R}^n$是$f(x)$的稳定点,即$\nabla f(a)=0$。如果$\nabla^2 f(x)$在$x=a$处为正定的,证明$a$是$f(x)$的一个局部极小值点
\item[(2)] 设$f(x)$的Hesse矩阵在所有$x\in\mathbf{R}^n$点处正定,证明$f(x)$至多有一个稳定点
\item[证(1)] 定理原句,见南开大学数分下P110
\item[证(2)] 多元泰勒公式,在$x_0$展开,得出所有点大于$x_0$,再反证,设$x_1$的$\nabla f(x_1)=0$,再$x_1$到$x_0$,然后就矛盾
\end{description}





















\end{enumerate}