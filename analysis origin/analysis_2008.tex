\section{2008数分}
\begin{enumerate}
\item 设$f(x)\in C[0,1]$,证
\begin{description}
\item[(1)] $\exists x_0 \in [0,1]$,使得$\sin^2(\pi f(x_0))=x_0$
\item[解(1)] $g(x) = \sin^2(\pi f(x)) - x,g(0)\geq 0,g(1)\leq 0$
\item[(2)] 若$f(x)=x$,则$\sin^2(\pi x)=x$有且仅有三个根
\item[解(2)] $g(0)=0$
\end{description}

\item 设$f(x,y)$是定义在$D=[0,1]\times [0,1]$上的实值连续函数,求证$g(x)=\sup\{f(x,y)|0\leq y \leq 1\}$在$[0,1]$上连续
\begin{description}
\item[证] 见数学分析中的典型问题和方法P139,都是上确界,看看有什么区别
\end{description}

\item 设函数$f(x)$在$[0,1]$上二阶可导且$f''(x)\geq 0$,证明$\int_0^1 f(x)\ud x\geq f(\frac{1}{2})$

\item 设$\{x_n\},\{a_n\}$为两个非负无穷数列,满足$x_{n+1}\leq x_n +a_n$且$\sum_{n=1}^\infty a_n$收敛,证明$\lim_{n \to \infty}x_n$存在
\begin{description}
\item[证] $\{a_n\}$为非负无穷数列,那么$\{a_n\}$将渐渐趋向于零,这个可以由stolz证,也就是说,一旦$\{x_n\}$下降,它再次上升的空间越来越小,我们甚至可以断定如果趋势是下降的,那无话可说,有下界有极限,如果趋势是上升的,也只有最近一次下降之后的上升有意义。之前都是废话。可设$\exists M>0,\st \sum_{n=1}^\infty a_n \leq M$,因为$x_{n+1}\leq x_n +a_n$,所以$x_n-x_1=\sum_{k=1}^{n-1}(x_{k+1}-x_k)\leq \sum_{n=1}^\infty a_n \leq M$,从而$0\leq x_n\leq M+|x_1|$,说明$\{x_n\}$有界,有收敛子列,设$\{x_{n_k}\}$为一收敛子列,$x_{n_k}\to a$。找到$x_{n_{k-1}}-\epsilon<x_n<x_{n_{k}}+\epsilon$,另外有$|x_{n_{k}}-a|<\epsilon$和$|x_{n_{k+1}}-a|<\epsilon$,三个不等式合在一起就够了
\end{description}

\item 设实系数多项式序列$\{f_n(x)\}$在$\mathbf{R}$上一致收敛于实值函数$f(x)$,证明$f(x)$也是多项式

\item 计算曲面积分$I=\iint_{\Sigma}(x^2+x^7y^2+z^3)\ud S$,其中$\Sigma$为$x^2+y^2+z^2=1$

\item 证明
\[
I(r)=\int_0^{2\pi} \ln(1-2r\cos\theta+r^2)\ud \theta =
\begin{cases}
4\pi \ln r & r \geq 1,\\
0 & 0\leq r <1
\end{cases}
\]
\begin{description}
\item[证] 见吉米多维奇数学分析习题集题解4P193[3049]
\end{description}

\item 设函数$f(x,y)$在$\mathbf{R}^2$上二阶连续可微,满足$f|_{\partial \Omega}=0,\partial \Omega=\{(x,y)|x^2+y^2=1\}$,及$\lim_{x \to +\infty}f(x,0)=1$,证明$\exists (x_0,y_0)$,使得$\Delta f|_{(x_0,y_0)}\geq 0$


\item 设函数$f(x)$在$\mathbf{R}$上无穷次可微,且满足$f(0)=0,f'(0)\neq 0$。证明
\begin{description}
\item[(1)] $\exists \delta > 0$和定义在$(-\delta ,\delta)$上的可微函数$\psi(t)$,使得$f(\psi(t))=\sin t$
\item[(2)] 求$\psi(t)$在$t=0$处的二阶泰勒展开式
\end{description}

\item 设定义在$\mathbf{R}$上的非负连续函数$f(x)$,满足$f(x)\leq\int_0^x[f(t)]^a\ud t$,证明
\begin{description}
\item[(1)] 当$a\geq 1$时,$f(x)=0$
\item[(2)] 举例说明,当$0<a<1$时$f(x)$不一定恒为$0$
\end{description}














\end{enumerate}