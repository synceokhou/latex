\section{2012数分}
\begin{enumerate}
\item 计算极限
\begin{description}
\item[(1)]
\[
\lim_{n \to \infty} n^3\left(2\sin\frac{1}{n}-\sin\frac{2}{n} \right)
\]
\item[(2)]
\[
\lim_{n \to \infty} \left(\sqrt{\cos\frac{1}{x^2}}\right)^{x^4}
\]
\item[解(1)] 用$x=\frac{1}{n}$代换后能用洛必达
\[
=\lim_{x \to 0} \frac{2\sin x-\sin 2x}{x^3}
\]
\item[解(2)] 用$x=\frac{1}{x}$代换后,取对数,能用洛必达
\[
=\lim_{x \to 0} \frac{\ln (\cos x^2)}{2x^4}
\]
\end{description}

\item 计算积分
\begin{description}
\item[(1)]
\[
I=\int_0^{\frac{\pi}{2}} \frac{\ud x}{1+\tan^3 x}
\]
\item[(2)]
\[
J=\iint_S x(1+yf(x^2+y^2))\ud x \ud y
\]
其中$S$为曲线$y=x^3,y=1,x=-1$所围成的区域,$f(x)$为实值连续函数
\item[解(1)] 取代换$t = \frac{\pi}{2} - x$,然后两个加起来变1了

\end{description}

\item 求下列幂级数的收敛域
\[
\sum_{n=1}^{\infty} \frac{x^n}{1+\frac{1}{2}+\cdots +\frac{1}{n}}
\]
\item[解] 求出来收敛半径之后,要看收敛区间是闭还是开,正一的时候是调和级数,负一还是调和级数?

\item 证明函数列$s_n(x)=\frac{x}{1+n^2+x^2}(n>1)$在开区间$(-\infty,+\infty)$上一致收敛;函数列$t_n(x)=\frac{nx}{1+n^2+x^2}(n>1)$在开区间$(0,1)$上不一致收敛

\item 设在区间$[a,b]$上,$f(x)$连续,$g(x)$可积,并且$f(x)>0,g(x)>0$,证
\[
\lim_{n \to \infty} \left(\int_a^b f^n(x)g(x)\ud x \right)^{\frac{1}{n}} = \max_{a\leq x\leq b}f(x)
\]
\begin{description}
\item[证] 见数学分析精选习题全解上P216的223
\end{description}

\item 设在区间$[0,a]$上,$f(x)$二次可导,并且$|f(x)|\leq 1,|f''(x)|\leq 1$,则当$x\in [0,a]$时,$|f'(x)|\leq \frac{2}{a}+\frac{a}{2}$
\begin{description}
\item[证] 直接泰勒展开
\begin{eqnarray*}
f(x-h) &=& f(x)-hf'(x) +\frac{h^2}{2}f''(\xi)\\
f(x+h) &=& f(x)+hf'(x) +\frac{h^2}{2}f''(\eta)
\end{eqnarray*}
两式相减
\begin{eqnarray*}
f(x+h) - f(x-h) = 2hf'(x) +\frac{h^2}{2}[f''(\xi)-f''(\eta)]\\
2hf'(x) = f(x+h)- f(x-h) -\frac{h^2}{2}[f''(\xi)-f''(\eta)]\\
2h|f'(x)| \leq |f(x+h)|+ |f(x-h)| +\frac{h^2}{2}(|f''(\xi)|+|f''(\eta)|)
\end{eqnarray*}
其中$h<a,|f(x)|\leq 1,|f''(x)|\leq 1$
\end{description}

\item 设$n$是一个正整数,证$x^n+nx-1=0$有唯一正实根$x_n$,并且当$\alpha>1$时,级数$\sum_{n=1}^{\infty}x_n^{\alpha}$收敛
\begin{description}
\item[解] 令$f(x)=x^n+nx-1$,则$f(0)=-1,f(1)=n$,然后求导得单调增。$nx-1=-x^n>0$然后得$x<\frac{1}{n}$最后用大M判别法
\end{description}

\item 设$\omega(x,y,z)$是原点$O$到椭球面$\frac{x^2}{2}+\frac{y^2}{2}+z^2=1$的上半部分(满足$z\geq 0$)$\Sigma$的任一点$(x,y,z)$处的切面的距离,求积分
\[
\iint_{\Sigma}\frac{z}{\omega(x,y,z)}\ud S
\]
































\end{enumerate}