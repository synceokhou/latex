\section{2010数分}
\begin{enumerate}
\item 计算
\begin{description}
\item[(1)]
\[
\lim_{x \to 0}\frac{\int_0^{\sin^2x}\ln(1+t)\ud t}{\sqrt{1+x^4}-1}
\]
\item[(2)]
\[
\iint_{|x|+|y|\leq 1}|xy|\ud x\ud y
\]
\item[解(1)] 非常明显的洛必达
\[
= \lim_{x \to 0}\frac{\ln(1+\sin^2x)2\sin x\cos x}{\frac{1}{2}(1+x^4)^{-\frac{1}{2}}4x^3}
\]
\end{description}

\item 证明
\begin{description}
\item[(1)] 令
\[
f(x) =
\begin{cases}
x^2\sin\frac{1}{x} & x\neq0,\\
0 & x=0\\
\end{cases}
\]
求$f'(0)$,并证明$f'(x)$在$x=0$处不连续
\item[(2)] 若$\lambda = \sum_{k=1}^n \frac{1}{k}$,证明$e^{\lambda}>n+1$
\item[证(1)]
\[
f'(0) =\lim_{x\to 0} \frac{x^2\sin\frac{1}{x}-0}{x-0} =\lim_{x\to 0} x\sin\frac{1}{x}=0
\]
$f'(x)$在$x=0$处不连续很明显
\[
\lim_{x\to 0} f'(x) = \lim_{x\to 0} 2x\sin\frac{1}{x}+\cos\frac{1}{x}\neq f'(0)
\]
\item[证(2)] 很明显的归纳法
\end{description}

\item 若$f(x)$在$[0,1]$上连续,在$(0,1)$上二次可微,并且$f(0)=f(\frac{1}{4})=0$,以及$\int_{\frac{1}{4}}^1 f(y)\ud y=\frac{3}{4}f(1)$,证明$\exists \xi \in (0,1)$,使得$f''(\xi)=0$
\begin{description}
\item[证] 左边中值定理有$\frac{3}{4}f(\eta_1)=\frac{3}{4}f(1)$,接着找到$f'(\eta_2)=0$,又$f(0)=f(\frac{1}{4})$可以找到$f'(\eta_3)=0$,最后由$f'(\eta_3)=f'(\eta_2)=0$得$f''(\xi)=0$
\end{description}

\item 求级数$\sum_{n=1}^{\infty}\frac{n}{(n+1)!}$的和
\begin{description}
\item[解] 加一减一之后用幂级数
\end{description}

\item 证明
\[
\frac{2n}{3}\sqrt{n}<\sum_{k=1}^n\sqrt{k}<\left(\frac{2n}{3}+\frac{1}{2}\right)\sqrt{n}
\]
\begin{description}
\item[证] 见数学分析中的典型问题和方法P51。另,证左边,由拉格朗日中值定理,$\forall k \in \mathbb{N},\exists \xi_k\in(k-1,k),\st$
\[
\int_{k-1}^k \frac{3}{2}x^\frac{1}{2}\ud x = k^\frac{3}{2}-(k-1)^\frac{3}{2}=\frac{3}{2}\xi_k^\frac{1}{2}<\frac{3}{2}k^\frac{1}{2}
\]
求和有左边成立。右边用数学归纳法。又见数学分析中的典型问题和方法P273,这里有通过拉格朗日公式差成两项的方法
\end{description}

\item 计算
\[
\iiint_V (x^3+y^3+z^3)\ud x \ud y \ud z
\]
其中$V$表示曲面$x^2+y^2+z^2-2a(x+y+z)+2a^2=0(a>0)$所围成的区域

\item 应用Green公式计算积分
\[
I=\oint_L \frac{e^x(x\sin y-y\cos y)\ud x + e^x(x\cos y-y\sin y)\ud y}{x^2+y^2}
\]
其中$L$是包围原点的简单光滑闭曲线,逆时针方向

\item 设$f(x)$定义在$(-\infty,\infty)$上,且在$x=0$连续,并且对所有$x,y\in (-\infty,\infty)$,有$f(x+y) = f(x)+f(y)$,证明$f(x)$在$(-\infty,\infty)$上连续,且$f(x)=f(1)x$

\item 证明
\[
\int_0^1 \frac{\ud x}{x^x}=\sum_{n=1}^{\infty} \frac{1}{n^n}
\]

\item 设$f(x)$在$[0,1]$上连续且$f(x)>0$,讨论函数
\[
g(y)=\int_0^1\frac{yf(x)}{x^2+y^2}\ud x
\]
在$(-\infty,\infty)$上的连续性
\begin{description}
\item[证] 数学分析精选习题全解上P277的278,也可以参考数学分析中的典型问题和方法P317
\end{description}

































\end{enumerate}