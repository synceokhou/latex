\section{2014数分}
\begin{enumerate}
\item 求极限
\[
\lim_{n \to \infty} \left( \frac{\sin \frac{\pi}{n}}{n+1} + \cdots + \frac{\sin\pi}{n+\frac{1}{n}}\right)
\]
\begin{description}
\item[解] 见吉米多维奇数学分析习题集题解三[2230],见数学分析中的典型问题和方法P57
\[
\frac{1}{n+1} \sum_{i=1}^n \sin\frac{i}{n}\pi \leq \sum_{i=1}^n \frac{\sin\frac{i}{n}\pi}{n+\frac{i}{n}} \leq \frac{1}{n+\frac{1}{n}} \sum_{i=1}^n \sin\frac{i}{n}\pi
\]
左边
\[
\lim_{n\to \infty} \frac{n}{(n+1)\pi} \frac{\pi}{n}\sum_{i=1}^n\sin\frac{i}{n}\pi = \frac{1}{\pi}\int_0^\pi \sin x\ud x = \frac{2}{\pi}
\]
右边
\[
\lim_{n\to \infty} \frac{1}{(1+\frac{1}{n^2})\pi} \frac{\pi}{n}\sum_{i=1}^n\sin\frac{i}{n}\pi = \frac{1}{\pi}\int_0^\pi \sin x\ud x = \frac{2}{\pi}
\]
\end{description}

\item $f(x)=(1+x)^{\frac{1}{x}}$,求$f'(0),f''(0),f'''(0)$
\begin{description}
\item[解]
\begin{eqnarray*}
\ln f(x)&=&\frac{\ln(1+x)}{x}\\
\frac{f'(x)}{f(x)}&=&\frac{\frac{1}{1+x}x-\ln(1+x)}{x^2}\\
&=&\frac{x-(1+x)\ln(1+x)}{(1+x)x^2}
\end{eqnarray*}
\end{description}

\item $\psi(y)$是$f(x)$的反函数,求用$f',f'',f'''$表示$\psi',\psi'',\psi'''$
\begin{description}
\item[解] 令$y=f(x),x=\psi(y)$,则有
\begin{eqnarray*}
f'(x) &=& \frac{\ud y}{\ud x}\\
\psi'(y) &=& \frac{\ud x}{\ud y} = \frac{1}{f'(x)}\\
\psi''(y) &=& \frac{\ud^2x}{\ud y^2} \\
&=& \frac{\ud (\frac{\ud x}{\ud y})}{\ud y}\\
&=& \frac{\ud (\frac{\ud x}{\ud y})}{\ud x} \frac{\ud x}{\ud y}\\
&=& \frac{-f''(x)}{(f'(x))^2} \frac{1}{f'(x)}\\
\psi'''(y) &=& \frac{\ud (\frac{\ud^2x}{\ud y^2})}{\ud y}\\
&=& \frac{\ud (\frac{\ud^2x}{\ud y^2})}{\ud x} \frac{\ud x}{\ud y}
\end{eqnarray*}
\end{description}

\item $f$在$[0,1]$上连续,$(0,1)$上可导,$f(0)=0,f(1)=\frac{1}{2}$,求证$\exists \xi,\eta \in (0,1),f'(\xi)+f'(\eta)=\xi+\eta$
\begin{description}
\item[证] 由$f'(\xi)-\xi$的格式可以猜中值定理的原式的样子$g(x)=f(x)-\frac{x^2}{2}$,再看有左右两边,可以$h(x)=g(x)-g(1-x)$,再用两端点做中值,$h(0)=h(1)=0$,故$\exists \xi \in (0,1),\st h'(\xi)=0$
\end{description}

\item $a_0 > 0,a_n=\sqrt{a_{n-1}+6}$,求极限
\begin{description}
\item[证] 见南开大学数分一P41,先$a_n-a_{n-1}$,判断单调性,然后放缩可得不等式解出$a_n$的界,然后在等式两端取极限。另解
\begin{eqnarray*}
a_n-a_{n-1} &=& \sqrt{a_{n-1}+6} - \sqrt{a_{n-2}+6}\\
&=& \frac{a_{n-1}+6-(a_{n-2}+6)}{\sqrt{a_{n-1}+6} + \sqrt{a_{n-2}+6}}
\end{eqnarray*}
可以看出分母是大于1的,于是$a_n$之间的距离是不断缩小的。
\end{description}

\item $1\leq \iint_\omega \sin x^2+\cos y^2 \ud x \ud y\leq \sqrt{2}$,其中$\omega$为$\{0\leq x\leq 1,0\leq y\leq 1\}$
\begin{description}
\item[证] 令原式的值为$I$
\begin{eqnarray*}
I & = & \iint_\omega \sin x^2 \ud x \ud y+\iint_\omega \cos y^2 \ud x \ud y\\
& = & \int_0^1 \sin x^2 \ud x + \int_0^1 \cos y^2 \ud y\\
& = & \int_0^1 \sin x^2 \ud x + \int_0^1 \cos x^2 \ud x\\
& = & \int_0^1 (\sin x^2 + \cos x^2) \ud x\\
& = & \int_0^1 \sqrt{2}\sin (x^2+\frac{\pi}{4}) \ud x
\end{eqnarray*}
又因为$x^2+\frac{\pi}{4}$能取到$\frac{\pi}{4}$和$\frac{\pi}{2}$
\end{description}

\item 求$z=x+y$与$z=x^2+y^2$围成的面积
\begin{description}
\item[解] $z_1-z_2$然后积平面
\end{description}

\item 讨论
\[
\sum_{n=0}^{\infty} \frac{(x^2+x+1)^n}{n(n+1)}
\]
收敛性与一致收敛性

\item $u(x)$为在$[0,1]$上连续,$u'(x)$绝对可积,求证
\[
\sup_{x\in [0,1]} |u(x)| \leq \int_0^1 |u(x)|\ud x+\int_0^1 |u'(x)|\ud x
\]
$u(x,y)$在$\Omega = \{0\leq x\leq 1,0\leq y\leq 1\}$上连续,$\frac{\partial u}{\partial x},\frac{\partial u}{\partial y},\frac{\partial^2 u}{\partial x\partial y}$绝对可积,求证
\[
\sup_{(x,y)\in \Omega} |u|\leq\iint_\Omega |u|\ud x\ud y+\iint_\Omega\left( \left|\frac{\partial u}{\partial x}\right|+\left|\frac{\partial u}{\partial y}\right|\right)\ud x\ud y+\iint_\Omega\left|\frac{\partial^2 u}{\partial x\partial y}\right|\ud x\ud y
\]
\begin{description}
\item[证(1)] 见数学分析精选习题全解上P263的263,另见数学分析中的典型问题和方法P359。$\exists \xi,x_0\in[0,1],\st$
\[
|f(\xi)| = \left|\int_0^1 f(x)\ud x \right|
\]
于是
\[
\sup_{x\in [0,1]}|f(x)| = |f(x_0)| = \left|\int_\xi^{x_0} f'(t)\ud t +f(\xi)\right|
\]
\item[证(2)] 按照上面看来,要二元的中值定理
\end{description}















\end{enumerate}