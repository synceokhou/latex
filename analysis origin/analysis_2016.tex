\section{2016数分}
\begin{enumerate}
\item 计算极限
\[
\lim_{x \to 0}\left( \frac{e^x+e^{2x}+\cdots + e^{nx}}{n} \right)^{\frac{1}{x}}
\]
\begin{description}
\item[解]
\begin{eqnarray*}
 & = & \lim_{x \to 0} \exp\left[\frac{1}{x} \ln\left( \frac{e^x+e^{2x}+\cdots + e^{nx}}{n} \right)\right] \\
 & = & \lim_{x \to 0} \exp\left[\frac{1}{x} \ln\left(1+ \frac{(e^x-1)+(e^{2x}-1)+\cdots + (e^{nx}-1)}{n} \right)\right] \\
 & = & \lim_{x \to 0} \exp\left[\frac{(e^x-1)+(e^{2x}-1)+\cdots + (e^{nx}-1)}{nx} \right] 
\end{eqnarray*}
\end{description}

\item 求定积分
\[
I = \int_0^1 \log(1+\sqrt{x}) \ud x
\]
\begin{description}
\item[解] 令$t=\sqrt{x}$,则$t\in [0,1]$
\[
I = \int_0^1 \log(1+t)\ud t^2 = t^2\log(1+t)|_0^1-\int_0^1 \frac{t^2}{1+t}\ud t
\]
后者加1减1即可积出
\end{description}

\item 求二重极限
\[
\lim_{ \substack {x \to \infty \\ y \to \infty} } \frac{x+y}{x^2-xy+y^2}
\]
\begin{description}
\item[解] 由$x^2+y^2 \geq 2xy$
\begin{eqnarray*}
\left|\frac{x+y}{x^2-xy+y^2}\right| & \leq  & \frac{|x+y|}{x^2+y^2-|xy|}\\
& \leq  & \frac{|x+y|}{x^2+y^2-\frac{1}{2}(x^2+y^2)}\\
& \leq  & 2\frac{|x|+|y|}{x^2+y^2}\\
& \leq  & 2\frac{|x|+|y|}{2|x||y|}
\end{eqnarray*}
\end{description}

\item $f(x)$是[a,b]上的连续正函数,求证$\exists \xi \in (a,b)$,使得
\[
\int_a^{\xi} f(x)\ud x = \int_{\xi}^b f(x)\ud x = \frac{1}{2}\int_a^b f(x)\ud x
\]
\begin{description}
\item[证] 令$g(t) = \int_a^t f(x)\ud x - \frac{1}{2}\int_a^b f(x)\ud x$,显然$g(x)$连续,再由零点定理。另证
\[
\int_a^{\xi} f(x)\ud x - \int_{\xi}^b f(x)\ud x = \int_a^{\xi} f(x)\ud x + \int_b^{\xi} f(x)\ud x
\]
由此,可记后者为$F(\xi)$,又$F(a)<0,F(b)>0$,由零点定理可知。
\end{description}

\item 求以下曲面所围立体的体积
\[
\begin{array}{l}
S_1:\frac{x^2}{a^2}+\frac{y^2}{b^2}+\frac{z^2}{c^2} =1\\
S_2:\frac{x^2}{a^2}+\frac{y^2}{b^2} = \frac{z^2}{c^2}(z\geq 0)
\end{array}
\]
\begin{description}
\item[解] 我的方法是二重积分,因为$z\geq 0$。所以表示出$z_1-z_2$之后用极坐标替换变量,注意求出交线,以确定平面积分范围
\end{description}

\item $f(x)$是$[a,b]$上的连续函数,且$f(x)$单调递增。求证
\[
\int_a^b tf(t)\ud t \geq \frac{a+b}{2}\int_a^b f(t)\ud t
\]
\begin{description}
\item[证] 令$x=t-\frac{a+b}{2}$,原式变为
\[
\int_0^{\frac{b-a}{2}} x\left(f(\frac{a+b}{2}+x)-f(\frac{a+b}{2}-x)\right) \ud x
\]
数学分析中的典型问题和方法P370的例子4.3.32
\end{description}

\item 若数列$\{a_n\},\{b_n\}$满足以下条件
\begin{description}
\item[(1)] $a_1 \geq a_1 \geq \cdots$且$\lim_{n \to \infty}a_n = 0$
\item[(2)] $\exists M \in \mathbf{R^+},\forall n \in \mathbf{Z^+}$,有$|\sum_{k=1}^n b_k|\leq M$
\end{description}
证明$\sum_{n=1}^\infty a_nb_n$收敛
\begin{description}
\item[证] 阿贝尔变换的式子,把$\sum_{i=m+1}^n a_nb_n$拆开变成$|\sum_{i=m+1}^{n-1}(a_i-a_{i-1})B_i+a_nB_n-a_{m+1}B_m|$
\end{description}

\item 设$0\leq a < \frac{b}{2}$,$f(x)$在$[a,b]$上连续,在$(a,b)$上可导且$f(a)=a,f(b)=b$
\begin{description}
\item[(1)] 求证$\exists \xi \in (a,b)$,使得$f(\xi) = b -\xi$
\item[(2)] 若$a=0$,求证$\exists \alpha,\beta \in (a,b),\alpha\neq\beta$,使得$f'(\alpha)f'(\beta)=1$
\item[证(1)] 令$F(x)=f(x)-b+x$,则由$F(a)=2a-b<0,F(b)=b>0$和零点定理。
\item[证(2)] 刚才的$\xi$两边用拉格朗日
\begin{eqnarray*}
\frac{b-\xi}{\xi} &=& \frac{f(\xi)-f(0)}{\xi}=f'(\alpha)\\
\frac{b-(b-\xi)}{b-\xi} &=& \frac{f(b)-f(\xi)}{b-\xi}=f'(\beta)
\end{eqnarray*}
上下两式乘
\end{description}

\item 求椭圆$x^2+4y^2=4$上到直线$2x+3y=6$距离最短的点,并求其最短距离
\begin{description}
\item[解] 这是一题函数最值的应用,按照题设的做法是拉格朗日乘数法,先列出等式了。但是解的时候要找切线,然后求椭圆的切线斜率与直线斜率相等的点,显然会得出两个点,反过来求拉格朗日的参数
\end{description}

\item 半径为$R$的球面$S$的球心在单位球面$x^2+y^2+z^2=1$上,求球面$S$在单位球内面积的最大值,并求出此时的$R$
\begin{description}
\item[解] 把球心放在单位圆与坐标轴交点上,画图可求出在单位圆内的角度,然后可以重现建坐标系,以$S$的球心为原点用极坐标算面积
\end{description}
























\end{enumerate}