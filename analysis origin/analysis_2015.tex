\section{2015数分}
\begin{enumerate}
\item 已知数列$x_n$且$\lim_{n \to \infty} x_n = \infty$,证明$\lim_{n \to \infty}\frac{x_1+x_2+\cdots+x_n}{n}=\infty$
\begin{description}
\item[证] 缺条件了吧,举个反例,当$x_n=(-1)^nn$时。所以可以补个条件,见数学分析中的典型问题和方法P25,吉米多维奇数学分析习题集题解一P40[139]。不妨设$x_n>0$,$\forall M>0,\exists N>0,\st n>N,x_n>M$
\begin{eqnarray*}
\frac{x_1+x_2+\cdots+x_n}{n} - A& = &\frac{x_1-A+x_2-A+\cdots+x_n-A}{n}\\
& > &\frac{x_1-A+x_2-A+\cdots+x_N-A}{n}+ \frac{M(n-N)}{n}\\
& > & \frac{x_1-A+x_2-A+\cdots+x_N-A}{n}+ \frac{nM}{n}
\end{eqnarray*}
\end{description}


\item 求极限$\lim_{\substack{x \to 0 \\ y \to 0}} x^2y^2\ln(x^2+y^2)$
\begin{description}
\item[解] $2xy<x^2+y^2=z$
\end{description}

\item 已知
\[
f(x) = \begin{cases}
ax+b &  x < 0 \\
x^2+1 &  x \geq 0
\end{cases}
\]
若$f(x)$在$x=0$处导数连续,求$a,b$的值
\begin{description}
\item[证] 首先$f(x)$在$x=0$处连续,$f_+(0) = 0+1=0=f(0)=f_-(0)=0+b$。然后$f(x)$在$x=0$处导数连续,$f_+'(0) = 0^2=0=f_-'(0)=a$
\end{description}


\item 已知$f(x)$在$[a,b]$上连续,$(a,b)$内可导,且$f(a)=f(b)=0,\exists c \in (a,b)$,使得$f(c)>0$,证明$\exists \xi \in (a,b)$,使得$f''(\xi)<0$
\begin{description}
\item[证] 见数学分析中的典型问题和方法P267的3.3.5,由$f(a)=f(b)=0,\exists c \in (a,b)$,使得$f(c)>0$知最大值必在$(a,b)$内,设为$x_0$,$f'(x_0)=0,f(x_0)>0$,此时
\[
f(b)= f(x_0) +\frac{f''(\xi)}{2}(b-x_0)^2,\xi \in (x_0,b)
\]
即得$f''(\xi)<0$
\end{description}

\item 证明
\[
f(x)=\sum_{n=1}^\infty \frac{\sin(nx)}{n^3}
\]连续,且有连续的导函数
\begin{description}
\item[证] 大M判别法,之后有定理直接证有连续的导函数?
\end{description}

\item 已知一个半径为$r$的球,求与之相切的正圆锥的最小体积
\begin{description}
\item[解] 圆锥在球外面,画图确定圆锥的顶角,得出体积函数
\end{description}

\item 设$p(x)$是不超过三次的多项式,证明$\forall a,b ,\int_a^bf(x)\ud x =$

\item 求$\iint_D xy\ud x\ud y$,其中$D$由$y=x^2$与$y=\frac{x}{2}$围成$x>0$的部分

\item 已知空间中两点$A(1,0,-1),B(0,1,1)$,求线段$AB$绕$z$轴旋转一周与$z=1,z=-1$所围体积

\item 已知$f(x)$在$[0,1]$上单调递减,证明$\forall \lambda \in (0,1)$,有
\[
\int_0^\lambda f(x)\ud x \geq \lambda \int_0^1 f(x)\ud x
\]
\begin{description}
\item[证]
\begin{eqnarray*}
\int_0^\lambda f(x)\ud x \geq \lambda \int_0^{\lambda} f(x)\ud x + \lambda \int_{1-\lambda}^1 f(x)\ud x \\
(1-\lambda)\int_0^\lambda f(x)\ud x \geq \lambda \int_{1-\lambda}^1 f(x)\ud x \\
\frac{\int_0^\lambda f(x)\ud x}{\lambda} \geq \frac{\int_{1-\lambda}^1 f(x)\ud x}{1-\lambda}
\end{eqnarray*}
然后中值定理
\end{description}

\item 证明$\frac{\pi}{4}(1-\frac{1}{e})<(\int_0^1 e^{-x^2}\ud x)^2<\frac{16}{25}$
\begin{description}
\item[证] 将一元积分换成二元的,然后换极坐标,然后放缩,不用算复杂积分
\begin{eqnarray*}
(\int_0^1 e^{-x^2}\ud x)^2 & = & \int_0^1 \int_0^1 e^{-(x^2+y^2)}\ud x\ud y\\
& > & \iint_{\{x^2+y^2\leq 1,x,y>0\}}e^{-(x^2+y^2)}\ud x\ud y
\end{eqnarray*}
另一边
\begin{eqnarray*}
(\int_0^1 e^{-x^2}\ud x)^2 & = & \iint_{\{x^2+y^2\leq 1,x,y>0\}}e^{-(x^2+y^2)}\ud x\ud y\\ 
& &+\iint_{\{x^2+y^2> 1,0<x,y<1\}}e^{-(x^2+y^2)}\ud x\ud y
\end{eqnarray*}
\end{description}


\end{enumerate}