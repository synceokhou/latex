\section{2007数分}
\begin{enumerate}
\item 求幂级数
\[
\sum_{n=0}^{\infty}\frac{n^2+1}{2^nn!}x^n
\]
的收敛域,并求其和
\begin{description}
\item[解] 注意不要让分母的$n!$变成$(n-1)!$,因为求和从零开始,具体的拆分与积分方法见南开大学数分下P158
\begin{eqnarray*}
\sum_{n=0}^{\infty}\frac{n^2+1}{2^nn!}x^n=S_1(y)-\sum_{n=0}^{\infty}\frac{2n}{n!}y^n\\
S_1(x)=\sum_{n=0}^{\infty} \frac{(n+1)^2x^n}{n!}\\
\frac{S_2(x)}{x}=\sum_{n=1}^{\infty} \frac{nx^{n-1}}{n!}
\end{eqnarray*}
\end{description}

\item 讨论
\[
\int_0^{+\infty}\frac{e^{\sin x}\sin2x}{x^p}\ud x
\]
的绝对收敛和条件收敛
\begin{description}
\item[证] 见南开大学数分下P108的17(4)
\end{description}

\item 计算$\iint_{\Sigma}yz\ud y \ud z+(x^2+z^2)y\ud z \ud x+xy\ud x \ud y$,其中$\Sigma$为$4-y=x^2+z^2$在$xoz$平面的右侧部分的外侧
\begin{description}
\item[证] 补全$xoz$平面上的东西,然后求三重积分
\end{description}

\item 证明
\begin{description}
\item[(1)] $x^n(1-x)<\frac{1}{ne}(0<x<1,n\in\mathbf{N^+})$
\item[(2)] $x^y+y^x>1(x,y>0)$
\item[证(1)] 非常经典的求导
\item[证(2)] 先证伯努利不等式$(1+x)^r>1+rx(x>0,r>1)$,令$f(x)=(1+x)^r-rx-1$,则$f'(x)=r(1+x)^{(r-1)}-r=r[(1+x)^{(r-1)}-1]$。因为$1+x>1,r-1>0$,所以$(1+x)^{(r-1)}>1^{(r-1)}=1$,所以$f'(x)>0$,即$f(x)$单调递增,$f(x)>f(0)=0,(1+x)^r>1+rx(x>0,r>1)$。回到原题,注意到$\frac{y}{x}>0,\frac{1}{y}>1$,在引理中令$x$取$\frac{y}{x}$,$r$取$\frac{1}{y}$即得$(1+\frac{y}{x})^{(\frac{1}{y})}>1+\frac{1}{x}>\frac{1}{x}$,所以$1+\frac{y}{x}>(\frac{1}{x})^y=\frac{1}{x^y}>0$,所以$x^y>\frac{x}{x+y}$,同理可证另一个,然后加起来

\end{description}

\item 设级数$\sum_{n=1}^{\infty}b_n$收敛,且$\sum_{n=1}^{\infty}(a_n-a_{n-1})$绝对收敛,证明$\sum_{n=1}^{\infty}a_nb_n$收敛
\begin{description}
\item[证] 阿贝尔求和法见南开大学数分下P26
\end{description}

\item 假设$f(x)$为二次连续可微实值函数,对于所有实数$x$,满足$|f(x)|\leq 1$且满足$(f(0))^2+(f'(0))^2=4$。证明存在实数$x_0$,满足$f(x_0)+f''(x_0)=0$
\begin{description}
\item[证] 中值定理
\end{description}

\item 假设$|f(x)|\leq 1$和$|f''(x)|\leq 1$对一切$x\in [0,2]$成立,证在$[0,2]$上有$|f'(x)|\leq 2$
\begin{description}
\item[证] 泰勒展开吧
\end{description}

\item 设$D=[0,1]\times[0,1],f(x,y)$是定义在$D$上的二元函数,$f(0,0)=0$,且$f(x,y)$在$(0,0)$处可微,求
\[
\lim_{x \to 0+}\frac{\int_0^{x^2}\ud t\int_x^{\sqrt{t}}f(t,u)\ud u}{1-e^{-\frac{x^4}{4}}}
\]

\item 设$-\infty<x_0<\infty$,$\psi(x)$和$f(x)$在$[x_0,x_0+h]$上连续,且存在$M>0,K>0$,使得
\[
|\psi(x)|\leq M\left(1+K\int_{x_0}^x|\psi(t)f(t)|\ud t\right),x\in(x_0,x_0+h)
\]
证明$\psi(x)$必满足
\[
|\psi(x)|\leq M\exp\left(KM\int_{x_0}^x|f(t)|\ud t\right),x\in(x_0,x_0+h)
\]


\item 设$\alpha \in(0,1)$,记$e=(1,1,\cdots,1)^T\in\mathbf{R}^n,S(\frac{e}{n},\frac{\alpha}{n})=\left\{ x\in\mathbf{R}^n:\|x-\frac{e}{n}\|\leq \frac{\alpha}{n} \right\}$,对于$x\in S (\frac{e}{n},\frac{\alpha}{n})$且$e^Tx=1$,证明
\[
-\sum_{i=1}^n\ln x_i\leq n\ln n+\frac{\alpha^2}{2(1-\alpha)^2}
\]




















\end{enumerate}