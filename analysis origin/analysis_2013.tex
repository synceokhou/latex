\section{2013数分}
\begin{enumerate}
\item 计算
\begin{description}
\item[(1)] $\lim_{n\to \infty} \sin^2(\pi\sqrt{n^2+n})$
\item[解]
\begin{eqnarray*}
&=& \lim_{n\to \infty}\sin^2(\pi\sqrt{n^2+n}-\pi n)\\
&=& \lim_{n\to \infty}\sin^2\left(\frac{n\pi}{\sqrt{n^2+n}+n} \right)\\
&=& \lim_{n\to \infty}\sin^2\frac{\pi}{\sqrt{1+\frac{1}{n}}+1}\\
& =& \sin^2 \left(\lim_{n\to \infty}\frac{\pi}{\sqrt{1+\frac{1}{n}}+1}\right)\\
&=& \sin^2\frac{\pi}{2} = 1
\end{eqnarray*}

\item[(2)] $\lim_{n\to \infty} a_n$,其中设$a_1=1,a_{n+1}=1+\frac{1}{a_n}$
\item[解] 见数学分析中的典型问题和方法P95,首先解$A=1+\frac{1}{A}$,并且发现$a_n>0$。再通分$|a_{n+1}-a_n|$,然后发现$a_na_{n+1}>1$,相邻项之间间距缩小
\end{description}

\item 设$f(x)$连续,$g(x)=\int_0^x f(x-t)\sin t\ud t$,证$g''(x)+g(x)=f(x),g(0)=g'(0)=0$
\begin{description}
\item[证]
\begin{eqnarray*}
g'(x) &=& \int_0^x f'(x-t)\sin t\ud t + f(0)\sin x\\
g''(x) &=& \int_0^x f''(x-t)\sin t\ud t + f'(0)\sin x + f(0)\cos x\\
\end{eqnarray*}
显然$g(0)=g'(0)=0$
\begin{eqnarray*}
\int_0^x f(x-t)\sin t\ud t &=& -\int_0^x f(x-t)\ud \cos t \\
&=& - f(x-t) \cos t |_0^x + \int_0^x f'(x-t) \cos t \ud t\\
\end{eqnarray*}
后者
\begin{eqnarray*}
\int_0^x f'(x-t)\ud \sin t = f'(x-t) \sin t |_0^x - \int_0^x f''(x-t) \sin t \ud t
\end{eqnarray*}
好像符号不太对,再检查下
\end{description}

\item 求曲线$y=e^x$的曲率的最大值
\begin{description}
\item[解] 由曲率公式
\[
\kappa = \frac{|y''|}{(1+y'^2)^\frac{3}{2}} (y''\neq 0)
\]
即
\begin{eqnarray*}
\kappa &=& \frac{e^x}{(1+e^{2x})^\frac{3}{2}}\\
\kappa' &=& \frac{e^x(1+e^{2x})^\frac{3}{2}-3e^x(1+e^{2x})^\frac{1}{2}e^{2x}}{(1+e^{2x})^3}
\end{eqnarray*}
解得$x=\ln(\sqrt{\frac{1}{2}})$
\end{description}

\item 计算积分
\begin{description}
\item[(1)]
\[
I = \int_{\frac{1}{4}}^{\frac{1}{2}}\ud y \int_{\sqrt{y}}^{\frac{1}{2}}e^{\frac{y}{x}} \ud x + \int_{\frac{1}{2}}^1 \ud y \int_{\sqrt{y}}^1 e^{\frac{y}{x}} \ud x
\]
\item[(2)]
\[
J = \iint_{\omega} |x^2+y^2-1|\ud x \ud y
\]
其中$\omega=\{(x,y)|x\leq x\leq 1,0\leq y\leq 1\}$
\end{description}

\item 讨论级数$\sum_{n=1}^{\infty}\frac{n^2}{(x+\frac{1}{n})^n}$的收敛性和一致收敛性(包括内闭一致收敛性)

\item 证明
\begin{description}
\item[(1)] 当$0<x<\frac{\pi}{2}$时,$\frac{2}{\pi}<\frac{\sin x}{x}<1$
\item[解] 右边$f(x)=\sin x-x$,则$f(0)=0$,$f'(x)=\cos x -1$,$x>0$时,$f'(x)<0$。左边$g(x)=\pi\sin x-2x$,则$g(\frac{\pi}{2})=g(0)=0$,$g'(x)=\pi\cos x-2=0$,得$x=?$,所有极值点都是大于0的
\item[(2)] 设函数$f(x)$在闭区间$[a,b]$上二次可微,且$f''(x)<0$,则
\[
\frac{f(a)+f(b)}{2} \leq \frac{1}{b-a}\int_a^b f(t)dt
\]
\item[解] 看上凸函数的性质然后就是梯形面积和积分面积的比较,把梯形斜边表示出来,然后用$f(t)$去减,再用上凸函数的性质。另见数学分析中的典型问题和方法P365的4.3.26
\end{description}

\item 求函数$f(x,y)=x^2+y^2+\frac{3}{2}x +1$在集合$G=\{(x,y)\in \mathbf{R}^2|4x^2+y^2-1=0\}$上的最值

\item 设无穷实数列$\{a_n\},\{b_n\}$满足
\[
a_{n+1} = b_n - \frac{na_n}{2n+1}
\]
证
\begin{description}
\item[(1)] 若$\{b_n\}$有界,则$\{a_n\}$也有界
\item[(2)] 若$\{b_n\}$收敛,则$\{a_n\}$也收敛
\item[证(1)] 不会
\item[证(2)] 用上面的有界,然后经典的$a_{n+1}-kb=-\frac{1}{2}(a_n-kb)+\epsilon_n$

\end{description}
























\end{enumerate}