\section{2015高代}
\begin{enumerate}
\item 设$x_1,x_2,\cdots,x_n$两两相异,$f(x) = (x-x_1)(x-x_2)\cdots(x-x_n)$
\[
s_k = x_1^k + x_2^k + \cdots + x_n^k
\]
求证:
\begin{description}
\item[(1)] $f'(x)^2-f(x)f''(x)$无实根
\item[(2)] 如果
\[
x^{k+1}f'(x) = (s_0x^k + s_1x^{k-1} + \cdots + s_k)f(x) + g(x)
\]
则$g(x)$次数小于$n$
\item[证(1)] 这是一道非常直接的题目,甚至让人错觉是数分题
\[
f^2(x)\left(\frac{f'(x)}{f(x)}\right)' = f'(x)^2-f(x)f''(x)
\]
然后
\[
f^2(x)\left(-\sum\frac{1}{(x-x_i)^2}\right) <0 ,\qquad \forall x \in \mathbf{R}
\]
\item[证(2)] 出题人应该是从某一现实计算中发现技巧然后硬生生凑出来的
\begin{eqnarray*}
g(x) & = & x^{k+1}f'(x)-(s_0x^k+s_1x^{k-1}+\cdots+s_k)f(x) \\
     & = & f(x)\sum_{j=1}^k\frac{1}{x-x_j}-\sum_{l=0}^k\sum_{j=1}^nx_j^lx^{k-l}f(x) \\
     & = & x^{k+1}f(x)\sum_{j=1}^k\frac{1}{x-x_j}-x^kf(x)\sum_{l=0}^k\sum_{j=1}^n(\frac{x_j}{x})^l \\
     & = & f(x)\sum_{j=1}^k\frac{x^{k+1}}{x-x_j} - x^kf(x)\sum_{j=1}^n\frac{1-(\frac{x_j}{x})^{k+1}}{1-\frac{x_j}{x}} \\
     & = & f(x)\sum_{j=1}^k\frac{x^{k+1}}{x-x_j} - f(x)\sum_{j=1}^n\frac{x^{k+1}-x^{k+1}_j}{x-x_j} \\
     & = & f(x)\sum_{j=1}^n\frac{x^{k+1}}{x-x_j} \\
\end{eqnarray*}
\end{description}

\item $A_n=(a^{|i-j|})_{n\times n}$,求$|A_n|$和$|A_n^*|$
\begin{description}
\item[解] 第$i$行$\times (-a) +$第$i+1$行,记住是从下往上逐个进行,$|A_n|=(1-a^2)^{n-1}$。当$a\neq\pm1$时,$A_n^*=|A_n|An^{-1} \Rightarrow|A_n^*| = (1-a^2)^{(n-1)^2}$。当$a =\pm1$时,$\rank(A_n)=1$,所以$A_n^*=0$
\end{description}

\item
\[
A=\left(
\begin{array}{ccc}
a & 0 & 1 \\
-2 & 0 & 1 \\
2 & b & -1 \\
\end{array}
\right),
B=\left(
\begin{array}{ccc}
2 & -1 & 0 \\
0 & 0 & c \\
2 & b & 1 \\
\end{array}
\right)
\]
\begin{description}
\item[(1)] $a,b,c$为何值时,$AB=BA$
\item[(2)] $AB=BA$时,求$A,B$的公共单位特征向量
\item[解(1)]
\[
AB=\left(
\begin{array}{ccc}
2a+2 & b-a & 1 \\
-2 & 2+b & 1 \\
2 & -2-b & bc-1\\
\end{array}\right)
,
BA=\left(
\begin{array}{ccc}
2a+2 & 0 & 1 \\
2c & bc & -c \\
2a-2b+2 & b & 1+b\\
\end{array}\right)
\]
可知$a=b=c=-1$
\item[解(2)]
\[
\left(
\begin{array}{ccc}
-1-\lambda & 0 & 1 \\
-2 & -\lambda & 1 \\
2 & -1 & -1-\lambda\\
2-\mu & -1 & 0\\
0 & -\mu & -1\\
2 & -1 & 1-\mu\\
\end{array}\right)
\left(
\begin{array}{c}
x\\
y\\
z\\
\end{array}
\right)=0
\]
其中
\[
\left[
\begin{array}{cc}
-1 & 0 \\
-\mu & -1 \\
\end{array}
\right] = 1
\]
第二列和第三列线性无关,而前式有非零解,所以第一列可由第二列和第三列线性表出。
\[
\left(
\begin{array}{c}
x\\
y\\
z\\
\end{array}
\right) = \alpha
\left(
\begin{array}{c}
1\\
2-\mu\\
1+\lambda\\
\end{array}
\right)
\]
代入原式解出$\mu,\lambda$,再解出向量
\end{description}

\item $A^2 = -E$,则$\rank(A+iE)$与$\rank(A-iE)$满足什么关系,并证之
\begin{description}
\item[证] 
\begin{eqnarray*}
\left(
\begin{array}{cc}
A-iE & \\
 & A+iE \\
\end{array}
\right) & \rightarrow & \left(
\begin{array}{cc}
A-iE & A+iE\\
 & A+iE \\
\end{array}
\right) \\
& \rightarrow & \left(
\begin{array}{cc}
-2iE & A+iE\\
-A-iE & A+iE\\
\end{array}
\right) \\
& \rightarrow & \left(
\begin{array}{cc}
E & \frac{i}{2}A+\frac{1}{2}E\\
-A-iE & A+iE\\
\end{array}
\right) \\
& \rightarrow & \left(
\begin{array}{cc}
E & \frac{i}{2}A+\frac{1}{2}E\\
0 & 0 \\
\end{array}
\right) \\
\end{eqnarray*}
最后一步很简单,最后一行$\times(A-iE)$
\end{description}

\item
\[
A = \left(
\begin{array}{ccc}
-1 & -2 & 6 \\
-1 & 0 & 3 \\
-1 & -1 & 4 \\
\end{array}
\right)
\]
求$A^{2015}$
\begin{description}
\item[解] $|\lambda E-A|=(\lambda-1)^3$,然后由高等代数下P349例1可知,主对角元为$1$的若当块总数为$3-1=2$,$A$的若当标准型$J$为
\[
J = \left(
\begin{array}{ccc}
1 & & \\
 & 1 & 1\\
 &  & 1 \\
\end{array}
\right)
\]
由于$P^{-1}AP=J$,因此$AP=PJ$,设$P=(X_1,X_2,X_3)$,由$A(X_1,X_2,X_3)=(X_1,X_2,X_2+X_3)$,解$(A-I)Y=0$,得
\end{description}

\item $A,B$无公共特征根,$f(\lambda)$为$A$的特征多项式。证明
\begin{description}
\item[(1)] $f(B)$可逆
\item[(2)] $AX=XB$只有零解
\item[证(1)] 设$\lambda_1,\cdots,\lambda_n$为$A$的全部复特征值,则
\[
f(x) = \prod_{i=1}^n(x-\lambda_i) \Rightarrow f(B) = \prod_{i=1}^n(B-\lambda_iE)
\]
又因为$\lambda_i$不是$B$的特征值
\[
|f(B)| = \prod_{i=1}^n|B-\lambda_iE| \neq 0
\]
\item[证(2)]
\[
A^2X = A(AX)=(AX)B = (XB)B = XB^2
\]
如此类推
\[
A^nX=XB^n \rightarrow 0 = f(A)X = Xf(B)
\]
因为$f(B)$可逆,所以$X=0$
\end{description}

\item $V$上两个线性变换$f,g$满足$\ker f \subset \ker g$。求证
\begin{description}
\item[(1)] 存在线性变换$h$,使得$g=hf$
\item[(2)] 若$\ker f=\ker g$,则存在可逆线性变换$h$,使得$g=hf$
\item[证(1)] $f,g$在某基下的矩阵为$A,B$,由$\ker f \subset \ker g$得$Ax=0$的解是$Bx=0$的解,从而$Ax=0$与
\[
\begin{cases}
Ax=0 \\
Bx=0 
\end{cases}
\]同解故
\[
\rank(A) = \rank( \left(
\begin{array}{c}
A\\
B
\end{array}
\right) )
\]
从而$B$的行向量可由$A$的行向量线性表示,即存在$C$使得$B=CA$,则$g=hf$
\item[证(2)] 见高等代数下P266
\end{description}

\item $V$是$n$维复向量空间,$(\cdot , \cdot):V\times V\rightarrow \mathbf{C}$是反对称的非退化双线性型,$\phi:V\rightarrow V$是一个线性变换,满足
\[
(\phi(u),\phi(\nu))=(u,\nu),\qquad \forall u,\nu \in V
\]
求证
\begin{description}
\item[(1)] $\dim V$是偶数
\item[(2)] $\phi$是可逆的
\item[(3)] 若$\lambda$是$\phi$的特征值,则$\frac{1}{\lambda}$也是$\phi$的特征值
\item[证(2)] 由$\phi:V\rightarrow V$,可知是满射,只需证是单射。若$u\neq 0$,$(u,u)\neq 0$,$(\phi(u),\phi(u))\neq 0$,即$u\neq 0$时,$\phi(u)\neq 0$,即$\ker \phi={0}$,单射
\item[证(3)] 设$\phi$在基$\{e_i\}_{1\leq i\leq n}$下的表示矩阵为$B$,则$\phi(e_k)=\sum_{j=1}^nb_{jk}e_j \Rightarrow(\phi(e_k),\phi(e_m))=\sum_{j,l=1}^nb_{jk}(e_j,e_k)b_{lm}=(e_k,e_m)$,所以$B^TAB=A$。设$B\alpha=\lambda\alpha$,则由$\phi$是可逆的知$\lambda\neq 0$,又$\alpha \neq 0 \Rightarrow A\alpha \neq 0$,$B^TAB\alpha = A\alpha \Rightarrow B^TA\alpha = \frac{1}{\lambda}A\alpha$,所以$\frac{1}{\lambda}$是$B^T$的特征值
\end{description}
\end{enumerate}