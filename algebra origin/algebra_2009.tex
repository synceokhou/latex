\section{2009高代}
\begin{enumerate}
\item 计算行列式
\[
D_n =
\left[
\begin{array}{cccc}
x+y & x & \cdots & 0 \\
y & x+y & \cdots & \vdots \\
\vdots & \vdots & \ddots & x \\
0 & \cdots & y & x+y \\
\end{array}\right]
\]
\begin{description}
\item[解] $(x+y)D_{n-1}-xyD_{n-2}=D_n$
\[
\begin{cases}
D_n-xD_{n-1}=y(D_{n-1}-xD_{n-2})\\
D_n-yD_{n-1}=x(D_{n-1}-yD_{n-2})
\end{cases}
\]
又解得$D_1=x+y,D_2=x^2+xy+y^2$
\end{description}

\item 已知$A^*$,$B$满足$ABA^{-1}=BA^{-1}+3I$,证$B$可逆,并求$B^{-1}$
\begin{description}
\item[解] 通过$A^*$可以求$A,A^{-1}$,然后$AB=B+3A$,然后验证$A-I$可逆,再两边左乘$A^{-1}/3$
\end{description}

\item 已知$A$,求$A$的特征值,特征向量,$A^k$,$\lim_{l \to \infty}(-\frac{1}{4}A)^k$
\begin{description}
\item[解] 求出对角型,显然最后正负号交替只能收敛到零矩阵
\end{description}

\item 设$A,B$为实正定矩阵,求证$AB$正定的充要条件是$AB=BA$
\begin{description}
\item[证] $\Rightarrow$显然,详见高等代数上P348例10
\end{description}


\item 设$f$为$n$次多项式,$f'|f$的充要条件是$f$有$n$重根
\begin{description}
\item[证] $\Leftarrow$显然,详见高等代数下P45例8
\end{description}

\item 设$A$可逆,证$A+\alpha\beta'$可逆的充要条件是$1+\beta'A^{-1}\alpha \neq 0$,并求$A+\alpha\beta'$的逆
\begin{description}
\item[证]
\[
\left(
\begin{array}{cc}
A & \alpha \\
-\beta' & 1
\end{array}
\right)
\left(
\begin{array}{cc}
E & 0 \\
\beta' & 1
\end{array}
\right) = \left(
\begin{array}{cc}
A+\alpha\beta' & \alpha \\
0 & 1
\end{array}
\right)
\]
另一边
\[
\left(
\begin{array}{cc}
A & \alpha \\
-\beta' & 1
\end{array}
\right)
\left(
\begin{array}{cc}
E & -A^{-1}\alpha \\
0 & 1
\end{array}
\right) = \left(
\begin{array}{cc}
A & 0 \\
-\beta' & 1+\beta' A^{-1}\alpha
\end{array}
\right)
\]
这时可以上下两个秩相等,最后算逆矩阵
\[
\left(
\begin{array}{cc}
A & 0 \\
-\beta' & 1+\beta' A^{-1}\alpha
\end{array}
\right)
\left(
\begin{array}{cc}
A^{-1} & 0 \\
\frac{\beta'A^{-1}}{1+\beta' A^{-1}\alpha} & \frac{1}{1+\beta' A^{-1}\alpha}
\end{array}
\right) = \left(
\begin{array}{cc}
E & 0 \\
0 & 1
\end{array}
\right)
\]
\end{description}

\item 已知$\psi$为线性空间$V$上的线性变换,$\psi$有特征值为$1$,且特征值$1$的特征子空间是$n-1$维的,证明存在单位向量$\eta$使得$\psi:\alpha \rightarrow \alpha-(\alpha,\eta)\eta$
\begin{description}
\item[证] 见高等代数下P483例5,在正交变换的条件下,非常自然引出一维不变子空间
\end{description}


\item 设$A,B$为实矩阵,若$A,B$在复数域上相似则在实数域上也相似
\begin{description}
\item[证] 见高等代数上P268例13
\end{description}









\end{enumerate}