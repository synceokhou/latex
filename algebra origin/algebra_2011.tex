\section{2011高代}
\begin{enumerate}
\item 设$\frac{p}{q}$是既约分数,$f(x)=a_nx^n+a_{n-1}x^{n-1}+\dots+a_1x+a_0$是整系数多项式,而且$f(\frac{p}{q})=0$。证明
\begin{description}
\item[(1)] $p|a_0$而$q|a_n$
\item[(2)] 对任意整数$m$,有$(p-mq)|f(m)$
\item[证(1)] $f(x)=(qx-p)g(x)$
\end{description}

\item 设$n$阶方阵$A_n=(|i-j|)_{1\leq i,j \leq n}$,其行列式记为$D_n$。证明$D_n+4D_{n-1}+4D_{n-2}=0$,并求出$D_n$
\begin{description}
\item[证]
\[
D_n = \left[
\begin{array}{ccccc}
0 & 1 & 2 & \cdots & n-1 \\
1 & 0 & 1 & \cdots & n-2 \\
2 & 1 & 0 & \cdots & n-3 \\
\vdots & \vdots & \vdots & \ddots & \vdots \\
n-1 & n-2 & n-3 & \cdots & 0 \\
\end{array}\right]
\]
第三行加到第一行,第一行减去两倍第二行
\[
D_n = \left[
\begin{array}{ccccc}
0 & 2 & 0 & \cdots & 0 \\
1 & 0 & 1 & \cdots & n-2 \\
2 & 1 & 0 & \cdots & n-3 \\
\vdots & \vdots & \vdots & \ddots & \vdots \\
n-1 & n-2 & n-3 & \cdots & 0 \\
\end{array}\right] = -2 \left[
\begin{array}{cccc}
1 & 1 & \cdots & n-2 \\
2 & 0 & \cdots & n-3 \\
\vdots & \vdots & \ddots & \vdots \\
n-1 & n-3 & \cdots & 0 \\
\end{array}\right]
\]
再第二列加到第一列
\[
-2 \left[
\begin{array}{cccc}
2 & 1 & \cdots & n-2 \\
2 & 0 & \cdots & n-3 \\
\vdots & \vdots & \ddots & \vdots \\
2n-4 & n-3 & \cdots & 0 \\
\end{array}\right]
\]
分解成两个
\[
-4 \left[
\begin{array}{cccc}
0 & 1 & \cdots & n-2 \\
1 & 0 & \cdots & n-3 \\
\vdots & \vdots & \ddots & \vdots \\
n-2 & n-3 & \cdots & 0 \\
\end{array}\right]-2\left[
\begin{array}{cccc}
2 & 1 & \cdots & n-2 \\
0 & 0 & \cdots & n-3 \\
\vdots & \vdots & \ddots & \vdots \\
0 & n-3 & \cdots & 0 \\
\end{array}\right]
\]
\end{description}

\item 已知二阶矩阵
\[
A=\left(
\begin{array}{cc}
a & b \\
c & d \\
\end{array}\right)
\]
的特征多项式为$(\lambda-1)^2$,求$A^{2011}-2011A$
\begin{description}
\item[解] $f(\lambda) = (\lambda-1)^2,g(\lambda)=\lambda^{2011}-2011\lambda$ \\
$g(\lambda) = f(\lambda)p(\lambda) + k\lambda + b$\\
$g'(\lambda) = f'(\lambda)p(\lambda)+ f(\lambda)p'(\lambda)+ k$\\
\end{description}

\item 设$\alpha,\beta,\gamma$是$3$维线性空间$V$的一组基,线性变换$\mathcal{A}$满足
\[
\left\{
\begin{array}{l}
\mathcal{A}(\alpha+2\beta+\gamma) = \alpha \\
\mathcal{A}(3\beta+4\gamma) = \beta \\
\mathcal{A}(4\beta+5\gamma) = \gamma \\
\end{array}
\right.
\]
求$\mathcal{A}$在基$\alpha,2\beta+\gamma,\gamma$下的矩阵
\begin{description}
\item[解]
\[
\mathcal{A}^{-1}\left(
\begin{array}{c}
\alpha \\
\beta \\
\gamma \\
\end{array}\right) = A^{-1} \left(
\begin{array}{c}
\alpha \\
\beta \\
\gamma \\
\end{array}\right)
\]
用$B$表示两个基之间的过渡矩阵
\[
\left(
\begin{array}{c}
\alpha \\
2\beta+\gamma \\
\gamma \\
\end{array}\right) = B \left(
\begin{array}{c}
\alpha \\
\beta \\
\gamma \\
\end{array}\right)
\]
又把$\mathcal{A}(\alpha,\beta,\gamma)^T$看成一个整体,这个转置不转内部向量
\[
\mathcal{A}\left(
\begin{array}{c}
\alpha \\
2\beta+\gamma \\
\gamma \\
\end{array}\right) = B\mathcal{A}\left(
\begin{array}{c}
\alpha \\
\beta \\
\gamma \\
\end{array}\right) = AB \left(
\begin{array}{c}
\alpha \\
\beta \\
\gamma \\
\end{array}\right)
\]
\end{description}


\item 已知
\[
A = \left(
\begin{array}{ccc}
2 & 2 & -2 \\
2 & 5 & -4 \\
-2 & -4 & 5 \\
\end{array}\right)
\]
\begin{description}
\item[(1)] 求$A$的特征多项式,并确定其是否有重根
\item[(2)] 求正交矩阵$P$,使得$PAP^{-1}$为对角矩阵
\item[(3)] 令$V$是所有与$A$可交换的实矩阵全体,证明$V$是一个实数域上的线性空间,并确定$V$的维数
\item[解(1)] $f(x)=(x-1)^2(x-10)$
\item[解(3)] $A=PDP^{-1}$,又$AB=BA$,即$PDP^{-1}PCP^{-1}=PCP^{-1}PDP^{-1}$,即$CD=DC$,然后$B=PCP^{-1}$
\end{description}

\item 设$A,B$是两个$n$阶复方阵,$n>1$
\begin{description}
\item[(1)] 如果$AB=BA$,证明$A,B$有公共的特征向量
\item[(2)] 如果$AB-BA=\mu B$,其中$\mu$是非零复数,那么$A,B$是否会有公共的特征向量
\item[证(1)] 对于$A$的特征值$\lambda$有特征子空间$V_{\lambda}$,由于$AB=BA$,所以$V_{\lambda}$是$B$的不变子空间,$B|V_{\lambda}$有特征值$\mu$,而$\mu$有特征子空间$V_{\mu}$,故可取$\xi\in V_{\mu}$
\item[证(2)] 归纳法证$AB^k-B^kA=k\mu B^k$,$\tr(k\mu B^k)=\tr(AB^k-B^kA)=0$,故$B$为幂零矩阵。取$\xi \in \ker B$,有$AB\xi-BA\xi=\mu B\xi$,即$A\xi \in \ker B$。此时在$\ker B$中有特征值,有特征向量,如上题证
\end{description}

\item 设$A$是$n$阶实方阵,其特征多项式有如下分解
\[
p(\lambda) = \det(\lambda E-A)=(\lambda-\lambda_1)^{r_1}(\lambda-\lambda_2)^{r_2}\dots(\lambda-\lambda_s)^{r_s}
\]
其中$E$为$n$阶单位方阵,$\lambda_i$两两不等,证明$A$的若当标准型中以$\lambda_i$为特征值的若当块的个数等于特征子空间$V_{\lambda_i}$的维数
\begin{description}
\item[证] 设$A$的若当标准型为
\[
J = \left(
\begin{array}{cc}
J_{\lambda_i} & \\
 & B \\
\end{array}\right)
\]
其中$\lambda_i$不是$B$的特征值
\[
J_{\lambda_i} = \left(
\begin{array}{ccc}
J_1 & & \\
 & \ddots & \\
 & & J_t\\
\end{array}\right),J_i = \left(
\begin{array}{cccc}
\lambda_i & 1 & & \\
 & \lambda_i & \ddots & \\
 & & \ddots & 1 \\
 & & & \lambda_i \\
\end{array}\right)
\]
\end{description}

\item 设$A$是$n$阶实方阵,证明$A$为实对称矩阵当且仅当$AA^T=A^2$
\begin{description}
\item[证] 若$\tr(AA^T)=0$,则$A$为零矩阵。证$\tr((A-A^T)(A-A^T)^T)=0$即可
\end{description}



























\end{enumerate}