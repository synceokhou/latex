\section{2007高代}
\begin{enumerate}
\item 设多项式$f(x),g(x),h(x)$只有非零常数公因子,证明存在多项式$u(x),v(x),w(x)$,使得$u(x)f(x)+v(x)g(x)+w(x)h(x)=1$
\begin{description}
\item[证] 记$k(x)=(f(x),g(x))$,有$u_1(x)f(x)+v_1(x)g(x)=k(x)$,假设$(k(x),h(x))\neq 1$,则与$f(x),g(x),h(x)$只有非零常数公因子矛盾,故$v_2(x)k(x)+w_2(x)h(x)=1$,然后代人即可
\end{description}

\item 设$m,n,p$都是非负整数,证明$(x^2+x+1)$整除$(x^{3m}+x^{3n+1}+x^{3p+2})$
\begin{description}
\item[证] $x^{3m+2}=(x^{3m+2}+x^{3m+1}+x^{3m})-(x^{3m+1}+x^{3m}+x^{3m-1})+x^{3m-1}$,同理凑另一个
\end{description}

\item 设$A$是$n$阶实数矩阵,$A\neq 0$,而且$A$的每个元素都和它的代数余子式相等。证明$A$是可逆矩阵
\begin{description}
\item[证] 设$\rank(A)=n-1$,但$\rank(A^*)=1$,又设$\rank(A)<n-1$,但$\rank(A^*)=0$,显然矛盾,因为$A=(A^*)^T$
\end{description}


\item 计算$n$阶行列式
\[
D_n =
\left[
\begin{array}{cccc}
2cos\alpha & 1 & & \\
1 & 2cos\alpha & \ddots & \\
 & \ddots & \ddots & 1 \\
 & & 1 & 2cos\alpha \\
\end{array} \right]
\]
\begin{description}
\item[解] $2\cos\alpha D_{n-1}-D_{n-2}=D_n$
\[
\begin{cases}
D_n - bD_{n-1}=k(D_{n-1}-bD_{n-2})\\
D_n - kD_{n-1}=b(D_{n-1}-kD_{n-2})
\end{cases}
\]
然后解得$D_2=4\cos^2\alpha-1,D_1=1,b=\cos\alpha+i\sin\alpha,k=\cos\alpha-i\sin\alpha$
\end{description}

\item 设$\alpha_1,\alpha_2,\dots,\alpha_k,\in \mathbf{R}^n$是齐次线性方程组$AX=0$的基础解系,$s,t\in \mathbf{R},\beta_1=s\alpha_1+t\alpha_2,\dots,\beta_{k-1}=s\alpha_{k-1}+t\alpha_k,\beta_k=s\alpha_k+t\alpha_1$。$s,t$应满足什么关系,使得$\beta_1,\dots,\beta_{k-1},\beta_k$是方程组$AX=0$的基础解系。反之,当$\beta_1,\dots,\beta_{k-1},\beta_k$是方程组$AX=0$的基础解系时,这个关系必须成立
\begin{description}
\item[证] $\alpha_1,\alpha_2,\dots,\alpha_k$是解空间$V$的一组基,设$\beta=T\alpha$,然后当且仅当$T$是过渡矩阵,满秩时$\beta_1,\dots,\beta_{k-1},\beta_k$是解空间$V$的一组基。反之亦然
\end{description}

\item 设$A$是实对称矩阵,如果$A$是半正定的,则存在实的半正定矩阵$B$,使得$A=B^2$
\begin{description}
\item[证] $P^TAP=D$,然后$D$可以开根号变$D_1$,然后有$A=PD_1P^TPD_1P^T$,详见高等代数上P346例6
\end{description}

\item 已知
\[
A = \left(
\begin{array}{ccc}
1 & 0 & 0 \\
1 & 0 & 1 \\
0 & 1 & 0 \\
\end{array}\right)
\]
证明对于$n\geq 3$有$A^n = A^{n-2} + A^2 - I$,并计算$A^{100}$
\begin{description}
\item[证] $(A^{n-2}-I)(A^2-I)$,然后又$|\lambda E-A|=(\lambda-1)^2(\lambda+1)$,这个特征多项式就是最小多项式,最后就是在$A^{n-2}-I$找$\lambda-1$因子
\end{description}

\item 设二次型$f=x_1^2 + x_2^2 + x_3^2 + 2ax_1x_2 + 2x_1x_3+4bx_2x_3$通过正交变换化为标准型$f=y_2^2+2y_3^2$,求参数$a,b$及正交变换
\begin{description}
\item[证] 把特征值代进$|\lambda E-A|$发现$a=2b=0$
\end{description}

\item 设$A$是复数域上$6$维线性空间$V$的线性变换,$A$的特征多项式为$(\lambda-1)^3(\lambda+1)^2(\lambda+2)$,证明$V$能过分解成三个不变子空间的直和,而且它们的维数分别是$1,2,3$
\begin{description}
\item[证] 见高等代数下P285定理1
\end{description}















\end{enumerate}