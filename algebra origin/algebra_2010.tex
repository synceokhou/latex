\section{2010高代}
\begin{enumerate}
\item 设$A,B$分别是$n\times m$和$m\times n$矩阵
\begin{description}
\item[(1)] 求证$|I_n-AB|=|I_m-BA|$
\item[(2)] 计算行列式
\[
D_n = \left[
\begin{array}{cccc}
1+a_1+x_1 & a_1+x_2 & \cdots & a_1+x_n\\
a_2+x_1 & 1+a_2+x_2 & \cdots & a_2+x_n \\
\vdots & \vdots & \vdots & \vdots \\
a_n+x_1 & a_n+x_2 & \cdots & 1+a_n+x_n \\
\end{array}\right]
\]
\item[证(1)] 见高等代数上P198命题2
\item[证(2)]
\[
D_n = \left[I+\left(
\begin{array}{cc}
a_1 & 1\\
a_2 & 1\\
\vdots & \vdots\\
a_n & 1\\
\end{array}\right)\left(
\begin{array}{cccc}
x_1 & x_2 & \cdots & x_n \\
1 & 1 & \cdots & 1 \\
\end{array}\right)
\right]
\]
\end{description}

\item 已知3阶正交矩阵$A$的行列式为$1$,求证$A$的特征多项式一定为$f(\lambda)=\lambda^3-a\lambda^2+a\lambda-1$,其中$a\in \mathbf{R}$且$-1 \leq a\leq 3$
\begin{description}
\item[证] 在高等代数上P302的3
\[
|I-A|=|A'||I-A|=|-(I-A)|
\]
3阶,所以1为一个特征值
\[
T^{-1}AT=\left(
\begin{array}{cc}
1 & \alpha\\
0 & A_2\\
\end{array}
\right) = D_2
\]
然后证$\alpha=0$和$A_2$为正交阵
\[
D_2D_2^T=\left(
\begin{array}{cc}
1+\alpha\alpha^T & \alpha A_2^T\\
A_2\alpha^T & A_2A_2^T\\
\end{array}
\right)
\]
然后再找$A_2$长啥样,最后可以求原式
\end{description}


\item 设$A,B$是$n$阶方阵,$A$可逆,$B$幂零,$AB=BA$
\begin{description}
\item[(1)] 求证$A+B$可逆
\item[(2)] 举例说明使上结论成立,$AB=BA$不可缺
\item[证(1)] 高等代数下P290例5,然后就两个矩阵特征值相加,注意这两个上三角阵的对角线都是特征值
\end{description}

\item 任意$n$阶实方阵$A$的特征向量也是其伴随矩阵$A^*$的特征向量
\begin{description}
\item[证] 分类讨论,对于$\rank(A)=n,\rank(A)<n-1$的情况很简单,下面看$\rank(A)=n-1,\rank(A^*)=1$,设
\[
A^*=\left(
\begin{array}{cccc}
a & b_2a & \cdots & b_na \\
c_2a & b_2c_2a & \cdots & b_nc_2a \\
\vdots & \vdots &\ddots & \vdots \\
c_na & b_2c_na & \cdots & b_nc_na \\
\end{array}
\right)=a\left(
\begin{array}{c}
c_1\\
c_2\\
\vdots\\
c_n
\end{array}
\right)\left(
\begin{array}{cccc}
b_1 & b_2 & \cdots & b_n \\
\end{array}
\right)=aCB^T
\]
然后$A^*$的特征值为$\lambda_1=a\sum b_ic_i$,现在取$A$的特征向量$\xi$,对应的特征值为$\lambda$,有$\lambda A^*\xi = A^*\lambda\xi = A^*(A\xi)=0\xi$,如果$\lambda=0,A^*\xi\neq =0$(另外两种情况自己考虑),则$A\xi=0\xi$,由$AA^*=0$,则$C$为$A$属于特征值$0$的特征向量,因为$\rank(A)=n-1$,$A$特征值$0$是单根,故$\xi=kC,k\neq 0$,所以$A^*\xi=kA^*C=ak\sum b_ic_iC$
\end{description}

\item $n$阶方阵$A$能表示成$A=H+K$,其中$H=\bar{H}^T,K=-\bar{K}^T$,设$a,h,k$分别是$A,H,K$中元素的最大模,若$z=x+iy(x,y\in \mathbf{R})$是$A$的任意特征值,求证
\begin{description}
\item[(1)] $|z|\leq na,|x|\leq nh,|y|\leq nk$
\item[(2)] Hermite的矩阵的特征值都是实数
\item[(3)] 反对称矩阵的非零特征值都是纯虚数
\item[证(1)] $A\xi=z\xi$其中$\xi=(\epsilon_1,\epsilon_2,\cdots,\epsilon_n)^T,\epsilon_{max}=|\epsilon_k|=\max\{|\epsilon_1|,|\epsilon_2|,\cdots,|\epsilon_n|\}$,则$|z\epsilon_k|=|z||\epsilon_k|=|a_{k1}\epsilon_1+a_{k2}\epsilon_2+\cdots+a_{kn}\epsilon_n|\leq n|\epsilon_k|a$,然后通过$A\xi=H\xi+K\xi=x\xi+iy\xi$和$\bar{K}^T=-K,\bar{H}^T=H$得到$\bar{\xi}^T(H-xI_n)\xi=\bar{\xi}^T(-K+iyI_n)\xi=0$
\end{description}

\item 设$\mathcal{A}$是$n$维实线性空间$V$的线性变换,求证$\mathcal{A}$至少有一个维数为$1$或$2$的不变子空间
\begin{description}
\item[证] 在实数域上有特征值,那就有维数为$1$的不变子空间。若无,转到复数域,必有两互补的特征值,求出特征值向量再生成维数为$2$的不变子空间
\end{description}

\item 设循环矩阵$C$为
\[
\left(
\begin{array}{cccc}
c_0 & c_1 & \cdots & c_{n-1} \\
c_{n-1} & c_0 & \cdots & c_{n-2} \\
\vdots & \vdots & \ddots & \vdots\\
c_1 & c_2 & \cdots & c_0 \\
\end{array} \right)
\]
\begin{description}
\item[(1)] 求$C$的所有特征值以及相应的特征向量
\item[(2)] 求$|C|$
\item[证(1)] 见高等代数上P279例13
\end{description}

\item 设$M_n(\mathbf{C})$是复数域上所有$n$阶方阵组成的线性空间,$\tau:M_n(\mathbf{C})\rightarrow \mathbf{C}$是线性映射,满足$\tau(AB)=\tau(BA)$,求证$\forall A \in M_n(\mathbf{C}),\exists \lambda \in \mathbf{C}$使得$\tau(A) = \lambda\cdot\tr(A)$
\begin{description}
\item[证] $T(AB-BA)=0$,故有$T(E_{ij}E_{ji}-E_{ji}E_{ij})=0=T(E_{ii}-E_{jj})$,又有$T(E_{ij})=T(E_{ii}E_{ij}-E_{ij}E_{ii})=0$
\end{description}











\end{enumerate}