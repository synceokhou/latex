\begin{itemize}
\item 已知
\[
A=\left(
\begin{array}{ccc}
1 & 0 & a\\
0 & 1 & b \\
c & d &1 
\end{array}
\right)
\]
\begin{description}
\item[(1)] 求$\det(A)$
\item[(2)] 求$\tr(A)$
\item[(3)] 证明$\rank(A)\geq 2$
\item[(4)] 为使$\rank(A)=2$,求出$a,b,c,d$应满足的条件
\item[证(3)] 因为存在
\[
\left[
\begin{array}{cc}
1 & 0 \\
0 & 1 \\
\end{array}
\right]
\]
$\rank(A)\geq 2$
\item[解(4)] 由于(3),只需$\det(A)\neq 0$
\end{description}

\item 设$A$是欧氏空间$\mathbf{R}^n$的一个变换,证若$A$保内积,那么一定线性且正交
\begin{description}
\item[证] 此类证明就是把要等于0的东西丢到内积里面。比如证线性。$A(\alpha+\beta)=A(\alpha)+A(\beta)$
\begin{eqnarray*}
& &(A(\alpha+\beta)-A(\alpha)-A(\beta),A(\alpha+\beta)-A(\alpha)-A(\beta))\\
& = & (\alpha+\beta,\alpha+\beta)-2(\alpha+\beta,\alpha)-2(\alpha+\beta,\beta)+(\alpha,\alpha)+(\beta,\beta)+2(\alpha,\beta)\\
& = & 0
\end{eqnarray*}
同理可证$A(k\alpha)=kA(\alpha)$。接下来证正交,只需要证单射,就能得到满射,根据高等代数下10.4定义1可得。证单射,当$\alpha\neq 0$,$(\alpha,\alpha)\neq 0\Rightarrow(A\alpha,A\alpha)\neq 0\Rightarrow A\alpha\neq 0\Rightarrow\ker A=0$
\end{description}

\item 设$A$是$2003$阶实方阵,且$A^r=0$,$r$是自然数,问$\rank(A)$的最大值是多少
\begin{description}
\item[解] 和2006年的一边很像。请使用Sylvester秩不等式,见高等代数上4.5例2
\end{description}

\item 给定$\mathbf{R}$上线性空间$V$的子空间$W_1,W_2$,证明$\dim(W_1\cap W_2)\geq\dim(W_1)+\dim(W_2)-\dim(V)$
\begin{description}
\item[证] 维数公式$\dim(W_1\cap W_2)+\dim(W_1+ W_2) = \dim(W_1)+\dim(W_2)$,显然$W_1+ W_2\subseteq V$
\end{description}

\item 给定$n$个不同的数$a_1,a_2,\cdots,a_n$,试求一个小于等于$n-1$次的多项式$f(x)$,使得$f(a_i)=b_i$,其中$b_i(i=1,\cdots,n)$也是给定的值
\begin{description}
\item[证] $f(x)=c_nx^{n-1}+\cdots+c_1$,然后带进去解范德蒙
\end{description}

\item 给定$\mathbf{R}$上二维线性空间$V$的线性变换$A$,$A$在一组基下的矩阵表示为
\[
A=\left(
\begin{array}{cc}
0 & 1 \\
1-a & 0
\end{array}
\right)
\]
$a\neq 0$,求$A$的不变子空间
\begin{description}
\item[证] $a=1$,$A$为若当块,不变子空间为$V,\{0\}$。$a\neq 1$,自己算
\end{description}

\item 若$Q$为$n$阶对称正定方阵,$x$为$n$维实向量,证$0\leq x^T(Q+xx^T)^{-1}x<1$
\begin{description}
\item[证] 令$A=(Q+xx^T)^{-1}$
\begin{eqnarray*}
x^TQ^{-1}x&=&x^TA(Q+xx^T)Q^{-1}x\\
&=& x^TA(E+xx^TQ^{-1})x\\
&=& x^TAx+x^TAxx^TQ^{-1}x\\
&=& x^TAx(1+x^TQ^{-1}x)\\
\end{eqnarray*}
从而
\[
xA^{-1}x=\frac{x^TQ^{-1}x}{1+x^TQ^{-1}x}
\]
当$Q$为$n$阶对称正定,$Q^{-1}$为$n$阶对称正定,从而$x^TQ^{-1}x\geq 0$
\end{description}






















\end{itemize}