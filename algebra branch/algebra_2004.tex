\begin{itemize}
\item 已知
\[
\begin{cases}
x_{n+1}=x_n+4y_n\\
y_{n+1}=2x_n+y_n
\end{cases}
\]
且$x_0=1,y_0=0$,求$x_{100},y_{100}$
\begin{description}
\item[解]
\[
\left(
\begin{array}{c}
x_{n+1}\\
y_{n+1}\\
\end{array}
\right) = \left(
\begin{array}{cc}
x_n & 4y_n\\
2x_n & y_n\\
\end{array}
\right)
\left(
\begin{array}{c}
x_{n}\\
y_{n}\\
\end{array}
\right)
\]
所以剩下的问题就是变成对角型
\end{description}

\item 设$A,B$为同阶对称正定阵,若$A>B$(即$A-B$为正定阵),试问是否一定有$A^2>B^2$?为什么?
\begin{description}
\item[证] 因为$A,B$为同阶对称正定阵,故存在正交阵$P$,使得$A,B$为对角型,然后剩下的你都懂了。注意左边的解法是错的,原因是$B$的特征向量不一定是$A$的特征向量。其实可以设$B$是对角阵
\[
A = \left(
\begin{array}{cc}
a_1 & a_2\\
a_2 & a_4\\
\end{array}
\right),B = \left(
\begin{array}{cc}
b_1 & 0\\
0 & b_4\\
\end{array}
\right)
\]
$|A-B|=(a_1-b_1)(a_4-b_4)-a_2^2$,$|A^2-B^2|=(a_1^2+a_2^2-b_1^2)(a_2^2+a_4^2-b_4^2)-(a_2^2+a_1a_4)^2$,太多变量了,可以随便弄矩阵满足$A^2-B^2$半负定
\end{description}

\item 证明若$S$为$n$阶对称正定阵,则
\begin{description}
\item[(1)] 存在唯一的对称正定矩阵$S_1$,使得$S=S_1^2$
\item[(2)] 若$A$是$n$阶实对称矩阵,则$AS$的特征值是实数
\item[证(1)] 存在性就是开根号,唯一性见高等代数上6.3例7
\item[证(2)] 因为$A$是$n$阶实对称矩阵,$AS$还是对称阵
\end{description}

\item 设$A=(a_{ij})$是$2004$阶方阵,且$a_{ij}=ij,1\leq i,j\leq 2004$,$I$是$2004$阶单位阵,计算$f(x)=\det(I+Ax),x\in \mathbf{R}$
\begin{description}
\item[解] 猥琐解法,分类$x=0$,$f(0)=1$,$x\neq 0$,$f(x)=x^n\det(\frac{1}{x}I+A)$,这里就是$A$的特征多项式
\end{description}

\item 令$f(x,y)=2x^2-7xy+y^2$。求$f(x,y)$在$\mathbf{R}^2$中单位圆上的极大值与极小值及极值点
\begin{description}
\item[解] 单位圆,$(x,y)$考虑单位向量。也就是单位特征向量,以及最大最小特征值
\end{description}

\item 设$A,B$是$n$阶实方阵,而$I$是$n$阶单位阵。证若$I-AB$可逆,则$I-BA$也可逆
\begin{description}
\item[解] 就是行列式值,打洞
\end{description}

\item 设$A$为$n \times n$阶实对称矩阵,$b$为$n\times 1$维实向量,证明$A-bb^T>0$的充分必要条件是$A>0$及$b^TA^{-1}b<1$
\begin{description}
\item[解] 就又是打洞
\end{description}

\item 设$V$是$n$维向量空间,$f,g$是$V$上的线性变换,且$f$有$n$个互异的特征根。证$fg=gf$的充要条件是$g$是$I,f,f^2,\cdots,f^{n-1}$的线性组合
\begin{description}
\item[解] $\Rightarrow$显然,下面证$\Leftarrow$。在一组基下,$A$为有$n$个互异特征根的对角矩阵,由高等代数上4.2例1可知,$B$也为对角矩阵。显然$\dim(B)=n$,故只需证$f^i$线性无关
\end{description}



















\end{itemize}