\begin{itemize}

\item 设$A,B$为满秩方阵,试求
\[
Q=\left(
\begin{array}{cc}
A & C\\
O & B \\
\end{array}
\right)
\]
的逆矩阵
\begin{description}
\item[解] 先打洞,然后乘各自逆矩阵
\end{description}

\item 设$a_1,a_2,\cdots,a_n$为$n$个实数,求
\[
A=\left(
\begin{array}{cccc}
a_1 & a_1 & \cdots & a_1 \\
a_2 & a_2 & \cdots & a_2 \\
\vdots & \vdots & \ddots & \vdots\\
a_n & a_n & \cdots & a_n \\
\end{array}
\right)
\]
的所有特征值
\begin{description}
\item[解] 首先$\rank(A)=1$,故有$n-1$个特征值为0,另外$\tr(A)=\sum a_i$,所以最后一个特征值是$\tr(A)=\sum a_i$
\end{description}

\item 设$a,b,c,d$为正实数,求出满足$y \geq ax+b$与$y\geq -cx+d$之$y$的最小值
\begin{description}
\item[解] 画图然后解
\end{description}

\item 设$A,B$为方阵,且$B$为满秩阵,$s$为实数,$C=A+sB$,证$\exists a>0$,使得在$0<s<a$时,$C$满秩
\begin{description}
\item[解] $|C|=|AB^{-1}+sE|$其实是一个关于$s$的$n$次多项式,不妨设$a$是距离0最近的一个根,显然$|C|$在区间$(0,a)$不等于零
\end{description}

\item 设$\alpha_i=(a_{i1},a_{i2},\cdots,a_{in})^T,i=1,2,\cdots,m(\leq n)$为$n$维欧氏空间中的$m$个向量,有设$P=(p_{ij})_{1\leq i,j\leq m}$,其中$p_{ij}=\sum_{k=1}^n a_{ik}a_{jk}$,证$\alpha_1,\alpha_2,\cdots,\alpha_m$为线性无关的,当且仅当$P$为满秩
\begin{description}
\item[解] $A$线性无关$\Leftrightarrow A^TA$满秩,高等代数上4.3例3有结论$\rank(A'A)=\rank(A)$,$AX=0$的解是$A'AX=0$的解,$A'AX=0$的解是$X'A'AX=0$的解,而$X'A'AX=0$可知$AX=0$因为这个都是平方
\end{description}

\item 设$A,B$为对称方阵,证$\tr(ABAB)\leq \tr(AABB)$
\begin{description}
\item[解] 非常神奇的一个trick
\[
A=(\alpha_1,\cdots,\alpha_n) = A^T = \left(\begin{array}{c}
\alpha_1\\
\vdots\\
\alpha_n\\
\end{array}
\right)
\]
同理可得$B$,然后$ABAB=A^TBA^TB$,$AABB=A^TAB^TB$
\end{description}





















\end{itemize}