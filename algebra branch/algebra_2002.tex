\begin{itemize}
\item $A=J_r(0)_{n \times n}$,求$A^{n-1}$
\begin{description}
\item[解] 最右上角为1
\end{description}

\item 设$A,I-A,I-A^{-1}$均为可逆方阵,证$(I-A)^{-1}+(I-A^{-1})^{-1}=I$
\begin{description}
\item[解] 两边乘上$I-A$和$I-A^{-1}$
\end{description}

\item 求$\lim_{k \to +\infty}A^k$
\[
A=\left(
\begin{array}{ccc}
\frac{1}{2} & 4 & 2 \\
0 & \frac{1}{3} & 4 \\
0 & 0 & \frac{1}{5} \\
\end{array}
\right)
\]
\begin{description}
\item[解] 特征值就在对角线上,解出特征向量,施密斯正交化
\end{description}

\item 已知$A=(a_{ij}),B=(b_{ij})$均为$2\times 2$对称正定矩阵,定义$A*B=(a_{ij}b_{ij})$,证$A*B$也是$2\times 2$对称正定矩阵
\begin{description}
\item[解] 对于$n$的情况可见高等代数上补充题六10
\end{description}

\item 设$A=J_r(\lambda)_{n \times n}$为实方阵,而$V$是所有与$A$可交换的$n \times n$实方阵全体,即$V=\{B|BA=AB\}$,证
\begin{description}
\item[(1)] $V$是线性空间
\item[(2)] $\dim(V)=n$
\item[证(2)] 生生代矩阵进去,最后解得只有$n$个数
\end{description}

\item 设$\sum_{i=0}^{n-1}x^iP_i(x^{in})=P(x^n)$,且$(x-1)|P(x)$,其中$P_i(0\leq i\leq n-1),P$均为实系数多项式,证
\begin{description}
\item[(1)] $P_i(x)=0(1\leq i\leq n-1)$
\item[(2)] $P(x)=0$
\item[(3)] $P_0(1)=0$
\item[证] 假设$P_i$中最大的幂次是$m$,所以$j$可取范围是0到$m$,左边的$x$幂次只能是$i(jn+1)$,只要$i$和$j$不同,幂次不同,不可能出现$(x-1)$因子,所以所有幂次大于0的系数只能为0,右边所有幂次大于0的系数显然为零,最后一个当$x=1$时为0,代入也只能为0
\end{description}

































\end{itemize}

