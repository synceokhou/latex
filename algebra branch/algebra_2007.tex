\begin{itemize}
\item 设$A$是方阵,$A^2+A-3I=0$
\begin{description}
\item[(1)] 证明$A-2I$可逆
\item[(2)] 求满足$AX+3(A-2I)^{-1}A=5X+8I$的方阵$X$
\item[证(1)] 由
\[
(A-2I)\frac{A+3I}{-3}=I
\]
\item[证(2)] 同上$A-5I$可逆
\[
(A-5I)X=(A+3I)A+8I
\]
\end{description}

\item 已知方程组
\[
\begin{cases}
x_1 + x_2+ax_3+x_4=1\\
-x_1 + x_2-x_3+bx_4=2\\
2x_1 + x_2+x_3+x_4=c
\end{cases}
\begin{cases}
x_1 +x_4=-1\\
x_2 - 2x_4=d\\
x_3+x_4=e
\end{cases}
\]
同解,求$a,b,c,d,e$
\begin{description}
\item[解] 非齐次方程组的解是对应齐次方程组的解空间的商空间,所以商空间相同,则证明特解相同,并且对应齐次方程组的解空间也相同。于是分两步,拿出两个齐次方程组,把系数矩阵弄相等可求$a,b$,这一步还要记住拓展矩阵的变化。第二步拓展矩阵也相同可求出$c,d,e$
\end{description}

\item $A$为$n$阶矩阵,且$A^k=0$,证明
\begin{description}
\item[(1)] 矩阵
\[
I+A+\frac{A^2}{2!}+\cdots+\frac{A^{k-1}}{(k-1)!}
\]可逆
\item[(2)] 矩阵$I+A$与
\[
I+A+\frac{A^2}{2!}+\cdots+\frac{A^{k-1}}{(k-1)!}
\]
相似
\item[证(1)] $A$是幂零矩阵,$A$的零化多项式$x^k=0$包含了特征多项式所有的根,也就是说幂零矩阵的特征值都为0,此时$\exists P,P^{-1}AP=J(0)$,或者多个若当块。然后给那个矩阵左乘右乘
\item[证(2)] 我们可以按(1)把它变成上三角之后,用行列式因子吗
\end{description}

\item 设$A,B:V \to V$为线性变换,$A$可逆,$B$幂零($\exists k \in \mathbf{N}^+$使得$B^k=0$),且$AB=BA$。证明$A-B$的核空间$\ker(A-B)$等于零空间
\begin{description}
\item[证] $A,B$可以用同一个$P$上三角化,于是在$V$中可以找到一组基使得$A,B$的矩阵都是上三角
\item[另证] $\forall x\in \ker(A-B),Ax=Bx$,两边用$B^{k-1}$作用$B^{k-1}Ax=AB^{k}x=0$,显然$B^{k-1}$和$A$都不为零,只能$x=0$
\end{description}

\item 对常微分系统
\[
\frac{\ud Z}{\ud t}=J^{-1}\nabla H(Z),
J=\left(
\begin{array}{cc}
O_n & I_n \\
-I_n & O_n \\
\end{array}
\right),
Z=(z_1,\cdots,z_{2n})^T\in \mathbf{R}^{2n}
\]
其中$\nabla H =(\frac{\partial H}{\partial z_1},\cdots,\frac{\partial H}{\partial z_{2n}})^T$,证明
\begin{description}
\item[(1)] $J^{-1}=J^T=-J$
\item[(2)] 系统的相流$\{g^t:\mathbf{R}^{2n} \to \mathbf{R}^{2n}|t \in \mathbf{R}\}$总是满足
\[
\left[\frac{\partial g^t(Z)}{\partial Z}\right]^T J \left[\frac{\partial g^t(Z)}{\partial Z}\right] = J
\]
系统的相流是指任意给定一个$Z$,$g^t(Z)$是系统的解
\[
\frac{\ud g^t(Z)}{\ud t}=J^{-1}\nabla H(g^t(Z))
\]
\item[(3)] 证明更一般的结论,若$J^{-1}$变为$K^{-1}(Z)$,其中$K(Z)=(k_{ab})\in \mathbf{R^{2n\times 2n}}$反对称,可逆且满足恒等式$\frac{\partial k_{ab}}{\partial z_c}+\frac{\partial k_{bc}}{\partial z_a}+\frac{\partial k_{ca}}{\partial z_b}=0,1\leq a,b,c\leq 2n$,则系统的相流满足
\[
\left[\frac{\partial g^t(Z)}{\partial Z}\right]^T K(g^t(Z)) \left[\frac{\partial g^t(Z)}{\partial Z}\right] = K(Z)
\]
这更一般的情况下系统的相流是指任意给定一个$Z$,$g^t(Z)$是系统的解
\[
\frac{\ud g^t(Z)}{\ud t}=K^{-1}(Z)\nabla H(g^t(Z))
\]
\end{description}

\item 设$A$是$n \times n$正定实对称矩阵,$y\in \mathbf{R}^n$,且$y\neq 0$,证明
\[
\lim_{m \to \infty}\frac{y^TA^{m+1}y}{y^TA^my}
\]
存在且等于$A$的一个特征值
\begin{description}
\item[证] $\exists P,A=P^{-1}BP$,其中
\[
A=\left(
\begin{array}{cccc}
\lambda_1 & 0 & \cdots & 0 \\
0 & \lambda_2 & \cdots & 0 \\
\vdots & \vdots & \ddots & \vdots \\
0 & 0 & \cdots & \lambda_n\\
\end{array}
\right)
\]
且$0<\lambda_1\leq \lambda_2 \leq \cdots \leq \lambda_n$为$A$的特征值
\begin{eqnarray*}
\lim_{m \to \infty}\frac{y^TA^{m+1}y}{y^TA^my} & =& \lim_{m \to \infty}\frac{y^TB^{m+1}y}{y^TA^my}\\
& =& \lim_{m \to \infty}\frac{\sum_{k=1}^n \lambda_k^{m+1}y_k^2}{\sum_{k=1}^n \lambda_k^{m}y_k^2}\\
& =& \lambda_j \lim_{m \to \infty}\frac{\sum_{k=1}^{j-1} (\frac{\lambda_k}{\lambda_j})^{m+1}y_k^2+y_j^2}{(\frac{\lambda_k}{\lambda_j})^my_k^2+y_j^2}\\
\end{eqnarray*}
其中$j=\max\{k|y_k\neq 0\}$
\end{description}

\item 设
\[
A=\left(
\begin{array}{cc}
0 & -1 \\
1 & 0\\
\end{array}
\right)
\]
为线性变换$\mathcal{A}$在一组基下的矩阵表示
\begin{description}
\item[(1)] 求$\mathcal{A}$的不变子空间
\item[(2)] 设$\mathbf{R}^n$上的线性变换$\mathcal{A}$在基$\alpha_1,\cdots,\alpha_n$下的方阵表示为对角型,且对角线上元素互异,问$\mathcal{A}$的所有不变子空间共有多少个
\item[解(1)]
\[
\begin{cases}
A\alpha_1 = \alpha_1 +\alpha_2\\
A\alpha_2 = -\alpha_1
\end{cases}
\]
若子空间维数为1,则只能是$\{\alpha_1\}$或者$\{\alpha_2\}$,显然不可能
\end{description}


\item 令$A$与$B$是两个$n \times n$实矩阵,假定存在一$n \times n$可逆复矩阵$U$使得$A=UBU^{-1}$,证明
存在一$n \times n$可逆实矩阵$V$使得$A=VBV^{-1}$(即证两个实矩阵在$\mathbf{C}$上相似,则它们在$\mathbf{R}$上相似)
\begin{description}
\item[证] 好面熟,找到$A(M+iN)=(M+iN)B$之后,再找$x\in \mathbf{R},|M+iN|\neq 0$
\end{description}

\item 证明
\begin{description}
\item[(1)] 如果$\sum_{i=1}^n\sum_{j=1}^n a_{ij}x_ix_j(a_{ij}=a_{ji})$是正定二次型,证
\[
f(y_1,y_2,\cdots,y_n)=\left[
\begin{array}{ccccc}
a_{11} & a_{12} & \cdots & a_{1n} & y_1\\
a_{21} & a_{22} & \cdots & a_{2n} & y_2\\
\vdots & \vdots & \ddots & \vdots & \vdots\\
a_{n1} & a_{n2} & \cdots & a_{nn} & y_n\\
y_1 & y_2 & \cdots & y_n & 0\\
\end{array}
\right]
\]
是负定二次型
\item[(2)] 证$f(y_1,y_2,\cdots,y_n)$的表示矩阵是$A$的负伴随矩阵$-A^*$
\item[(3)] 如果$A$是对称正定矩阵,那么其行列式满足$|A|\leq a_{11}a_{22}\cdots a_{nn}$
\item[(4)] 如果$A=(a_{ij})$仅仅是$n$阶实可逆方阵,那么其行列式$|A|$满足$|A|^2\leq \prod_{i=1}^n(a_{1i}^2+\cdots+a_{ni}^2)$
\item[证(1)] 打洞
\[
\left[
\begin{array}{cc}
A & Y\\
Y^T & 0\\
\end{array}
\right] \Rightarrow \left[
\begin{array}{cc}
A & 0\\
0 & Y^T(-A^{-1}Y)\\
\end{array}
\right]
\]
所以重点在$A^{-1}$,若$P^{-1}AP=\diag\{\lambda_1,\cdots,\lambda_n\}$,两边移位有$\diag\{\lambda_1^{-1},\cdots,\lambda_n^{-1}\}=PA^{-1}P^{-1}$,然后令$PY=Z$的代换,就能变成负定二次型
\item[证(2)] (1)的代换之前把$|A|$丢进中间
\item[证(3)] 面熟,归纳法把行向量史密斯正交化
\item[证(4)] 令$B=AA^T$,这时$B$是对称正定阵,能用(3)
\end{description}




























\end{itemize}