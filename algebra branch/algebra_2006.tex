\begin{itemize}
\item 已知$\alpha,\beta,\gamma$,求
\[
A= \left(
\begin{array}{cccc}
 \alpha & \beta & & \\
\gamma & \alpha & \ddots & \\
 & \ddots & \ddots & \beta \\
 & & \gamma & \alpha
\end{array}
\right) \in \mathbf{R}^{n \times n}
\]
的行列式的值
\begin{description}
\item[解] 2014年考了同样的题目,$D_n=\alpha D_{n-1}-\gamma\beta D_{n-2}$
\[
\begin{cases}
D_n-x_1D_{n-1}=x_2(D_{n-1}-x_1D_{n-2})=x_2^{n-2}(D_2-x_1D_1)\\
D_n-x_2D_{n-1}=x_1(D_{n-1}-x_2D_{n-2})=x_1^{n-2}(D_2-x_2D_1)
\end{cases}
\]
这个时候就要解出$x_1,x_2,D_1,D_2$
\end{description}

\item 线性方程组
\[
\begin{cases}
a_{11}x_1 + a_{12}x_2 + \cdots + a_{1n}x_n =0\\
a_{21}x_1 + a_{22}x_2 + \cdots + a_{2n}x_n =0\\
\vdots \\
a_{n-1,1}x_1 + a_{n-1,2}x_2 + \cdots + a_{n-1,n}x_n =0\\
\end{cases}
\]
的系数矩阵为$A$,设$M_j(j=1,2,\cdots,n)$是在矩阵$A$中划去第$j$列所得到的$n-1$阶子式,证
\begin{description}
\item[(1)] $(M_1,-M_2,\cdots,(-1)^{n-1}M_n)$是方程组的一个解
\item[(2)] 如果$A$的秩为$n-1$,那么方程组的解全是$(M_1,-M_2,\cdots,(-1)^{n-1}M_n)$的倍数
\item[证(1)] 和2016年的题目很像啊。在这题,不是随便加一行,然后按该行展开。而是要确保加了这一行,变成方阵后的行列式值为零
\item[证(2)] 解空间的维数为1,而$(M_1,-M_2,\cdots,(-1)^{n-1}M_n)$属于那个解空间
\end{description}

\item 若$\alpha$为一实数,计算
\[
\lim_{n \to +\infty}\left(
\begin{array}{cc}
1 & \frac{\alpha}{n}\\
\frac{\alpha}{n} & 1 \\
\end{array}
\right)^n
\]
\begin{description}
\item[解] 分$\alpha=0$或者$\alpha\neq 0$,下面讨论$\alpha\neq 0$。$|\lambda E-A|=(\lambda-1)^2-\frac{\alpha^2}{n^2}=0$,解得$\lambda=1 \pm \frac{\alpha}{n}$
\[
P^{-1}AP=\left(
\begin{array}{cc}
1+\frac{\alpha}{n} & \\
 & 1-\frac{\alpha}{n} \\
\end{array}
\right),
P=\frac{1}{\sqrt{2}}\left(
\begin{array}{cc}
-1 & 1\\
1 & 1\\
\end{array}
\right)
\]
注意这里有数分一个极限
\begin{eqnarray*}
\lim_{n \to +\infty} A^n &=& \lim_{n \to +\infty} PA^nP^{-1}\\
&=& \frac{1}{2} \left(
\begin{array}{cc}
-1 & 1\\
1 & 1\\
\end{array}
\right)
\left(
\begin{array}{cc}
e^\alpha & \\
 & e^{-\alpha} \\
 \end{array}
\right)
\left(
\begin{array}{cc}
-1 & 1\\
1 & 1\\
\end{array}
\right)
\end{eqnarray*}
\end{description}


\item 设$\alpha$为实数
\[
A=\left(
\begin{array}{cccc}
a & 1 & & \\
 & a & \ddots & \\
 & & \ddots & 1 \\
 & & & a
\end{array}
\right) \in \mathbf{R}^{100 \times 100}
\]
求$A^{50}$第一行元素之和
\begin{description}
\item[解] 极小多项式然后取模。但是这里不行,只能找规律。二项式展开,然后再收回到矩阵里面来,再数学归纳法
\end{description}


\item 若向量组$\alpha_1,\alpha_2,\cdots,\alpha_s(s>2)$线性无关,讨论$\alpha_1+\alpha_2,\alpha_2+\alpha_3,\cdots,\alpha_{s-1}+\alpha_s,\alpha_s+\alpha_1$线性相关性
\begin{description}
\item[解]
\[
(\alpha_1,\alpha_2,\cdots,\alpha_s)H=(\alpha_1+\alpha_2,\alpha_2+\alpha_3,\cdots,\alpha_{s-1}+\alpha_s,\alpha_s+\alpha_1)
\]
$H$长啥样会求吧,$\rank(H)$决定后者线性相关性,其实这里是不是要说一下。假定线性相关,存在不全为零的$k_1,\cdots,k_n$,使得$k_1(\alpha_1+\alpha_2)+k_2(\alpha_2+\alpha_3)+\cdots+k_{s-1}(\alpha_{s-1}+\alpha_s)+k_s(\alpha_s+\alpha_1)=0$,又$\alpha_1,\alpha_2,\cdots,\alpha_s(s>2)$线性无关,$k_1+k_2=\cdots=k_n+k_1=0$
\[
\left(
\begin{array}{ccccc}
1 & 1 & 0 & \cdots & 0\\
0 & 1 & 1 & \cdots & 0\\
0 & 0 & 1 & \cdots & 0\\
\vdots & \vdots & \vdots & \ddots & \vdots\\
1 & 0 & 0 & \cdots & 1\\
\end{array}
\right)
\left(
\begin{array}{c}
k_1\\
\vdots\\
k_n\\
\end{array}
\right)=0
\]
\end{description}

\item 已知二次曲面方程$x^2+ay^2+z^2+2bxy+2xz+2yz=4$可以经正交变换
\[
\left(
\begin{array}{c}
x \\
y\\
z
\end{array}
\right) = P \left(
\begin{array}{c}
x' \\
y' \\
z'
\end{array}
\right)
\]
化为椭圆柱面方程$y'^2+4z'^2=4$,求$a,b$的值和正交矩阵$P$
\begin{description}
\item[解] 看迹马上知道$a=3$,因为正交,直接求行列式可得$b=0$
\end{description}

\item 设有实二次型$f(x)=x^TAx$,其中$A$是$3\times 3$实对称矩阵并满足$A^3-6A^2+11A-6I=0$,计算
\[
\max_{A}\max_{\|x\|=1}f(x)
\]
其中$\|x\|^2=x_1^2+x_2^2+x_3^2$,第一个极大值是对满足以上方程的所有实对称矩阵$A$来求
\begin{description}
\item[解] 首先$\|x\|=1$是单位向量,其次特征向量对矩阵多项式不变。找到最大特征值即可
\end{description}

\item $A\in \mathbf{R}^{2006 \times 2006}$是给定的幂零矩阵(即$\exists p\in \mathbf{N}^+$使得$A^p=0$而$A^{p-1}\neq 0$),分析$Ax=0$非零独立解个数的最大值和最小值
\begin{description}
\item[解] 说白了是求$\rank(A)$的范围。幂零矩阵至少有一个若当块,$\rank(A)\geq p-1$。另一方面,请使用Sylvester秩不等式,见高等代数上4.5例2
\end{description}

\item 设$f$是有限维向量空间$V$上的线性变换,且$f^n$是$V$上的恒等变换,$n$是某个正整数。设$W=\{v\in V|f(v)=v\}$。证明$W$是$V$的一个子空间,并且其维数等于线性变换$\frac{f+f^2+\cdots+f^n}{n}$的迹
\begin{description}
\item[解] 首先考虑几个问题,子空间的定义要背,在某组基下,矩阵为$A$,则$A^n-I=0$是零化多项式。另外零化多项式包含所有特征根,就是那$n$个复根,每个根都不同且为一次,可以判断是最小多项式,并且 相似于对角形。然后$W$其实是$\ker(A-I)$,后面都用对角的话超级快
\end{description}




























\end{itemize}