\begin{itemize}
\item 设四元齐次方程组$(1)$
\[
\begin{cases}
x_1 + x_3 = 0 \\
x_2 - x_4 = 0
\end{cases}
\]
又知某线性齐次方程组$(2)$的通解为$k_1(0,1,1,0)^T+k_2(-1,2,2,1)^T$
\begin{description}
\item[(1)] 求线性方程组$(1)$的基础解系
\item[(2)] 问线性方程组$(1)$和$(2)$是否有非零公共解?若有,求出所有的非零公共解,若没有,说明理由
\item[解(1)]
\[
\left(
\begin{array}{c}
-1 \\
0 \\
1 \\
0 \\
\end{array}
\right) \left(
\begin{array}{c}
0 \\
1 \\
0 \\
1 \\
\end{array}
\right)
\]
\item[解(2)] $k_1(0,1,1,0)^T+k_2(-1,2,2,1)^T$ = $k_3(-1,0,1,0)^T+k_4(0,1,0,1)^T$,把$(k_1,k_2,k_3,k_4)^T$解出来得到通解即可
\end{description}

\item 给定两个四维向量$\alpha_1=(\frac{1}{3},-\frac{2}{3},0,\frac{2}{3})^T,\alpha_2=(-\frac{2}{\sqrt{6}},0,\frac{1}{\sqrt{6}},\frac{1}{\sqrt{6}})^T$。求作一个四阶正交矩阵$Q$,以$\alpha_1,\alpha_2$作为它的前两个列向量
\begin{description}
\item[解]
\[
A = \left(
\begin{array}{cccc}
\frac{1}{3}&-\frac{2}{3}&0&\frac{2}{3}\\
-\frac{2}{\sqrt{6}}&0&\frac{1}{\sqrt{6}}&\frac{1}{\sqrt{6}}\\
\end{array}
\right) \Rightarrow \left(
\begin{array}{cccc}
2&0&-1&-1\\
0&-4&1&5\\
\end{array}
\right)
\]
剩下的就是解方程然后施密斯正交化,请注意,解出来的第一个向量只用单位化,另一个才开始和前一个正交化
\end{description}

\item 求矩阵
\[
A=\left(
\begin{array}{cccc}
0 & 1 & 1 & 1\\
0 & 0 & 1 & 1\\
0 & 0 & 0 & 1\\
0 & 0 & 0 & 0\\
\end{array}
\right)
\]
的若当标准型,并计算$e^A(e^A=I+A+\frac{A^2}{2!}+\frac{A^3}{3!}+\cdots)$
\begin{description}
\item[解] 可见高等代数下9.8的例1,那里较怎样用$\rank(\lambda E-A)$来求若当块
\end{description}

\item 设
\[
B=\left(
\begin{array}{ccc}
4 & 4.5 & -1\\
-3 & -3.5 & 1 \\
-2 & -3 & 1.5
\end{array}
\right)
\]
求$B^{2005}$(精确到小数点后4位)
\begin{description}
\item[解] 同上,见高等代数下9.8的例1,这题有技巧,都是$\frac{1}{2}$,整个矩阵乘2
\end{description}

\item 证明函数$\log \det(\cdot)$在对称正定矩阵集上是凹函数,即对于任意两个$n\times n$对称正定矩阵$A,B$及$\forall \lambda \in [0,1]$,有
\[
\log \det(\lambda A+(1-\lambda)B)\geq \lambda \log\det(A)+(1-\lambda)\log\det(B)
\]
\begin{description}
\item[解] 把$A$变成$I$,把$B$变成$\diag{b_i}$,然后剩下的就是$\ln(x)$的运算和凸性
\end{description}

\item 考虑
\[
P=\left(
\begin{array}{cccc}
a_1^2 & a_1a_2 & \cdots &a_1a_n\\
a_2a_1 & a_2^2 & \cdots &a_2a_n\\
\vdots & \vdots & \ddots & \vdots \\
a_na_1 & a_na_2 & \cdots &a_n^2\\
\end{array}
\right)
\]
其中$a_i,1\leq i \leq n$都为实数
\begin{description}
\item[(1)] 证$P$半正定
\item[(2)] 证非零二次型$f(x_1,\cdots,x_n)$可以写成$f(x_1,\cdots,x_n)=(u_1x_1+\cdots+u_nx_n)(v_1x_1+\cdots+v_nx_n)$的充要条件是它的秩为1或秩为2且符号差为0
\item[证(1)] $P=(a_1,a_2,\cdots,a_n)^T(a_1,a_2,\cdots,a_n)$,半正定有个充要性质,高等代数6.3定理5
\item[证(2)] $\Leftarrow$证比较简单,把矩阵写出来,下面证$\Rightarrow$。当$(u_1,\cdots,u_n)$和$(v_1,\cdots,v_n)$相关,令$y=(u_1x_1+\cdots+u_nx_n)$,则$f=ky^2$。当$(u_1,\cdots,u_n)$和$(v_1,\cdots,v_n)$不相关,$(u,v)$的一个极大线性无关向量组为
\[
\left(
\begin{array}{cc}
u_1 & v_1 \\
u_2 & v_2 \\
\end{array}
\right)
\]
令
\[
\begin{cases}
y_1 = u_1x_1+\cdots+u_nx_n \\
y_2 = v_1x_1+\cdots+v_nx_n \\
y_3 = x_3\\
\vdots
\end{cases}
\]
此时$f=y_1y_2$
再令
\[
\begin{cases}
x_1 = z_1+z_2 \\
y_2 = z_1-z_2 \\
\vdots
\end{cases}
\]
\end{description}

\item 证
\begin{description}
\item[(1)] 任何$n$阶实对称方阵$A$必合同于对角阵$D=\diag\{\delta_1,\delta_2,\cdots,\delta_n\}$,即存在$n$阶非奇异实方阵$C$使得$C^TAC=D$,其中$\delta_i=-1$或0或1
\item[(2)] 任何$n$阶实反对称方阵$B$必为偶数阶(即$n=2k$),且合同于块对角阵$F=\diag\{J_1,J_2,\cdots,J_k\}$,即存在$n$阶非奇异实方阵$E$使得$E^TBE=F$,其中
\[
J_i=\left(
\begin{array}{cc}
0 & -1 \\
1 & 0
\end{array}
\right)
\]
\item[(3)] 对迹为$0$的$n$阶实方阵$G$,存在实正交阵$H$,使得$H^TGH$的主对角元全为零
\item[证] (1)(2)(3)都是归纳法
\end{description}


\item 求$7$次多项式$f(x)$,使$f(x)+1$能被$(x-1)^4$整除,而$f(x)-1$能被$(x+1)^4$整除
\begin{description}
\item[解] $f'(x)=a(x^3-1)(x^3+1),f(1)=-1,f(-1)=1$
\end{description}

\item 给定一单调递减序列$b_1 > b_2 > \cdots > b_p >0$,定义
\[
\beta = \left(p!\frac{p}{p-1} \right)^{\frac{1}{\min_{1\leq k\leq p-1}(b_k-b_{k+1})}}
\]
假设复数$a_i,i=1,2,\cdots,p$满足$|a_i|>\beta|a_{i+1}|,i=1,2,\cdots,p-1$且$|a_p|\geq 1$。证
\[
D=\left[
\begin{array}{cccc}
a_1^{b_1} & a_1^{b_2} & \cdots & a_1^{b_p} \\
a_2^{b_1} & a_2^{b_2} & \cdots & a_2^{b_p} \\
\vdots & \vdots & \ddots & \vdots \\
a_p^{b_1} & a_p^{b_2} & \cdots & a_p^{b_p} \\
\end{array}
\right]
\]
其绝对值有上下界如下
\[
\frac{1}{p}\prod_{i=1}^p|a_i|^{b_i}<|D|<2\prod_{i=1}^p|a_i|^{b_i}
\]


















\end{itemize}